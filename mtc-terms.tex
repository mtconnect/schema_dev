

\newglossaryentry{abstimeseries}
{
  type=mtc,
  category=model,
  name= {AbsTimeSeries},
  description= {It is an abstract type element and will be replaced in the \glspl{mtconnectstream} document by the element name derived from the \gls{type} attribute defined for the associated \gls{dataitem} element defined in the \glspl{mtconnectdevice} document}
}


\newglossaryentry{abstractconfiguration}
{
  type=mtc,
  name= {AbstractConfiguration},
  category=model,
  kind={configuration},
  description= {It is an abstract type XML element.  It will never appear in the XML document representing a piece of equipment. }
}

\newglossaryentry{actuator}
{
  type=mtc,
  category=model,
  name={Actuator},
  kind={component},
  description={Redefined as a piece of equipment with the ability to be represented as a Lower Level component of a parent Component element or as a Composition element. See \gls{actuator type}}
}

\newglossaryentry{actual subtype}
{
  type= mtc,
  category=model,
  name={ACTUAL},
  kind={subtype},
  description={The measured value of the data item type given by a sensor or encoder.}
}

\newglossaryentry{auxiliaries}
{
  type=mtc,
  category=model,
  name= {Auxiliaries},
  kind={component},
  description={An XML container used to organize information for \gls{lower level} elements representing functional sub-systems that provide supplementary or extended capabilities for a piece of equipment, but they are not required for the basic operation of the equipment.}
}



\newglossaryentry{axes}
{
  type=mtc,
  category=model,
  name= {Axes},
  kind={component},
  description={An XML container used to organize the \glspl{structural element} of a piece of equipment that perform linear or rotational motion.}
}


\newglossaryentry{electric}
{
  type=mtc,
  category=model,
  name={Electric},
  kind={systems,component},
  description={\gls{electric} is an XML container that represents the information for the main power supply for device piece of equipment and the distribution of that power throughout the equipment. }
}


\newglossaryentry{loader}
{
  type=mtc,
  category=model,
  name={Loader},
  kind={auxiliaries,component},
  description={\gls{loader} is an XML container that represents the information for a unit comprised of all the parts involved in moving and distributing materials, parts, tooling, and other items to or from a piece of equipment.}
}


\newglossaryentry{wastedisposal}
{
  type=mtc,
  category=model,
  name={WasteDisposal},
  kind={auxiliaries,component},
  description={\gls{wastedisposal} is an XML container that represents the information for a unit comprised of all the parts involved in removing manufacturing byproducts from a piece of equipment.
}
}


\newglossaryentry{toolingdelivery}
{
  type=mtc,
  category=model,
  name={ToolingDelivery},
  kind={auxiliaries,component},
  description={\gls{toolingdelivery} is an XML container that represents the information for a unit involved in managing, positioning, storing, and delivering tooling within a piece of equipment.
}
}


\newglossaryentry{environmental}
{
  type=mtc,
  category=model,
  name={Environmental},
  kind={auxiliaries,component},
  description={\gls{environmental} is an XML container that represents the information for a unit or function involved in monitoring, managing, or conditioning the environment around or within a piece of equipment.}
}


\newglossaryentry{barfeeder}
{
  type=mtc,
  category=model,
  name={BarFeeder},
  kind={auxiliaries,component},
  description={\gls{barfeeder} is an XML container that represents the information for a unit involved in delivering bar stock to a piece of equipment.}
}


\newglossaryentry{units}
{
  type=mtc,
  category=model,
  name={units},
  kind={attribute},
  description={The unit of measurement for the reported value of the data item.}
}



\newglossaryentry{buffersize}
{
  type=mtc,
  category=model,
  name={bufferSize},
  kind={attribute},
  description={A value representing the maximum number of \glspl{data entity} that \MAY be retained in the \gls{agent} that published the \gls{response document} at any point in time.}
}


\newglossaryentry{calibrationdate}
{
  type=mtc,
  category=model,
  name={CalibrationDate},
  kind={element},
  description={Date upon which the \gls{sensor unit} was last calibrated. }
}


\newglossaryentry{nextcalibrationdate}
{
  type=mtc,
  category=model,
  name={NextCalibrationDate},
  kind={element},
  description={Date upon which the sensor unit is next scheduled to be calibrated. }
}


\newglossaryentry{calibrationinitials}
{
  type=mtc,
  category=model,
  name={CalibrationInitials},
  kind={element},
  description={The initials of the person verifying the validity of the calibration data.}
}


\newglossaryentry{category}
{
  type=mtc,
  category=model,
  name={category},
  kind={attribute},
  description={Specifies the kind of information provided by a data item. }
}


\newglossaryentry{channel}
{
  type=mtc,
  category=model,
  name={Channel},
  kind={element},
  attributes={\gls{number},\gls{name}},
  elements={\gls{description},\gls{calibrationdate},\gls{nextcalibrationdate},\gls{calibrationinitials}},
  description={\gls{channel} represents each \gls{sensing element} connected to a \gls{sensor unit}.}
}

\newglossaryentry{channels}
{
  type=mtc,
  category=model,
  name={Channels},
  kind={element},
  elements={\gls{channel}},
  description={When \gls{sensor} represents multiple \glspl{sensing element}, each \gls{sensing element} is represented by a \gls{channel} for the \gls{sensor}. \gls{channels} is an XML container used to organize information for the \glspl{sensing element}. }
}


\newglossaryentry{character data}
{
  type=mtc,
  category=model,
  name={CharacterData},
  description={See \gls{cdata}}
}


\newglossaryentry{components}
{
  type=mtc,
  category=model,
  name={Components},
  elements={\gls{device}},
  kind={element},
  description={XML container that consists of one or more types of \gls{component} XML elements. } 
}


\newglossaryentry{component}
{
  type=mtc,
  category=model,
  name={Component},
  kind={element},
  attributes={\gls{id},\gls{nativename},\gls{sampleinterval},\gls{uuid},\gls{name}},
  elements={\gls{description},\gls{configuration},\gls{dataitems},\gls{components},\glspl{composition},\gls{references}},
  description={An abstract XML element. Replaced in the XML document by types of \gls{component} elements representing physical parts and logical functions of a piece of equipment.}
}


\newglossaryentry{component componentstream}
{
  type=mtc,
  category=model,
  name={component},
  description={\gls{component componentstream} identifies the \gls{structural element} (\gls{device}, \gls{top level} \gls{component}, or \gls{lower level} \gls{component}) associated with the \gls{componentstream} element.}
}


\newglossaryentry{componentid}
{
  type=mtc,
  category=model,
  name={componentId},
  kind={attribute},
  description={The identifier attribute of the \gls{component} element that represents the physical part of a piece of equipment where the data represented by the \gls{dataitem} element originated.}
}


\newglossaryentry{componentref}
{
  type=mtc,
  category=model,
  name={ComponentRef},
  kind={reference},
  description={\gls{componentref} XML element is a pointer to all of the information associated with another \gls{structural element} defined elsewhere in the XML document for a piece of equipment. } 
}


\newglossaryentry{componentstream}
{
  type=mtc,
  category=model,
  name={ComponentStream},
  description={A XML container type element that organizes data returned from an MTConnect Agent in response to a current or sample HTTP request.} 
}


\newglossaryentry{composition}
{
  type=mtc,
  category=model,
  name={Composition},
  kind={element},
  attributes={\gls{id},\gls{uuid},\gls{name},\gls{type}},
  elements={\gls{description}},
  description={An XML element used to describe the lowest level structural building blocks contained within a \gls{component} element.}
}


\newglossaryentry{compositions}
{
  type=mtc,
  category=model,
  name={Compositions},
  kind={element},
  description={XML container consisting of one or more types of \gls{composition} XML elements.},
  elements={\gls{composition}}
}


\newglossaryentry{conditions}
{
  type=mtc,
  category=model,
  name={Condition},
  description={A XML container type element that organizes the data reported in the \glspl{mtconnectstream} document for \gls{dataitem} elements defined in the \glspl{mtconnectdevice} document with a \gls{category} attribute of \gls{condition category}.}
}

\newglossaryentry{condition}
{
  type=mtc,
  category=model,
  name={Condition},
  description={A XML element which provides the information and data reported from a piece of equipment for those \gls{dataitem} elements defined with a \gls{category} attribute of \gls{condition category} in the \glspl{mtconnectdevice} document.}
}


\newglossaryentry{constraint}
{
  type=mtc,
  name={Constraint},
  kind={element},
  category=model,
  description={A \gls{constraint} is used by a software application to evaluate the validity of the reported data.}
}

\newglossaryentry{constraints}
{
  type=mtc,
  name={Constraints},
  kind={element},
  category=model,
  description={\gls{constraints} is an optional container that provides a set of expected values that can be reported for this \gls{dataitem}.},
  elements={\gls{value},\gls{maximum},\gls{minimum},\gls{nominal}}
}


\newglossaryentry{configuration}
{
  type=mtc,
  category=model,
  name={Configuration},
  description={An XML element that contains technical information about a piece of equipment describing its physical layout or functional characteristics.},
  kind={element}
}



\newglossaryentry{current request}
{
  type=mtc,
  name={Current Request},
  description={An \glstext{http} request to the \gls{agent} for returning latest known values for the \gls{dataitem} as an \glspl{mtconnectstream} \glstext{xml} document}
}


\newglossaryentry{current httprequest}
{
  type=mtc,
  category=model,
  name={current},
  description={Used in the path portion of an HTTP Request Line, by a client software application, to initiate a \gls{current request} to an \gls{agent} to publish an \gls{mtconnectstream} document.}
}


\newglossaryentry{dataitem}
{
  type=mtc,
  name={DataItem},
  kind={element},
  attributes={\gls{name},\gls{id},\gls{type},\gls{subtype},\gls{statistic},\gls{units},\gls{nativeunits},\gls{nativescale},\gls{category},\gls{coordinatesystem},\gls{compositionid},\gls{samplerate},\gls{representation},\gls{significantdigits}},
  category=model,
  description={\gls{data entity} describing a piece of information reported about a piece of equipment.}
}


\newglossaryentry{dataitems}
{
  type=mtc,
  name={DataItems},
  kind={element},
  elements={\gls{dataitem}},
  category=model,
  description={XML container consisting of one or more types of \gls{dataitem} XML elements.}
}

\newglossaryentry{dataitemid}
{
  type=mtc,
  category=model,
  name={dataItemId},
  kind={attribute},
  description={The identifier attribute of the \gls{dataitem} that represents the originally measured value of the data referenced by this data item.}
}


\newglossaryentry{dataitemref}
{
  type=mtc,
  category=model,
  name={DataItemRef},
  kind={reference},
  description={\gls{dataitemref} XML element is a pointer to a \gls{data entity} associated with another \gls{structural element} defined elsewhere in the XML document for a piece of equipment.}
}


\newglossaryentry{description}
{
  type=mtc,
  name={Description},
  category=model,
  description={An XML element that can contain any descriptive content.},
  kind={element},
  attributes={\gls{manufacturer},\gls{model},\gls{serialnumber},\gls{station}}
}


\newglossaryentry{device}
{
  type=mtc,
  category=model,
  name={Device},
  description={The primary container element for each piece of equipment. \gls{device} is organized within the \gls{devices}  container.},
  kind={element},
  attributes={\gls{id},\gls{nativename},\gls{sampleinterval},\gls{uuid},\gls{name}},
  elements={\gls{description},\gls{configuration},\gls{dataitems},\gls{components},\glspl{composition},\glspl{reference}}
}

\newglossaryentry{devices}
{
  type=mtc,
  category=model,
  name={Devices},
  description={The first, or highest level, \gls{structural element} in a \glspl{mtconnectdevice} document.},
  elements={\gls{device}},
  kind={element}
}


\newglossaryentry{devicestream}
{
  type=mtc,
  category=model,
  name={DeviceStream},
  description={A XML container element provided in the \gls{streams} container in the \glspl{mtconnectstream} document.}
}



\newglossaryentry{discrete representation}
{
  type=mtc,
  category=model,
  name={DISCRETE},
  kind={representation},
  description={A \gls{data entity} where each discrete occurrence of the data may have the same value as the previous occurrence of the data.}
}


\newglossaryentry{duration}
{
  type=mtc,
  category=model,
  name={duration},
  description={The time-period over which the data was collected.}
}



\newglossaryentry{event}
{
  type=mtc,
  category=model,
  name={Event},
  description={A XML element which provides the information and data reported from a piece of equipment for those \gls{dataitem} elements defined with a \gls{category} attribute of \gls{event category} in the \glspl{mtconnectdevice} document.}
}


\newglossaryentry{events}
{
  type=mtc,
  category=model,
  name={Events},
  description={A XML container type element that organizes the data reported in the \glspl{mtconnectstream} document for \gls{dataitem} elements defined in the \glspl{mtconnectdevice} document with a \gls{category} attribute of \gls{event category}.}
}


\newglossaryentry{filter}
{
  type=mtc,
  category=model,
  kind={element},
  name={Filter},
  description={\gls{filter} provides a means to control when an \gls{agent} records updated information for a data item.}
}

\newglossaryentry{filters}
{
  type=mtc,
  category=model,
  kind={element},
  name={Filters},
  description={An XML container consisting of one or more types of \gls{filter} XML elements.},
  elements={\gls{filter}}
}


\newglossaryentry{firmwareversion}
{
  type=mtc,
  category=model,
  name={FirmwareVersion},
  kind={element},
  description={Version number for the sensor unit as specified by the manufacturer.}
}


\newglossaryentry{header}
{
  type=mtc,
  category=model,
  name={Header},
  description={The \gls{header} section contains protocol related information as defined in Section 6.5.1 of \citetitle{MTCPart1} of the MTConnect Standard.}
}


\newglossaryentry{id}
{
  type=mtc,
  category=model,
  name={id},
  description={The unique identifier for this element.}
}


\newglossaryentry{idref}
{
  type=mtc,
  category=model,
  name={idRef},
  description={A pointer to the \gls{id} attribute of an element that contains the information to be associated with this XML element.}
}


\newglossaryentry{initialvalue}
{
  name={InitialValue},
  category=model,
  type=mtc,
  kind={element},
  description={\gls{initialvalue} is an optional XML element that defines the starting value for a data item as well as the value to be set for the data item after a reset event.}
}


\newglossaryentry{interfaces}
{
  type=mtc,
  name=interfaces,
  description={\citetitle{MTCPart5} provides an interaction model for coordinating activities between manufacturing devices}
}


\newglossaryentry{interface component}
{
  type=mtc,
  category=model,
  name={Interface},
  kind={component},
  description={Each \gls{interface component} contains \glspl{data entity} available from the piece of equipment that may be needed to coordinate activities with associated pieces of equipment.}
}

\newglossaryentry{interfaces component}
{
  type=mtc,
  category=model,
  name={Interfaces},
  kind={component},
  description={An XML container that organizes information used to coordinate actions and activities between pieces of equipment that communicate information between each other. }
}


\newglossaryentry{linear}
{
  type=mtc,
  category=model,
  name={Linear},
  kind={axes,component},
  description={A \gls{linear} axis represents the movement of a physical piece of equipment, or a portion of the equipment, in a straight line. }
}


\newglossaryentry{lower camel case}
{
  type=mtc,
  name={Lower Camel Case},
  description={the first word is lowercase and the remaining words are capitalized and all spaces between words are removed.}
}


\newglossaryentry{lower level}
{
  type=mtc,
  name={Lower Level},
  description={A nested element that is below a higher level element.}
}


\newglossaryentry{machine}
{
  type=mtc,
  category=model,
  name={MACHINE},
  kind={coordinatesystem},
  description={An unchangeable coordinate system that has machine zero as its origin.}
}


\newglossaryentry{manufacturer}
{
  type=mtc,
  category=model,
  name={manufacturer},
  description={The name of the manufacturer of the physical or logical part of a piece of equipment represented by an XML element.}
}


\newglossaryentry{maximum}
{
  type=mtc,
  category=model,
  name={Maximum},
  kind={element},
  description={The upper limit of data reported for a data item.}
}


\newglossaryentry{minimum}
{
  type=mtc,
  category=model,
  name={Minimum},
  kind={element},
  description={The lower limit of data reported for a data item.}
}



\newglossaryentry{minimumdelta}
{
  type=mtc,
  category=model,
  name={MINIMUM\_DELTA},
  kind={filter},
  description={For \gls{filter} \gls{minimumdelta}, a new value \MUSTNOT be reported for a data item unless the measured value has changed from the last reported value by at least the delta given as the \gls{cdata} of this element.}
}


\newglossaryentry{mixes-in}
{
  type=mtc,
  name={<<Mixes In>>},
  description={A software architecture pattern represented as a \gls{stereotype} that combines the properties of one class with another. Similar to a sub-class, it allows for inheritance in single inheritance type systems.}
}


\newglossaryentry{modbus}
{
  type=mtc,
  name=MODBUS,
  description={Modbus is a communication protocol developed by Modicon systems and is a method used for transmitting information over serial lines between electronic devices.}
}


\newglossaryentry{model}
{
  type=mtc,
  category=model,
  name={model},
  description={The model description of the physical part or logical function of a piece of equipment represented by this XML element.}
}


\newglossaryentry{mtconnect}
{
  type=mtc,
  name={MTConnect},
  description={The name of the standard.}
}

\newglossaryentry{mtconnect standard}
{
  type=mtc,
  name={MTConnect Standard},
  description={The name of the standard.}
}

\newglossaryentry{mtconnect asset}
{
  type=mtc,
  name={MTConnect Asset},
  description={See \gls{asset}}
}


\newglossaryentry{mtconnectdevice}
{
  type=mtc,
  category=model,
  name={MTConnectDevice},
  plural={MTConnectDevices},
  description={XML response document.}
}


\newglossaryentry{mtconnectstream}
{
  type=mtc,
  category=model,
  name={MTConnectStream},
  plural={MTConnectStreams},
  description={XML response document}
}


\newglossaryentry{name}
{
  type=mtc,
  category=model,
  name={name},
  kind={attribute},
  description={The name of an element or a piece of equipment.}
}


\newglossaryentry{nativename}
{
  type=mtc,
  category=model,
  name={nativeName},
  description={The common name normally associated with a piece of equipment or an element.}
}


\newglossaryentry{nativescale}
{
  type=mtc,
  category=model,
  name={nativeScale},
  kind={attribute},
  description={\gls{nativescale} \MAY be used to convert the reported value to represent the original measured value.}
}


\newglossaryentry{nativecode}
{
  type=mtc,
  category=model,
  name={nativeCode},
  description={The native code (usually an alpha-numeric value) generated by the controller of a piece of equipment or the element.}
}


\newglossaryentry{nativeseverity}
{
  type=mtc,
  category=model,
  name={nativeSeverity},
  description={If the piece of equipment designates a severity level to a fault, \gls{nativeseverity} reports that severity information to a client software application. }
}


\newglossaryentry{nativeunits}
{
  type=mtc,
  category=model,
  name={nativeUnits},
  description={The native units of measurement for the reported value of the data item.}
}



\newglossaryentry{nominal}
{
  type=mtc,
  category=model,
  name={Nominal},
  kind={element},
  description={The target or expected value for this data item.}
}


\newglossaryentry{occurrence}
{
  name={Occurence},
  description={Occurrence defines the number of times the content defined in the tables \MAY be provided in the usage case specified},
  plural={Occurences}
}


\newglossaryentry{number}
{
  type=mtc,
  category=model,
  name={number},
  description={A unique identifier that will only refer to a specific \gls{sensing element}.}
}


\newglossaryentry{ontology}
{
  type=mtc,
  name=ontology,
  description={logical structure of the terms used to describe a domain of knowledge, including both the definitions of the applicable terms and their relationships ISO 20534:2018}
}


\newglossaryentry{pascal case}
{
  type=mtc,
  name={Pascal Case},
  description= {The first letter of each word is capitalized and the remaining letters are in lowercase. All space is removed between letters}
}


\newglossaryentry{path}
{
  type=mtc,
  category=model,
  name={Path},
  kind={controller,component},
  description= {\gls{path} is an XML container that represents the information for an independent operation or function within a \gls{controller}.}
}


\newglossaryentry{period}
{
  type=mtc,
  category=model,
  name={PERIOD},
  kind={filter},
  description={The data reported for a data item with this \gls{filter} is provided on a periodic basis.}
}



\newglossaryentry{probe httprequest}
{
  type=mtc,
  category=model,
  name={probe},
  description={The form \gls{probe httprequest} is used to designate a \gls{probe request} in the path portion of an HTTP Request Line.}
}


\newglossaryentry{probe request}
{
  type=mtc,
  name={Probe Request},
  description={An \glstext{http} request to the \gls{agent} for returning metadata as an MTConnectDevices \glstext{xml} document}
}


\newglossaryentry{qname}
{
  type=mtc,
  category=model,
  name={QName},
  description={A \gls{qname}, or qualified name, is the fully qualified name of an element, attribute, or identifier in an XML document. A  \gls{qname} concisely associates the URI of an XML namespace with the local name of an element, attribute, or identifier in that namespace.}
}


\newglossaryentry{qualifier}
{
  type=mtc,
  category=model,
  name={qualifier},
  kind={attribute},
  description={\gls{qualifier} provides additional information regarding a Fault State associated with the measured value of a process variable.}
}


\newglossaryentry{realization}
{
  type=mtc,
  name={Realization},
  description={Realization is a specialized abstraction relationship between two sets of model elements, one representing a specification (the supplier) and the other represents an implementation of the latter (the client). Realization can be used to model stepwise refinement, optimizations, transformations, templates, model synthesis, framework composition, etc.}
}


\newglossaryentry{reference}
{
  type=mtc,
  category=model,
  name={Reference},
  kind={element},
  attributes={\gls{idref},\gls{name}},
  description={\gls{reference} is a pointer to information that is associated with another \gls{structural element} defined elsewhere in the XML document for a piece of equipment.}
}


\newglossaryentry{references}
{
  type=mtc,
  category=model,
  name={References},
  kind={element},
  elements={\gls{reference}},
  description={XML container consisting of one or more types of \gls{reference} XML elements.}
}


\newglossaryentry{representation}
{
  type=mtc,
  category=model,
  name={representation},
  kind={attribute},
  description={Description of a means to interpret data consisting of multiple data points or samples reported as a single value.  \newline \gls{representation} is an optional attribute.  \newline \gls{representation} will define a unique format for each set of data.  \newline \gls{representation} for \gls{timeseries representation}, \gls{discrete representation}, and \gls{value representation} are defined below in Section {\color{red} 7.2.2.12}.  \newline If \gls{representation} is not specified, it \MUST be determined to be \gls{value representation}.}
}


\newglossaryentry{resettrigger}
{
  type=mtc,
  category=model,
  name={ResetTrigger},
  kind={element},
  description={\gls{resettrigger} is an optional XML element that identifies the type of event that may cause a reset to occur. It is additional information regarding the meaning of the data that establishes an understanding of the time frame that the data represents so that the data may be correctly understood by a client software application.}
}


\newglossaryentry{resettriggered}
{
  type=mtc,
  category=model,
  name={resetTriggered},
  description={For those \gls{dataitem} elements that report data that may be periodically reset to an initial value, \gls{resettriggered} identifies when a reported value has been reset and what has caused that reset to occur.  \newline resetTriggered is an optional attribute.  \newline \gls{resettriggered} \MUST only be provided for the specific occurrence of a Data Entity reported in the MTConnectStreams document when the reset occurred and \MUSTNOT be provided for any other occurrence of the Data Entity reported in a MTConnectStreams document.}
}


\newglossaryentry{resources}
{
  type=mtc,
  category=model,
  name={Resources},
  kind={component},
  description={An XML container used to organize information for \gls{lower level} elements representing types of items, materials, and personnel that support the operation of a piece of equipment or work to be performed at a location. \gls{resources} also represents materials or other items consumed or transformed by a piece of equipment for production of parts or other types of goods.}
}


\newglossaryentry{power}
{
  type=mtc,
  category=model,
  name={Power},
  kind={component},
  deprecated={true},
  description={\gls{power} was \DEPRECATED in MTConnect Version 1.1 and was replaced by the \gls{data entity} called \gls{availability event}.}
}


\newglossaryentry{materials}
{
  type=mtc,
  category=model,
  name={Materials},
  kind={resources,component},
  description={\gls{materials} is an XML container that provides information about materials or other items consumed or used by the piece of equipment for production of parts, materials, or other types of goods.}
}


\newglossaryentry{stock}
{
  type=mtc,
  category=model,
  name={Stock},
  kind={materials,component},
  description={Stock is an XML container that represents the information for the material that is used in a manufacturing process and to which work is applied in a machine or piece of equipment to produce parts.}
}


\newglossaryentry{personnel}
{
  type=mtc,
  category=model,
  name={Personnel},
  kind={resources,component},
  description={\gls{personnel} is an XML container that provides information about an individual or individuals who either control, support, or otherwise interface with a piece of equipment.
}
}


\newglossaryentry{response}
{
  type=mtc,
  name={Response},
  description={A response \gls{interface} which responds to a \gls{request}.}
}


\newglossaryentry{rotary}
{
  type=mtc,
  category=model,
  name={Rotary},
  kind={axes,component},
  description={A \gls{rotary} axis represents any non-linear or rotary movement of a physical piece of equipment or a portion of the equipment. }
}


\newglossaryentry{sample}
{
  type=mtc,
  category=model,
  name={Sample},
  description={A XML element that provides the information and data reported from a piece of equipment for those \gls{dataitem} elements defined with a \gls{category} attribute of \gls{sample category} in the \glspl{mtconnectdevice} document. }
}


\newglossaryentry{samples}
{
  type=mtc,
  category=model,
  name={Samples},
  description={A XML container type element that organizes the data reported in the \glspl{mtconnectstream} document for \gls{dataitem} elements defined in the \glspl{mtconnectdevice} document with a \gls{category} attribute of \gls{sample category}.}
}


\newglossaryentry{sample category}
{
  type=mtc,
  category=model,
  name={SAMPLE},
  kind={category},
  description={A \gls{sample category} is the reading of the value of a continuously variable or analog data value.}
}


\newglossaryentry{sample httprequest}
{
  type=mtc,
  category=model,
  name={sample},
  description={Used in the path portion of an HTTP Request Line, by a client software application, to initiate a \gls{sample request} to an \gls{agent} to publish an \gls{mtconnectstream} document.}
}


\newglossaryentry{samplecount}
{
  type=mtc,
  category=model,
  name={sampleCount},
  kind={attribute},
  description={The number of readings reported in the data returned for the \gls{dataitem} element defined in the \glspl{mtconnectdevice} document that this \gls{sample} element represents.}
}


\newglossaryentry{sampleinterval}
{
  type=mtc,
  category=model,
  name={sampleInterval},
  kind={attribute},
  description={An optional attribute that is an indication provided by a piece of equipment describing the interval in milliseconds between the completion of the reading of the data associated with the \gls{device} element until the beginning of the next sampling of that data.}
}


\newglossaryentry{sample request}
{
  type=mtc,
  name={Sample Request},
  description= {A request from the \gls{agent} for a stream of time series data.}
}


\newglossaryentry{samplerate}
{
  type=mtc,
  category=model,
  name={sampleRate},
  kind={attribute},
  description={The rate at which successive samples of a data item are recorded by a piece of equipment.}
}


\newglossaryentry{sensing element}
{
  type=mtc,
  name={sensing element},
  description={A mechanism that provides a signal or measured value.},
  plural={sensing elements}
}


\newglossaryentry{sensor element}
{
  type=mtc,
  name={sensor element},
  plural={sensor elements},
  description={A \gls{sensor element} provides a signal or measured value.}
}


\newglossaryentry{sensor term}
{
  type=mtc,
  name={Sensor},
  plural={Sensors},
  description= {A \gls{sensor term} is typically comprised of two major components: a \gls{sensor unit} that provides signal processing, conversion, and communications and the \glspl{sensing element} that provides a signal or measured value.}
}


\newglossaryentry{sensor}
{
  type=mtc,
  category=model,
  name={Sensor},
  kind={auxiliaries,component},
  plural={Sensors},
  description= {The \gls{sensor unit} is modeled as a \gls{lower level} \gls{component} called \gls{sensor}.}
}


\newglossaryentry{sensor unit}
{
  type=mtc,
  name={sensor unit},
  plural={sensor units},
  description= {A \gls{sensor unit} provides signal processing, conversion, and communications.}
}


\newglossaryentry{sensorconfiguration}
{
  type=mtc,
  name={SensorConfiguration},
  kind={configuration},
  elements={\gls{firmwareversion},\gls{calibrationdate},\gls{nextcalibrationdate},\gls{calibrationinitials},\gls{channels}},
  description= {An element that can contain descriptive content defining the configuration information for \gls{sensor}.}
}


\newglossaryentry{serialnumber}
{
  type=mtc,
  category=model,
  name={serialNumber},
  kind={attribute},
  description={The serial number associated with a piece of equipment. }
}


\newglossaryentry{sequence}
{
  type=mtc,
  category=model,
  name={sequence},
  description={A number representing the sequential position of an occurrence of a \gls{category} type in the data buffer of an \gls{agent}. }
}


\newglossaryentry{significantdigits}
{
  type=mtc,
  category=model,
  name={significantDigits},
  kind={attribute},
  description={The number of significant digits in the reported value.}
}


\newglossaryentry{slot}
{
  type=mtc,
  name=Slot,
  description={Slot is UML element which specifies that an instance has a value or values for a specific structural feature. UML specification also says that a slot gives the value or values of a structural feature of the instance. An instance can have one slot per structural feature of its classifiers, including inherited features.}
}


\newglossaryentry{station}
{
  type=mtc,
  category=model,
  name={station},
  plural={stations},
  kind={attribute},
  description={The station where the physical part or logical function of a piece of equipment is located when it is part of a manufacturing unit or cell with multiple stations.}
}


\newglossaryentry{source}
{
  type=mtc,
  category=model,
  name={Source},
  kind={element},
  attributes={\gls{componentid},\gls{dataitemid},\gls{compositionid}},
  description={\gls{source} identifies the \gls{structural element} from which a measured value originates.}
}


\newglossaryentry{statistic}
{
  type=mtc,
  category=model,
  name={statistic},
  kind={attribute},
  description={Describes the type of statistical calculation performed on a series of data samples to provide the reported data value.}
}


\newglossaryentry{stereotype}
{
  type=mtc,
  name=stereotype,
  description={A profile class which defines how an existing metaclass may be extended as part of a profile. It enables the use of a platform or domain specific terminology or notation in place of, or in addition to, the ones used for the extended metaclass. A stereotype is denoted by <<[name]>>.}
}


\newglossaryentry{streams}
{
  type=mtc,
  category=model,
  name={Streams},
  description={The first, or highest, level XML container element in an \glspl{mtconnectstream} \gls{response} Document provided by an \gls{agent} in response to a \gls{sample httprequest} or \gls{current httprequest} HTTP \gls{request}.}
}


\newglossaryentry{subtype}
{
  type=mtc,
  category=model,
  name={subType},
  description={A sub-categorization of the data item \gls{type}.}
}


\newglossaryentry{systems}
{
  type=mtc,
  category=model,
  name={Systems},
  kind={component},
  description={An XML container used to organize information for \gls{lower level} elements representing the major sub-systems that are permanently integrated into a piece of equipment.}
}


\newglossaryentry{time series}
{
  type=mtc,
  name={Time Series},
  description={A \gls{dataitem} representation of a contiguous vector of values supporting high frequency data rates}
}


\newglossaryentry{timeseries representation}
{
  type=mtc,
  category=model,
  name={TIME\_SERIES},
  kind={representation},
  description={A series of sampled data. }
}


\newglossaryentry{timestamp}
{
  type=mtc,
  category=model,
  name={timestamp},
  description={The most accurate time available to a piece of equipment that represents the point in time that the data was reported.}
}


\newglossaryentry{top level}
{
  type=mtc,
  name={Top Level},
  description={\glspl{structural element} that represent the most significant physical or logical functions of a piece of equipment.}
}


\newglossaryentry{type}
{
  type=mtc,
  category=model,
  name={type},
  kind={attribute},
  plural={types},
  description={The type of either a \gls{structural element} or a \gls{dataitem} being measured.}
}


\newglossaryentry{unavailable value}
{
  type=mtc,
  category=model,
  name={UNAVAILABLE},
  kind={enum},
  description={The value of the \gls{data entity} either when the data is not received or the entity is incapable of providing data.}
}


\newglossaryentry{uuid}
{
  type=mtc,
  category=model,
  name={uuid},
  description={The unique identifier for an XML element.}
}



\newglossaryentry{value}
{
  type=mtc,
  category=model,
  name={Value},
  kind={element},
  description={\gls{value} represents a single data value that is expected to be reported for a \gls{dataitem} element. }
}


\newglossaryentry{value representation}
{
  type=mtc,
  category=model,
  name={VALUE},
  kind={representation},
  description={The measured value of the sample data.}
}


\newglossaryentry{work}
{
  type=mtc,
  category=model,
  name={WORK},
  kind={coordinatesystem},
  description={The coordinate system that represents the working area for a particular workpiece whose origin is shifted within the \gls{machine} coordinate system. If the \gls{work} coordinates are not currently defined in the piece of equipment, the \gls{machine} coordinates will be used.}
}


\newglossaryentry{xs:lang}
{
  type=mtc,
  category=model,
  name={xs:lang},
  kind={attribute},
  description={An optional attribute that specifies the language of the \gls{cdata} returned for the \gls{condition}.}
}


\newglossaryentry{controller}
{
  type=mtc,
  category=model,
  name={Controller},
  kind={component},
  description= {An XML container used to organize information about an intelligent or computational function within a piece of equipment.}
}


\newglossaryentry{coordinatesystem}
{
  type=mtc,
  category=model,
  name={coordinateSystem},
  kind={attribute},
  description={For measured values relative to a coordinate system like \gls{position sample}, the coordinate system being used may be reported.}
}


\newglossaryentry{actuator type}
{
  type=mtc,
  category=model,
  name={ACTUATOR},
  kind={composition,type,condition},
  description={A mechanism for moving or controlling a mechanical part of a piece of equipment.   \newline It takes energy usually provided by air, electric current, or liquid and converts the energy into some kind of motion. }
}


\newglossaryentry{amplifier}
{
  type=mtc,
  category=model,
  name={AMPLIFIER},
  kind={composition},
  description={An electronic component or circuit for amplifying power, electric current, or voltage.}
}


\newglossaryentry{ballscrew}
{
  type=mtc,
  category=model,
  name={BALLSCREW},
  kind={composition},
  description={A mechanical structure for transforming rotary motion into linear motion.}
}


\newglossaryentry{belt}
{
  type=mtc,
  category=model,
  name={BELT},
  kind={composition},
  description={An endless flexible band used to transmit motion for a piece of equipment or to convey materials and objects.}
}


\newglossaryentry{brake}
{
  type=mtc,
  category=model,
  name={BRAKE},
  kind={composition},
  description={A mechanism for slowing or stopping a moving object by the absorption or transfer of the energy of momentum, usually by means of friction, electrical force, or magnetic force.}
}


\newglossaryentry{chain}
{
  type=mtc,
  category=model,
  name={CHAIN},
  kind={composition},
  description={An interconnected series of objects that band together and are used to transmit motion for a piece of equipment or to convey materials and objects.}
}


\newglossaryentry{chopper}
{
  type=mtc,
  category=model,
  name={CHOPPER},
  kind={composition},
  description={A mechanism used to break material into smaller pieces.}
}


\newglossaryentry{chuck}
{
  type=mtc,
  category=model,
  name={CHUCK},
  kind={composition},
  description={A mechanism that holds a part, stock material, or any other item in place.}
}


\newglossaryentry{chuck component}
{
  type=mtc,
  category=model,
  name={Chuck},
  kind={rotary,component},
  description={Chuck is an XML container that provides the information about a mechanism that holds a part or stock material in place.}
}


\newglossaryentry{chute}
{
  type=mtc,
  category=model,
  name={CHUTE},
  kind={composition},
  description={An inclined channel for conveying material.}
}


\newglossaryentry{circuitbreaker}
{
  type=mtc,
  category=model,
  name={CIRCUIT\_BREAKER},
  kind={composition},
  description={A mechanism for interrupting an electric circuit.}
}


\newglossaryentry{clamp}
{
  type=mtc,
  category=model,
  name={CLAMP},
  kind={composition},
  description={A mechanism used to strengthen, support, or fasten objects in place.}
}


\newglossaryentry{compressor}
{
  type=mtc,
  category=model,
  name={COMPRESSOR},
  kind={composition},
  description={A pump or other mechanism for reducing volume and increasing pressure of gases in order to condense the gases to drive pneumatically powered pieces of equipment.}
}


\newglossaryentry{door}
{
  type=mtc,
  category=model,
  name={DOOR},
  kind={composition},
  description={A mechanical mechanism or closure that can cover a physical access portal into a piece of equipment allowing or restricting access to other parts of the equipment.}
}


\newglossaryentry{door component}
{
  type=mtc,
  category=model,
  name={Door},
  kind={component},
  description={\gls{door component} is an XML container that represents the information for a mechanical mechanism or closure that can cover.}
}


\newglossaryentry{drain}
{
  type=mtc,
  category=model,
  name={DRAIN},
  kind={composition},
  description={A mechanism that allows material to flow for the purpose of drainage from, for example, a vessel or tank.}
}


\newglossaryentry{encoder}
{
  type=mtc,
  category=model,
  name={ENCODER},
  kind={composition},
  description={A mechanism used to measure rotary position.}
}


\newglossaryentry{fan}
{
  type=mtc,
  category=model,
  name={FAN},
  kind={composition},
  description={Any mechanism for producing a current of air.}
}


\newglossaryentry{filter type}
{
  type=mtc,
  category=model,
  name={FILTER},
  kind={composition},
  description={Any substance or structure through which liquids or gases are passed to remove suspended impurities or to recover solids.}
}


\newglossaryentry{gripper}
{
  type=mtc,
  category=model,
  name={GRIPPER},
  kind={composition},
  description={A mechanism that holds a part, stock material, or any other item in place.}
}


\newglossaryentry{hopper}
{
  type=mtc,
  category=model,
  name={HOPPER},
  kind={composition},
  description={A chamber or bin in which materials are stored temporarily, typically being filled through the top and dispensed through the bottom.}
}


\newglossaryentry{hydraulic}
{
  type=mtc,
  category=model,
  name={Hydraulic},
  kind={systems,component},
  description={\gls{hydraulic} is an XML container that represents the information for a system comprised of all the parts involved in moving and distributing pressurized liquid throughout the piece of equipment.}
}


\newglossaryentry{pneumatic}
{
  type=mtc,
  category=model,
  name={Pneumatic},
  kind={systems,component},
  description={\gls{pneumatic} is an XML container that represents the information for a system comprised of all the parts involved in moving and distributing pressurized gas throughout the piece of equipment.}
}


\newglossaryentry{coolant}
{
  type=mtc,
  category=model,
  name={Coolant},
  kind={systems,component},
  description={\gls{coolant} is an XML container that represents the information for a system comprised of all the parts involved in distribution and management of fluids that remove heat from a piece of equipment.}
}


\newglossaryentry{lubrication}
{
  type=mtc,
  category=model,
  name={Lubrication},
  kind={systems,component},
  description={\gls{lubrication} is an XML container that represents the information for a system comprised of all the parts involved in distribution and management of fluids used to lubricate portions of the piece of equipment.}
}


\newglossaryentry{enclosure}
{
  type=mtc,
  category=model,
  name={Enclosure},
  kind={systems,component},
  description={\gls{enclosure} is an XML container that represents the information for a structure used to contain or isolate a piece of equipment or area.}
}


\newglossaryentry{protective}
{
  type=mtc,
  category=model,
  name={Protective},
  kind={systems,component},
  description={Protective is an XML container that represents the information for those functions that detect or prevent harm or damage to equipment or personnel.}
}


\newglossaryentry{processpower}
{
  type=mtc,
  category=model,
  name={ProcessPower},
  kind={systems,component},
  description={\gls{processpower} is an XML container that represents the information for a power source associated with a piece of equipment that supplies energy to the manufacturing process separate from the \gls{electric} system.}
}


\newglossaryentry{feeder}
{
  type=mtc,
  category=model,
  name={Feeder},
  kind={systems,component},
  description={\gls{feeder} is an XML container that represents the information for a system that manages the delivery of materials within a piece of equipment. }
}


\newglossaryentry{dielectric}
{
  type=mtc,
  category=model,
  name={Dielectric},
  kind={systems,component},
  description={\gls{dielectric} is an XML container that represents the information for a system that manages a chemical mixture used in a manufacturing process being performed at that piece of equipment.}
}


\newglossaryentry{linearpositionfeedback}
{
  type=mtc,
  category=model,
  name={LINEAR\_POSITION\_FEEDBACK},
  kind={composition},
  description={A mechanism that measures linear motion or position.}
}


\newglossaryentry{motor}
{
  type=mtc,
  category=model,
  name={MOTOR},
  kind={composition},
  description={A mechanism that converts electrical, pneumatic, or hydraulic energy into mechanical energy.}
}


\newglossaryentry{oil}
{
  type=mtc,
  category=model,
  name={OIL},
  kind={composition},
  description={A viscous liquid.}
}


\newglossaryentry{powersupply}
{
  type=mtc,
  category=model,
  name={POWER\_SUPPLY},
  kind={composition},
  description={A unit that provides power to electric mechanisms.}
}


\newglossaryentry{pulley}
{
  type=mtc,
  category=model,
  name={PULLEY},
  kind={composition},
  description={A mechanism or wheel that turns in a frame or block and serves to change the direction of or to transmit force.}
}


\newglossaryentry{pump}
{
  type=mtc,
  category=model,
  name={PUMP},
  kind={composition},
  description={An apparatus raising, driving, exhausting, or compressing fluids or gases by means of a piston, plunger, or set of rotating vanes.}
}


\newglossaryentry{sensingelement}
{
  type=mtc,
  category=model,
  name={SENSING\_ELEMENT},
  kind={composition},
  description={A mechanism that provides a signal or measured value.}
}


\newglossaryentry{storagebattery}
{
  type=mtc,
  category=model,
  name={STORAGE\_BATTERY},
  kind={composition},
  description={A component consisting of one or more cells, in which chemical energy is converted into electricity and used as a source of power. }
}


\newglossaryentry{switch}
{
  type=mtc,
  category=model,
  name={SWITCH},
  kind={composition},
  description={A mechanism for turning on or off an electric current or for making or breaking a circuit.}
}


\newglossaryentry{tank}
{
  type=mtc,
  category=model,
  name={TANK},
  kind={composition},
  description={A receptacle or container for holding material.}
}


\newglossaryentry{tensioner}
{
  type=mtc,
  category=model,
  name={TENSIONER},
  kind={composition},
  description={A mechanism that provides or applies a stretch or strain to another mechanism.}
}


\newglossaryentry{transformer}
{
  type=mtc,
  category=model,
  name={TRANSFORMER},
  kind={composition},
  description={A mechanism that transforms electric energy from a source to a secondary circuit.}
}


\newglossaryentry{valve}
{
  type=mtc,
  category=model,
  name={VALVE},
  kind={composition},
  description={Any mechanism for halting or controlling the flow of a liquid, gas, or other material through a passage, pipe, inlet, or outlet.}
}


\newglossaryentry{water}
{
  type=mtc,
  category=model,
  name={WATER},
  kind={composition},
  description={A fluid.}
}


\newglossaryentry{average}
{
  type=mtc,
  category=model,
  name={AVERAGE},
  kind={statistic},
  description={Mathematical Average value calculated for the data item during the calculation period.}
}


\newglossaryentry{kurtosis}
{
  type=mtc,
  category=model,
  name={KURTOSIS},
  kind={statistic},
  description={A measure of the peakedness of a probability distribution; i.e., the shape of the distribution curve.}
}


\newglossaryentry{maximum value}
{
  type=mtc,
  category=model,
  name={MAXIMUM},
  kind={statistic,subtype},
  description={Maximum value of a data entity or attribute.}
}


\newglossaryentry{median}
{
  type=mtc,
  category=model,
  name={MEDIAN},
  kind={statistic},
  description={The middle number of a series of numbers.}
}


\newglossaryentry{minimum value}
{
  type=mtc,
  category=model,
  name={MINIMUM},
  kind={statistic,subtype},
  description={The minimum value of a data entity or attribute.}
}


\newglossaryentry{mode}
{
  type=mtc,
  category=model,
  name={MODE},
  kind={statistic},
  description={The number in a series of numbers that occurs most often.}
}


\newglossaryentry{range}
{
  type=mtc,
  category=model,
  name={RANGE},
  kind={statistic},
  description={Difference between the Maximum and Minimum value of a data item during the calculation period.  Also represents Peak-to-Peak measurement in a waveform.}
}


\newglossaryentry{rootmeansquare}
{
  type=mtc,
  category=model,
  name={ROOT\_MEAN\_SQUARE},
  kind={statistic},
  description={Mathematical Root Mean Square (RMS) value calculated for the data item during the calculation period.}
}


\newglossaryentry{standarddeviation}
{
  type=mtc,
  category=model,
  name={STANDARD\_DEVIATION},
  kind={statistic},
  description={Statistical Standard Deviation value calculated for the data item during the calculation period.}
}


\newglossaryentry{acceleration sample}
{
  type=mtc,
  category=model,
  name={ACCELERATION},
  elementname=\cfont{Acceleration},
  description={The measurement of the rate of change of velocity.},
  units=\cfont{\gls{millimeterpersecondsquared}},
  kind={type,sample},
  facet={\gls{float}}
}


\newglossaryentry{accumulatedtime sample}
{
  type=mtc,
  category=model,
  name={ACCUMULATED\_TIME},
  elementname=\cfont{AccumulatedTime},
  description={The measurement of accumulated time for an activity or event.},
  units=\cfont{\gls{second}},
  kind={type,sample},
  facet={\gls{float}}
}


\newglossaryentry{amperage sample}
{
  type=mtc,
  category=model,
  name={AMPERAGE},
  elementname=\cfont{Amperage},
  description={The measurement of electrical current.},
  units=\cfont{\gls{ampere}},
  kind={type,sample},
  facet={\gls{float}}
}



\newglossaryentry{direct subtype}
{
  type=mtc,
  category=model,
  name={DIRECT},
  description={The measurement of DC current or voltage.},
  kind={subtype}
}


\newglossaryentry{target subtype}
{
  type=mtc,
  category=model,
  name={TARGET},
  description={The desired measure or count for a data item value.},
  kind={subtype}
}


\newglossaryentry{angle sample}
{
  type=mtc,
  category=model,
  name={ANGLE},
  elementname=\cfont{Angle},
  description={The measurement of angular position.},
  units=\cfont{\gls{degree}},
  kind={type,sample},
  facet={\gls{float}}
}


\newglossaryentry{commanded subtype}
{
  type=mtc,
  category=model,
  name={COMMANDED},
  description={A value specified by the \gls{controller} type component.},
  kind={subtype}
}


\newglossaryentry{angularacceleration sample}
{
  type=mtc,
  category=model,
  name={ANGULAR\_ACCELERATION},
  elementname=\cfont{AngularAcceleration},
  description={The measurement rate of change of angular velocity.},
  units=\cfont{\gls{degreepersecondsquared}},
  kind={type,sample},
  facet={\gls{float}}
}


\newglossaryentry{angularvelocity sample}
{
  type=mtc,
  category=model,
  name={ANGULAR\_VELOCITY},
  elementname=\cfont{AngularVelocity},
  representation=\cfont{AngularVelocityTimeSeries},
  description={The measurement of the rate of change of angular position.},
  units=\cfont{\gls{degreepersecond}},
  kind={type,sample},
  facet={\gls{float}}
}


\newglossaryentry{axisfeedrate sample}
{
  type=mtc,
  category=model,
  name={AXIS\_FEEDRATE},
  elementname=\cfont{AxisFeedrate},
  description={The measurement of the feedrate of a linear axis.},
  units=\cfont{\gls{millimeterpersecond}},
  kind={type,sample},
  facet={\gls{float}}
}



\newglossaryentry{override subtype}
{
  type=mtc,
  category=model,
  name={OVERRIDE},
  description={The operators overridden value.},
  deprecated={true},
  kind={subtype}
}


\newglossaryentry{programmed subtype}
{
  type=mtc,
  category=model,
  name={PROGRAMMED},
  description={The value of a signal or calculation specified by a logic or motion program or set by a switch.},
  kind={subtype}
}


\newglossaryentry{rapid subtype}
{
  type=mtc,
  category=model,
  name={RAPID},
  description={The value of a signal or calculation issued to adjust the feedrate of a component or composition that is operating in a rapid positioning mode.},
  kind={subtype}
}


\newglossaryentry{clocktime sample}
{
  type=mtc,
  category=model,
  name={CLOCK\_TIME},
  elementname=\cfont{ClockTime},
  description={The value provided by a timing device at a specific point in time.},
  units=\cfont{yyyy-mm-ddthh:mm:ss.ffff},
  kind={type,sample},
  facet={\gls{datetime}}
}


\newglossaryentry{concentration sample}
{
  type=mtc,
  category=model,
  name={CONCENTRATION},
  elementname=\cfont{Concentration},
  description={The measurement of the percentage of one component within a mixture of components},
  units=\cfont{\gls{percent}},
  kind={type,sample},
  facet={\gls{float}}
}


\newglossaryentry{conductivity sample}
{
  type=mtc,
  category=model,
  name={CONDUCTIVITY},
  elementname=\cfont{Conductivity},
  description={The measurement of the ability of a material to conduct electricity.},
  units=\cfont{\gls{siemenspermeter}},
  kind={type,sample},
  facet={\gls{float}}
}


\newglossaryentry{displacement sample}
{
  type=mtc,
  category=model,
  name={DISPLACEMENT},
  elementname=\cfont{Displacement},
  description={The measurement of the change in position of an object.},
  units=\cfont{\gls{millimeter}},
  kind={type,sample},
  facet={\gls{float}}
}


\newglossaryentry{electricalenergy sample}
{
  type=mtc,
  category=model,
  name={ELECTRICAL\_ENERGY},
  elementname=\cfont{ElectricalEnergy},
  description={The measurement of electrical energy consumption by a component.},
  units=\cfont{\gls{wattsecond}},
  kind={type,sample},
  facet={\gls{float}}
}


\newglossaryentry{equipmenttimer sample}
{
  type=mtc,
  category=model,
  name={EQUIPMENT\_TIMER},
  elementname=\cfont{EquipmentTimer},
  description={The measurement of the amount of time a piece of equipment or a sub-part of a piece of equipment has performed specific activities.},
  units=\cfont{\gls{second}},
  kind={type,sample},
  facet={\gls{float}}
}



\newglossaryentry{working subtype}
{
  type=mtc,
  category=model,
  name={WORKING},
  description={A piece of equipment performing any activity, the equipment is active and performing a function under load or not.},
  kind={subtype}
}


\newglossaryentry{filllevel sample}
{
  type=mtc,
  category=model,
  name={FILL\_LEVEL},
  elementname=\cfont{FillLevel},
  description={The measurement of the amount of a substance remaining compared to the planned maximum amount of that substance.},
  units=\cfont{\gls{percent}},
  kind={type,sample},
  facet={\gls{float}}
}


\newglossaryentry{flow sample}
{
  type=mtc,
  category=model,
  name={FLOW},
  elementname=\cfont{Flow},
  description={The measurement of the rate of flow of a fluid.},
  units=\cfont{\gls{literpersecond}},
  kind={type,sample},
  facet={\gls{float}}
}


\newglossaryentry{frequency sample}
{
  type=mtc,
  category=model,
  name={FREQUENCY},
  elementname=\cfont{Frequency},
  description={The measurement of the number of occurrences of a repeating event per unit time.},
  units=\cfont{\gls{hertz}},
  kind={type,sample},
  facet={\gls{float}}
}


\newglossaryentry{globalposition sample}
{
  type=mtc,
  category=model,
  name=\deprecated{GLOBAL\_POSITION},
  elementname=\deprecated{\cfont{GlobalPosition}},
  description={DEPRECATED in Version 1.1},
  deprecated={true},
  units={},
  kind={type,sample},
  facet={}
}


\newglossaryentry{length sample}
{
  type=mtc,
  category=model,
  name={LENGTH},
  elementname=\cfont{Length},
  description={The measurement of the length of an object.},
  units=\cfont{\gls{millimeter}},
  kind={type,sample,subtype},
  facet={\gls{float}}
}


\newglossaryentry{standard length sample}
{
  type=mtc,
  category=model,
  name={STANDARD},
  description={The standard or original length of an object.},
  units=\cfont{\gls{millimeter}},
  kind={subtype,sample},
  facet={\gls{float}}
}


\newglossaryentry{useable length sample}
{
  type=mtc,
  category=model,
  name={USEABLE},
  description={The remaining useable length of an object.},
  units=\cfont{\gls{millimeter}},
  kind={subtype,sample},
  facet={\gls{float}}
}


\newglossaryentry{level sample}
{
  type=mtc,
  category=model,
  name={LEVEL},
  elementname=\cfont{Level},
  description={DEPRECATED in Version 1.2.  See \gls{filllevel sample}},
  deprecated={true},
  units={},
  kind={type,sample},
  facet={}
}


\newglossaryentry{linearforce sample}
{
  type=mtc,
  category=model,
  name={LINEAR\_FORCE},
  elementname=\cfont{LinearForce},
  description={The measurement of the push or pull introduced by an actuator or exerted on an object.},
  units=\cfont{\gls{newton}},
  kind={type,sample},
  facet={\gls{float}}
}


\newglossaryentry{load sample}
{
  type=mtc,
  category=model,
  name={LOAD},
  elementname=\cfont{Load},
  description={The measurement of the actual versus the standard rating of a piece of equipment.},
  units=\cfont{\gls{percent}},
  kind={type,sample},
  facet={\gls{float}}
}


\newglossaryentry{mass sample}
{
  type=mtc,
  category=model,
  name={MASS},
  elementname=\cfont{Mass},
  description={The measurement of the mass of an object(s) or an amount of material.},
  units=\cfont{\gls{kilogram}},
  kind={type,sample},
  facet={\gls{float}}
}


\newglossaryentry{pathfeedrate sample}
{
  type=mtc,
  category=model,
  name={PATH\_FEEDRATE},
  elementname=\cfont{PathFeedrate},
  description={The measurement of the feedrate for the axes, or a single axis, associated with a \glselementname{path} component a vector.},
  units=\cfont{\gls{millimeterpersecond}},
  kind={type,sample},
  facet={\gls{float}}
}





\newglossaryentry{pathposition sample}
{
  type=mtc,
  category=model,
  name={PATH\_POSITION},
  elementname=\cfont{PathPosition},
  description={A measured or calculated position of a control point associated with a \gls{controller} element, or \gls{path} element if provided, of a piece of equipment.},
  units=\cfont{\gls{millimeter3d}},
  kind={type,sample},
  facet={\gls{array3d}}
}


\newglossaryentry{probe pathposition sample}
{
  type=mtc,
  category=model,
  name={PROBE},
  description={The position provided by a measurement probe.},
  units=\cfont{\gls{millimeter3d}},
  kind={subtype,sample},
  facet={\gls{array3d}}
}


\newglossaryentry{ph sample}
{
  type=mtc,
  category=model,
  name={PH},
  elementname=\cfont{PH},
  description={A measure of the acidity or alkalinity of a solution.},
  units=\cfont{\gls{ph sample}},
  kind={type,sample,units},
  facet={\gls{float}}
}


\newglossaryentry{position sample}
{
  type=mtc,
  category=model,
  name={POSITION},
  elementname=\cfont{Position},
  description={A measured or calculated position of a \gls{component} element as reported by a piece of equipment.},
  units=\cfont{\gls{millimeter}},
  kind={type,sample},
  facet={\gls{float}}
}


\newglossaryentry{powerfactor sample}
{
  type=mtc,
  category=model,
  name={POWER\_FACTOR},
  elementname=\cfont{PowerFactor},
  description={The measurement of the ratio of real power flowing to a load to the apparent power in that AC circuit.},
  units=\cfont{\gls{percent}},
  kind={type,sample},
  facet={\gls{float}}
}


\newglossaryentry{pressure sample}
{
  type=mtc,
  category=model,
  name={PRESSURE},
  elementname=\cfont{Pressure},
  description={The measurement of force per unit area exerted by a gas or liquid.},
  units=\cfont{\gls{pascal}},
  kind={type,sample},
  facet={\gls{float}}
}


\newglossaryentry{processtimer sample}
{
  type=mtc,
  category=model,
  name={PROCESS\_TIMER},
  elementname=\cfont{ProcessTimer},
  description={The measurement of the amount of time a piece of equipment has performed different types of activities associated with the process being performed at that piece of equipment.},
  units=\cfont{\gls{second}},
  kind={type,sample},
  facet={\gls{float}}
}




\newglossaryentry{process processtimer sample}
{
  type=mtc,
  category=model,
  name={PROCESS},
  description={The measurement of the time from the beginning of production of a part or product on a piece of equipment until the time that production is complete for that part or product on that piece of equipment.  This includes the time that the piece of equipment is running, producing parts or products, or in the process of producing parts.},
  units=\cfont{\gls{second}},
  kind={subtype,sample},
  facet={\gls{float}}
}


\newglossaryentry{resistance sample}
{
  type=mtc,
  category=model,
  name={RESISTANCE},
  elementname=\cfont{Resistance},
  description={The measurement of the degree to which a substance opposes the passage of an electric current.},
  units=\cfont{\gls{ohm}},
  kind={type,sample},
  facet={\gls{float}}
}


\newglossaryentry{rotaryvelocity sample}
{
  type=mtc,
  category=model,
  name={ROTARY\_VELOCITY},
  elementname=\cfont{RotaryVelocity},
  description={The measurement of the rotational speed of a rotary axis.},
  units=\cfont{\gls{revolutionperminute}},
  kind={type,sample},
  facet={\gls{float}}
}



\newglossaryentry{soundlevel sample}
{
  type=mtc,
  category=model,
  name={SOUND\_LEVEL},
  elementname=\cfont{SoundLevel},
  description={The measurement of a sound level or sound pressure level relative to atmospheric pressure.},
  units=\cfont{\gls{decibel}},
  kind={type,sample},
  facet={\gls{float}}
}


\newglossaryentry{ascale subtype}
{
  type=mtc,
  category=model,
  name={A\_SCALE},
  description={A Scale weighting factor.   This is the default weighting factor if no factor is specified},
  kind={subtype}
}


\newglossaryentry{bscale subtype}
{
  type=mtc,
  category=model,
  name={B\_SCALE},
  description={B Scale weighting factor},
  kind={subtype}
}


\newglossaryentry{cscale subtype}
{
  type=mtc,
  category=model,
  name={C\_SCALE},
  description={C Scale weighting factor},
  kind={subtype}
}


\newglossaryentry{dscale subtype}
{
  type=mtc,
  category=model,
  name={D\_SCALE},
  description={D Scale weighting factor},
  kind={subtype}
}


\newglossaryentry{noscale subtype}
{
  type=mtc,
  category=model,
  name={NO\_SCALE},
  description={No weighting factor on the frequency scale},
  kind={subtype}
}


\newglossaryentry{spindlespeed sample}
{
  type=mtc,
  category=model,
  name={\deprecated{SPINDLE\_SPEED}},
  elementname=\deprecated{\cfont{SpindleSpeed}},
  description={DEPRECATED in Version 1.2.  Replaced by \gls{rotaryvelocity sample}},
  deprecated={true},
  units={},
  kind={type,sample},
  facet={}
}



\newglossaryentry{strain sample}
{
  type=mtc,
  category=model,
  name={STRAIN},
  elementname=\cfont{Strain},
  description={The measurement of the amount of deformation per unit length of an object when a load is applied.},
  units=\cfont{\gls{percent}},
  kind={type,sample},
  facet={\gls{float}}
}


\newglossaryentry{temperature sample}
{
  type=mtc,
  category=model,
  name={TEMPERATURE},
  elementname=\cfont{Temperature},
  description={The measurement of temperature.},
  units=\cfont{\gls{celsius}},
  kind={type,sample},
  facet={\gls{float}}
}


\newglossaryentry{tension sample}
{
  type=mtc,
  category=model,
  name={TENSION},
  elementname=\cfont{Tension},
  description={The measurement of a force that stretches or elongates an object.},
  units=\cfont{\gls{newton}},
  kind={type,sample},
  facet={\gls{float}}
}


\newglossaryentry{tilt sample}
{
  type=mtc,
  category=model,
  name={TILT},
  elementname=\cfont{Tilt},
  description={The measurement of angular displacement. },
  units=\cfont{\gls{microradian}},
  kind={type,sample},
  facet={\gls{float}}
}


\newglossaryentry{torque sample}
{
  type=mtc,
  category=model,
  name={TORQUE},
  elementname=\cfont{Torque},
  description={The measurement of the turning force exerted on an object or by an object.},
  units=\cfont{\gls{newtonmeter}},
  kind={type,sample},
  facet={\gls{float}}
}


\newglossaryentry{velocity sample}
{
  type=mtc,
  category=model,
  name={VELOCITY},
  elementname=\cfont{Velocity},
  description={The measurement of the rate of change of position of a \gls{component}.},
  units=\cfont{\gls{millimeterpersecond}},
  kind={type,sample},
  facet={\gls{float}}
}


\newglossaryentry{viscosity sample}
{
  type=mtc,
  category=model,
  name={VISCOSITY},
  elementname=\cfont{Viscosity},
  description={The measurement of a fluids resistance to flow.},
  units=\cfont{\gls{pascalsecond}},
  kind={type,sample},
  facet={\gls{float}}
}


\newglossaryentry{voltampere sample}
{
  type=mtc,
  category=model,
  name={VOLT\_AMPERE},
  elementname=\cfont{VoltAmpere},
  description={The measurement of the apparent power in an electrical circuit, equal to the product of root-mean-square (RMS) voltage and RMS current (commonly referred to as VA).},
  units=\cfont{\gls{voltampere sample}},
  kind={type,sample,units},
  facet={\gls{float}}
}


\newglossaryentry{voltamperereactive sample}
{
  type=mtc,
  category=model,
  name={VOLT\_AMPERE\_REACTIVE},
  elementname=\cfont{VoltAmpereReactive},
  description={The measurement of reactive power in an AC electrical circuit (commonly referred to as VAR).},
  units=\cfont{\gls{voltamperereactive sample}},
  kind={type,sample,units},
  facet={\gls{float}}
}


\newglossaryentry{voltage sample}
{
  type=mtc,
  category=model,
  name={VOLTAGE},
  elementname=\cfont{Voltage},
  description={The measurement of electrical potential between two points.},
  units=\cfont{\gls{volt}},
  kind={type,sample},
  facet={\gls{float}}
}


\newglossaryentry{alternating subtype}
{
  type=mtc,
  category=model,
  name={ALTERNATING},
  description={The measurement of alternating voltage or current.   If not specified further in statistic, defaults to RMS voltage. },
  kind={subtype}
}


\newglossaryentry{wattage sample}
{
  type=mtc,
  category=model,
  name={WATTAGE},
  elementname=\cfont{Wattage},
  description={The measurement of power flowing through or dissipated by an electrical circuit or piece of equipment.},
  units=\cfont{\gls{watt}},
  kind={type,sample},
  facet={\gls{float}}
}


\newglossaryentry{activeaxes event}
{
  type=mtc,
  category=model,
  name={ACTIVE\_AXES},
  elementname=\cfont{ActiveAxes},
  description={The set of axes currently associated with a \gls{path} or \gls{controller} \gls{structural element}.},
  kind={type,event},
  facet={\gls{arraystring}}
}


\newglossaryentry{actuatorstate event}
{
  type=mtc,
  category=model,
  name={ACTUATOR\_STATE},
  elementname=\cfont{ActuatorState},
  description={Represents the operational state of an apparatus for moving or controlling a mechanism or system.},
  kind={type,event},
  facet={\gls{string}},
  enumeration={\gls{active value},\gls{inactive value}}
}


\newglossaryentry{alarm event}
{
  type=mtc,
  category=model,
  name=\deprecated{ALARM},
  elementname=\deprecated{\cfont{Alarm}},
  description={DEPRECATED: Replaced with \gls{condition category} category data items in Version 1.1.0.},
  deprecated={true},
  kind={type,event},
  facet={}
}


\newglossaryentry{availability event}
{
  type=mtc,
  category=model,
  name={AVAILABILITY},
  elementname=\cfont{Availability},
  description={Represents the \gls{agent}s ability to communicate with the data source.},
  kind={type,event},
  facet={\gls{string}},
  enumeration={\gls{available value},\gls{unavailable value}}
}


\newglossaryentry{axiscoupling event}
{
  type=mtc,
  category=model,
  name={AXIS\_COUPLING},
  elementname=\cfont{AxisCoupling},
  description={Describes the way the axes will be associated to each other. This is used in conjunction with \gls{coupledaxes event} to indicate the way they are interacting.},
  kind={type,event},
  facet={\gls{string}},
  enumeration={\gls{tandem value},\gls{synchronous value},\gls{master value},\gls{slave value}}
}


\newglossaryentry{axisfeedrateoverride event}
{
  type=mtc,
  category=model,
  name={AXIS\_FEEDRATE\_OVERRIDE},
  elementname=\cfont{AxisFeedrateOverride},
  description={The value of a signal or calculation issued to adjust the feedrate of an individual linear type axis.},
  kind={type,event},
  facet={\gls{float}}
}



\newglossaryentry{axisinterlock event}
{
  type=mtc,
  category=model,
  name={AXIS\_INTERLOCK},
  elementname=\cfont{AxisInterlock},
  description={An indicator of the state of the axis lockout function when power has been removed and the axis is allowed to move freely.},
  kind={type,event},
  facet={\gls{string}},
  enumeration={\gls{active value},\gls{inactive value}}
}


\newglossaryentry{axisstate event}
{
  type=mtc,
  category=model,
  name={AXIS\_STATE},
  elementname=\cfont{AxisState},
  description={An indicator of the controlled state of a \gls{linear} or \gls{rotary} component representing an axis.},
  kind={type,event},
  facet={\gls{string}},
  enumeration={\gls{home value},\gls{travel value},\gls{parked value},\gls{stopped value}}
}


\newglossaryentry{block event}
{
  type=mtc,
  category=model,
  name={BLOCK},
  elementname=\cfont{Block},
  description={The line of code or command being executed by a \gls{controller} \gls{structural element}.},
  kind={type,event},
  facet={\gls{string}}
}


\newglossaryentry{blockcount event}
{
  type=mtc,
  category=model,
  name={BLOCK\_COUNT},
  elementname=\cfont{BlockCount},
  description={The total count of the number of blocks of program code that have been executed since execution started.},
  kind={type,event},
  facet={\gls{integer}}
}


\newglossaryentry{chuckinterlock event}
{
  type=mtc,
  category=model,
  name={CHUCK\_INTERLOCK},
  elementname=\cfont{ChuckInterlock},
  description={An indication of the state of an interlock function or control logic state intended to prevent the associated \gls{chuck} component from being operated.},
  kind={type,event,condition},
  facet={\gls{string}},
  enumeration={\gls{active value},\gls{inactive value}}
}


\newglossaryentry{manualunclamp chuckinterlock event}
{
  type=mtc,
  category=model,
  name={MANUAL\_UNCLAMP},
  description={An indication of the state of an operator controlled interlock that can inhibit the ability to initiate an unclamp action of an electronically controlled chuck.  The \gls{valid data value} \must be ACTIVE or INACTIVE. \newline When MANUAL\_UNCLAMP is ACTIVE, it is expected that a chuck cannot be unclamped until MANUAL\_UNCLAMP is set to INACTIVE. },
  kind={subtype,event},
  facet={\gls{string}},
  enumeration={\gls{active value},\gls{inactive value}}
}


\newglossaryentry{chuckstate event}
{
  type=mtc,
  category=model,
  name={CHUCK\_STATE},
  elementname=\cfont{ChuckState},
  description={An indication of the operating state of a mechanism that holds a part or stock material during a manufacturing process. It may also represent a mechanism that holds any other mechanism in place within a piece of equipment.},
  kind={type,event},
  facet={\gls{string}},
  enumeration={\gls{open value},\gls{closed value},\gls{unlatched value}}
}


\newglossaryentry{code event}
{
  type=mtc,
  category=model,
  name=\deprecated{CODE},
  elementname=\deprecated{\cfont{Code}},
  description={DEPRECATED in Version 1.1.},
  deprecated={true},
  kind={type,event},
  facet={}
}


\newglossaryentry{compositionstate event}
{
  type=mtc,
  category=model,
  name={COMPOSITION\_STATE},
  elementname=\cfont{CompositionState},
  description={An indication of the operating condition of a mechanism represented by a \gls{composition} type element.},
  kind={type,event},
  facet={\gls{string}}
}


\newglossaryentry{action subtype}
{
  type=mtc,
  category=model,
  name={ACTION},
  description={An indication of the operating state of a mechanism represented by a \gls{composition} type component.The operating state indicates whether the \gls{composition} element is activated or disabled. The \gls{valid data value} \must be \gls{active value} or \gls{inactive value}.},
  kind={subtype}
}


\newglossaryentry{lateral compositionstate event}
{
  type=mtc,
  category=model,
  name={LATERAL},
  description={An indication of the position of a mechanism that may move in a lateral direction.   The mechanism is represented by a \gls{composition} type component. \newline The position information indicates whether the \gls{composition} element is positioned to the right, to the left, or is in transition.  \newline The \gls{valid data value} \must be RIGHT, LEFT, or TRANSITIONING.},
  kind={subtype,event},
  facet={\gls{string}},
  enumeration={\gls{right value},\gls{left value},\gls{transitioning value}}
}


\newglossaryentry{motion compositionstate event}
{
  type=mtc,
  category=model,
  name={MOTION},
  description={An indication of the open or closed state of a mechanism.   The mechanism is represented by a \gls{composition} type component. \newline The operating state indicates whether the state of the \gls{composition} element is open, closed, or unlatched.   \newline The \gls{valid data value} \must be OPEN, UNLATCHED, or CLOSED.},
  kind={subtype,event},
  facet={\gls{string}},
  enumeration={\gls{open value},\gls{closed value},\gls{unlatched value}}
}


\newglossaryentry{switched compositionstate event}
{
  type=mtc,
  category=model,
  name={SWITCHED},
  description={An indication of the activation state of a mechanism represented by a \gls{composition} type component.The activation state indicates whether the \gls{composition} element is activated or not.The \gls{valid data value} \must be ON or OFF.},
  kind={subtype,event},
  facet={\gls{string}},
  enumeration={\gls{on value},\gls{off value}}
}


\newglossaryentry{vertical compositionstate event}
{
  type=mtc,
  category=model,
  name={VERTICAL},
  description={An indication of the position of a mechanism that may move in a vertical direction. The mechanism is represented by a \gls{composition} type component. \newline The position information indicates whether the \gls{composition} element is positioned to the top, to the bottom, or is in transition.  \newline The \gls{valid data value} \must be UP, DOWN, or TRANSITIONING.},
  kind={subtype,event},
  facet={\gls{string}},
  enumeration={\gls{up value},\gls{down value},\gls{transitioning value}}
}


\newglossaryentry{controllermode event}
{
  type=mtc,
  category=model,
  name={CONTROLLER\_MODE},
  elementname=\cfont{ControllerMode},
  description={The current operating mode of the \gls{controller} component.},
  kind={type,event},
  facet={\gls{string}},
  enumeration={\gls{automatic value},\gls{manual value},\gls{manualdatainput value},\gls{semiautomatic value},\gls{edit value}}
}


\newglossaryentry{controllermodeoverride event}
{
  type=mtc,
  category=model,
  name={CONTROLLER\_MODE\_OVERRIDE},
  elementname=\cfont{ControllerModeOverride},
  description={A setting or operator selection that changes the behavior of a piece of equipment.},
  kind={type,event},
  facet={\gls{string}},
  enumeration={\gls{on value},\gls{off value}}
}


\newglossaryentry{dryrun controllermodeoverride event}
{
  type=mtc,
  category=model,
  name={DRY\_RUN},
  description={A setting or operator selection used to execute a test mode to confirm the execution of machine functions.  The \gls{valid data value} \must be ON or OFF. \newline When DRY\_RUN is ON, the equipment performs all of its normal functions, except no part or product is produced.  If the equipment has a spindle, spindle operation is suspended.},
  kind={subtype,event},
  facet={\gls{string}},
  enumeration={\gls{on value},\gls{off value}}
}


\newglossaryentry{machineaxislock controllermodeoverride event}
{
  type=mtc,
  category=model,
  name={MACHINE\_AXIS\_LOCK},
  description={A setting or operator selection that changes the behavior of the controller on a piece of equipment.  The \gls{valid data value} \must be ON or OFF. \newline When MACHINE\_AXIS\_LOCK is ON, program execution continues normally, but no equipment motion occurs },
  kind={subtype,event},
  facet={\gls{string}},
  enumeration={\gls{on value},\gls{off value}}
}


\newglossaryentry{optionalstop controllermodeoverride event}
{
  type=mtc,
  category=model,
  name={OPTIONAL\_STOP},
  description={A setting or operator selection that changes the behavior of the controller on a piece of equipment.  The \gls{valid data value} \must be ON or OFF.The program execution is stopped after a specific program block is executed when OPTIONAL\_STOP is ON.    \newline In the case of a G-Code program, a program BLOCK containing a M01 code designates the command for an OPTIONAL\_STOP.EXECUTION \must change to OPTIONAL\_STOP after a program block specifying an optional stop is executed and the OPTIONAL\_STOP selection is ON.},
  kind={subtype,event},
  facet={\gls{string}},
  enumeration={\gls{on value},\gls{off value}}
}


\newglossaryentry{singleblock controllermodeoverride event}
{
  type=mtc,
  category=model,
  name={SINGLE\_BLOCK},
  description={A setting or operator selection that changes the behavior of the controller on a piece of equipment.  The \gls{valid data value} \must be ON or OFF. Program execution is paused after each BLOCK of code is executed when SINGLE\_BLOCK is ON.   \newline When SINGLE\_BLOCK is ON, EXECUTION \must change to INTERRUPTED after completion of each BLOCK of code. },
  kind={subtype,event},
  facet={\gls{string}},
  enumeration={\gls{on value},\gls{off value}}
}


\newglossaryentry{toolchangestop controllermodeoverride event}
{
  type=mtc,
  category=model,
  name={TOOL\_CHANGE\_STOP},
  description={A setting or operator selection that changes the behavior of the controller on a piece of equipment.  The \gls{valid data value} \must be ON or OFF. Program execution is paused when a command is executed requesting a cutting tool to be changed. EXECUTION \must change to INTERRUPTED after completion of the command requesting a cutting tool to be changed and TOOL\_CHANGE\_STOP is ON.},
  kind={subtype,event},
  facet={\gls{string}},
  enumeration={\gls{on value},\gls{off value}}
}


\newglossaryentry{coupledaxes event}
{
  type=mtc,
  category=model,
  name={COUPLED\_AXES},
  elementname=\cfont{CoupledAxes},
  description={Refers to the set of associated axes.},
  kind={type,event},
  facet={\gls{arraystring}}
}


\newglossaryentry{direction event}
{
  type=mtc,
  category=model,
  name={DIRECTION},
  elementname=\cfont{Direction},
  description={The direction of motion.},
  kind={type,event,condition},
  facet={\gls{string}}
}


\newglossaryentry{linear subtype}
{
  type=mtc,
  category=model,
  name={LINEAR},
  description={The direction of motion of a linear motion.},
  kind={subtype}
}


\newglossaryentry{rotary direction event}
{
  type=mtc,
  category=model,
  name={ROTARY},
  description={The rotational direction of a rotary motion using the right hand rule convention.The \gls{valid data value} \must be CLOCKWISE or COUNTER\_CLOCKWISE.},
  kind={subtype,event},
  facet={\gls{string}},
  enumeration={\gls{clockwise value},\gls{counterclockwise value}}
}


\newglossaryentry{doorstate event}
{
  type=mtc,
  category=model,
  name={DOOR\_STATE},
  elementname=\cfont{DoorState},
  description={The operational state of a \gls{door} type component or composition element.},
  kind={type,event},
  facet={\gls{string}},
  enumeration={\gls{closed value},\gls{closed value},\gls{unlatched value}}
}


\newglossaryentry{emergencystop event}
{
  type=mtc,
  category=model,
  name={EMERGENCY\_STOP},
  elementname=\cfont{EmergencyStop},
  description={The current state of the emergency stop signal for a piece of equipment, controller path, or any other component or subsystem of a piece of equipment.},
  kind={type,event},
  facet={\gls{string}},
  enumeration={\gls{armed value},\gls{triggered value}}
}


\newglossaryentry{endofbar event}
{
  type=mtc,
  category=model,
  name={END\_OF\_BAR},
  elementname=\cfont{EndOfBar},
  description={An indication of whether the end of a piece of bar stock being feed by a bar feeder has been reached.},
  kind={type,event,condition},
  facet={\gls{string}},
  enumeration={\gls{yes value},\gls{no value}}
}


\newglossaryentry{auxiliary subtype}
{
  type=mtc,
  category=model,
  name={AUXILIARY},
  description={When multiple locations on a piece of bar stock are referenced as the indication for the \gls{endofbar event}, the additional location(s) \must be designated as \gls{auxiliary subtype} indication(s) for the \gls{endofbar event}.  },
  kind={subtype}
}


\newglossaryentry{primary endofbar event}
{
  type=mtc,
  category=model,
  name={PRIMARY},
  description={Specific applications MAY reference one or more locations on a piece of bar stock as the indication for the END\_OF\_BAR.  The main or most important location \must be designated as the PRIMARY indication for the END\_OF\_BAR.   \newline If no \gls{subtype} is specified, PRIMARY \must be the default END\_OF\_BAR indication.},
  kind={subtype,event},
  facet={\gls{string}},
  enumeration={\gls{yes value},\gls{no value}}
}


\newglossaryentry{equipmentmode event}
{
  type=mtc,
  category=model,
  name={EQUIPMENT\_MODE},
  elementname=\cfont{EquipmentMode},
  description={An indication that a piece of equipment, or a sub-part of a piece of equipment, is performing specific types of activities.},
  kind={type,event},
  facet={\gls{string}},
  enumeration={\gls{on value},\gls{off value}}
}


\newglossaryentry{delay subtype}
{
  type=mtc,
  category=model,
  name={DELAY},
  description={A piece of equipment waiting for an event or an action to occur.},
  kind={subtype}
}


\newglossaryentry{loaded subtype}
{
  type=mtc,
  category=model,
  name={LOADED},
  description={Subparts of a piece of equipment are under load.},
  kind={subtype}
}


\newglossaryentry{operating subtype}
{
  type=mtc,
  category=model,
  name={OPERATING},
  description={A piece of equipment are powered or performing any activity.},
  kind={subtype}
}


\newglossaryentry{powered subtype}
{
  type=mtc,
  category=model,
  name={POWERED},
  description={Primary  power is  applied  to the  piece  of  equipment and,  as  a minimum, the controller or logic portion of the piece of equipment is powered and functioning or components that are required to remain on are powered.},
  kind={subtype}
}



\newglossaryentry{execution event}
{
  type=mtc,
  category=model,
  name={EXECUTION},
  elementname=\cfont{Execution},
  description={The execution status of the \gls{controller}.},
  kind={type,event},
  facet={\gls{string}},
  enumeration={\gls{ready value},\gls{active value},\gls{interrupted value},\gls{feedhold value},\gls{stopped value},\gls{optionalstop value},\gls{programstopped value},\gls{programcompleted value}}
}


\newglossaryentry{functionalmode event}
{
  type=mtc,
  category=model,
  name={FUNCTIONAL\_MODE},
  elementname=\cfont{FunctionalMode},
  description={The current intended production status of the device or component.},
  kind={type,event},
  facet={\gls{string}},
  enumeration={\gls{production value},\gls{setup value},\gls{teardown value},\gls{maintenance},\gls{processdevelopment value}}
}


\newglossaryentry{hardness event}
{
  type=mtc,
  category=model,
  name={HARDNESS},
  elementname=\cfont{Hardness},
  description={The measurement of the hardness of a material.},
  kind={type,event},
  facet={\gls{float}}
}


\newglossaryentry{brinell subtype}
{
  type=mtc,
  category=model,
  name={BRINELL},
  description={A scale to measure the resistance to deformation of a surface.},
  kind={subtype}
}


\newglossaryentry{leeb subtype}
{
  type=mtc,
  category=model,
  name={LEEB},
  description={A scale to measure the elasticity of a surface.},
  kind={subtype}
}


\newglossaryentry{mohs subtype}
{
  type=mtc,
  category=model,
  name={MOHS},
  description={A scale to measure the resistance to scratching of a surface.},
  kind={subtype}
}


\newglossaryentry{rockwell subtype}
{
  type=mtc,
  category=model,
  name={ROCKWELL},
  description={A scale to measure the resistance to deformation of a surface.},
  kind={subtype}
}


\newglossaryentry{shore subtype}
{
  type=mtc,
  category=model,
  name={SHORE},
  description={A scale to measure the resistance to deformation of a surface.},
  kind={subtype}
}


\newglossaryentry{vickers subtype}
{
  type=mtc,
  category=model,
  name={VICKERS},
  description={A scale to measure the resistance to deformation of a surface.},
  kind={subtype}
}


\newglossaryentry{interfacestate event}
{
  type=mtc,
  category=model,
  name={INTERFACE\_STATE},
  elementname=\cfont{InterfaceState},
  description={An indication of the operational state of an \gls{interface component} component.},
  kind={type,event,condition},
  facet={\gls{string}},
  enumeration={\gls{enabled value},\gls{disabled value}}
}


\newglossaryentry{line event}
{
  type=mtc,
  category=model,
  name=\deprecated{LINE},
  elementname=\deprecated{\cfont{Line}},
  description={DEPRECATED in Version 1.4.0.},
  deprecated={true},
  kind={type,event},
  facet={}
}



\newglossaryentry{linelabel event}
{
  type=mtc,
  category=model,
  name={LINE\_LABEL},
  elementname=\cfont{LineLabel},
  description={An optional identifier for a \gls{block event} of code in a \gls{program event}.},
  kind={type,event},
  facet={\gls{string}}
}


\newglossaryentry{linenumber event}
{
  type=mtc,
  category=model,
  name={LINE\_NUMBER},
  elementname=\cfont{LineNumber},
  description={A reference to the position of a block of program code within a control program.},
  kind={type,event},
  facet={\gls{integer}}
}


\newglossaryentry{absolute subtype}
{
  type=mtc,
  category=model,
  name={ABSOLUTE},
  description={The position of a block of program code relative to the beginning of the control program.},
  kind={subtype}
}


\newglossaryentry{incremental linenumber event}
{
  type=mtc,
  category=model,
  name={INCREMENTAL},
  description={The position of a block of program code relative to the occurrence of the last LINE\_LABEL encountered in the control program.},
  kind={subtype,event},
  facet={\gls{integer}}
}


\newglossaryentry{material event}
{
  type=mtc,
  category=model,
  name={MATERIAL},
  elementname=\cfont{Material},
  description={The identifier of a material used or consumed in the manufacturing process.},
  kind={type,event},
  facet={\gls{string}}
}


\newglossaryentry{message event}
{
  type=mtc,
  category=model,
  name={MESSAGE},
  elementname=\cfont{Message},
  representation=\cfont{MessageDiscrete},
  description={Any text string of information to be transferred from a piece of equipment to a client software application.},
  kind={type,event},
  facet={\gls{string}}
}


\newglossaryentry{operatorid event}
{
  type=mtc,
  category=model,
  name={OPERATOR\_ID},
  elementname=\cfont{OperatorId},
  description={The identifier of the person currently responsible for operating the piece of equipment.},
  kind={type,event},
  facet={\gls{string}}
}


\newglossaryentry{palletid event}
{
  type=mtc,
  category=model,
  name={PALLET\_ID},
  elementname=\cfont{PalletId},
  description={The identifier for a pallet.},
  kind={type,event},
  facet={\gls{string}}
}


\newglossaryentry{partcount}
{
  type=mtc,
  category=model,
  name={PART\_COUNT},
  elementname=\cfont{PartCount},
  representation=\cfont{PartCountDiscrete},
  description={The count of parts produced.},
  kind={type,event},
  facet={\gls{float}}
}


\newglossaryentry{all subtype}
{
  type=mtc,
  category=model,
  name={ALL},
  description={The count of all the parts produced.  If the subtype is not given, this is the default.},
  kind={subtype}
}


\newglossaryentry{bad subtype}
{
  type=mtc,
  category=model,
  name={BAD},
  description={Indicates the count of incorrect parts produced.},
  kind={subtype}
}


\newglossaryentry{good subtype}
{
  type=mtc,
  category=model,
  name={GOOD},
  description={Indicates the count of correct parts made.},
  kind={subtype}
}


\newglossaryentry{remaining subtype}
{
  type=mtc,
  category=model,
  name={REMAINING},
  description={Remaining measure of an object or an action.},
  kind={subtype}
}


\newglossaryentry{partid event}
{
  type=mtc,
  category=model,
  name={PART\_ID},
  elementname=\cfont{PartId},
  description={An identifier of a part in a manufacturing operation.},
  kind={type,event},
  facet={\gls{string}}
}


\newglossaryentry{partnumber event}
{
  type=mtc,
  category=model,
  name={PART\_NUMBER},
  elementname=\cfont{PartNumber},
  description={An identifier of a part or product moving through the manufacturing process.  The \gls{valid data value} \must be a text string. },
  kind={type,event},
  facet={\gls{string}}
}


\newglossaryentry{pathfeedrateoverride event}
{
  type=mtc,
  category=model,
  name={PATH\_FEEDRATE\_OVERRIDE},
  elementname=\cfont{PathFeedrateOverride},
  description={The value of a signal or calculation issued to adjust the feedrate for the axes associated with a Path component that may represent a single axis or the coordinated movement of multiple axes.},
  kind={type,event},
  facet={\gls{float}}
}


\newglossaryentry{jog subtype}
{
  type=mtc,
  category=model,
  name={JOG},
  description={The feedrate specified by a logic or motion program, by a pre-set value, or set by a switch as the feedrate for the \gls{axes}. },
  kind={subtype}
}


\newglossaryentry{pathmode event}
{
  type=mtc,
  category=model,
  name={PATH\_MODE},
  elementname=\cfont{PathMode},
  description={Describes the operational relationship between a \gls{path} \gls{structural element} and another \gls{path} \gls{structural element} for pieces of equipment comprised of multiple logical groupings of controlled axes or other logical operations.},
  kind={type,event},
  facet={\gls{string}},
  enumeration={\gls{independent value},\gls{master value},\gls{synchronous value},\gls{mirror value}}
}


\newglossaryentry{powerstate event}
{
  type=mtc,
  category=model,
  name={POWER\_STATE},
  elementname=\cfont{PowerState},
  description={The indication of the status of the source of energy for a \gls{structural element} to allow it to perform its intended function or the state of an enabling signal providing permission for the \gls{structural element} to perform its functions.},
  kind={type,event},
  facet={\gls{string}},
  enumeration={\gls{on value},\gls{off value}}
}


\newglossaryentry{control powerstate event}
{
  type=mtc,
  category=model,
  name={CONTROL},
  description={The state of the enabling signal or control logic that enables or disables the function or operation of the \gls{structural element}.},
  kind={subtype,event},
  facet={\gls{string}},
  enumeration={\gls{on value},\gls{off value}}
}


\newglossaryentry{line powerstate event}
{
  type=mtc,
  category=model,
  name={LINE},
  description={The state of the power source for the \gls{structural element}.},
  kind={subtype,event},
  facet={\gls{string}},
  enumeration={\gls{on value},\gls{off value}}
}


\newglossaryentry{powerstatus event}
{
  type=mtc,
  category=model,
  name=\deprecated{POWER\_STATUS},
  elementname=\deprecated{\cfont{PowerStatus}},
  description={DEPRECATED in Version 1.1.0.},
  deprecated={true},
  kind={type,event},
  facet={}
}


\newglossaryentry{program event}
{
  type=mtc,
  category=model,
  name={PROGRAM},
  elementname=\cfont{Program},
  description={The name of the logic or motion program being executed by the \gls{controller} component.},
  kind={type,event},
  facet={\gls{string}}
}


\newglossaryentry{programcomment event}
{
  type=mtc,
  category=model,
  name={PROGRAM\_COMMENT},
  elementname=\cfont{ProgramComment},
  description={A comment or non-executable statement in the control program.The \gls{valid data value} \must be a text string.},
  kind={type,event},
  facet={\gls{string}}
}


\newglossaryentry{programedit event}
{
  type=mtc,
  category=model,
  name={PROGRAM\_EDIT},
  elementname=\cfont{ProgramEdit},
  description={An indication of the status of the \gls{controller} components program editing mode. \newline On many controls, a program can be edited while another program is currently being executed.},
  kind={type,event},
  facet={\gls{string}},
  enumeration={\gls{active value},\gls{ready value},\gls{notready value}}
}


\newglossaryentry{programeditname event}
{
  type=mtc,
  category=model,
  name={PROGRAM\_EDIT\_NAME},
  elementname=\cfont{ProgramEditName},
  description={The name of the program being edited. \newline This is used in conjunction with \gls{programedit event} when in \gls{active value} state. \newline The \gls{valid data value} \must be a text string.},
  kind={type,event},
  facet={\gls{string}}
}


\newglossaryentry{programheader event}
{
  type=mtc,
  category=model,
  name={PROGRAM\_HEADER},
  elementname=\cfont{ProgramHeader},
  description={The non-executable header section of the control program.},
  kind={type,event},
  facet={\gls{string}}
}


\newglossaryentry{rotarymode event}
{
  type=mtc,
  category=model,
  name={ROTARY\_MODE},
  elementname=\cfont{RotaryMode},
  description={The current operating mode for a Rotary type axis.},
  kind={type,event},
  facet={\gls{string}},
  enumeration={\gls{spindle value},\gls{index value},\gls{contour value}}
}


\newglossaryentry{rotaryvelocityoverride event}
{
  type=mtc,
  category=model,
  name={ROTARY\_VELOCITY\_OVERRIDE},
  elementname=\cfont{RotaryVelocityOverride},
  description={The value of a command issued to adjust the programmed velocity for a \gls{rotary} type axis.This command represents a percentage change to the velocity calculated by a logic or motion program or set by a switch for a \gls{rotary} type axis.},
  kind={type,event},
  facet={\gls{float}}
}


\newglossaryentry{serialnumber event}
{
  type=mtc,
  category=model,
  name={SERIAL\_NUMBER},
  elementname=\cfont{SerialNumber},
  description={The serial number associated with a \gls{component}, \gls{asset}, or \gls{device}. The \gls{valid data value} \must be a text string.},
  kind={type,event},
  facet={\gls{string}}
}


\newglossaryentry{spindleinterlock event}
{
  type=mtc,
  category=model,
  name={SPINDLE\_INTERLOCK},
  elementname=\cfont{SpindleInterlock},
  description={An indication of the status of the spindle for a piece of equipment when power has been removed and it is free to rotate.},
  kind={type,event},
  facet={\gls{string}},
  enumeration={\gls{active value},\gls{inactive value}}
}


\newglossaryentry{toolassetid event}
{
  type=mtc,
  category=model,
  name={TOOL\_ASSET\_ID},
  elementname=\cfont{ToolAssetId},
  description={The identifier of an individual tool asset.The \gls{valid data value} \must be a text string.},
  kind={type,event},
  facet={\gls{string}}
}


\newglossaryentry{toolid event}
{
  type=mtc,
  category=model,
  name={TOOL\_ID},
  elementname=\cfont{ToolId},
  description={DEPRECATED in Version 1.2.0.   See \gls{toolassetid event}. \deprecated{The identifier of the tool currently in use for a given Path.}},
  deprecated={true},
  kind={type,event},
  facet={}
}


\newglossaryentry{toolnumber event}
{
  type=mtc,
  category=model,
  name={TOOL\_NUMBER},
  elementname=\cfont{ToolNumber},
  description={The identifier assigned by the \gls{controller} component to a cutting tool when in use by a piece of equipment. \newline The \gls{valid data value} \must be a text string.},
  kind={type,event},
  facet={\gls{string}}
}


\newglossaryentry{tooloffset event}
{
  type=mtc,
  category=model,
  name={TOOL\_OFFSET},
  elementname=\cfont{ToolOffset},
  description={A reference to the tool offset variables applied to the active cutting tool associated with a \gls{path} in a \gls{controller} type component.},
  kind={type,event},
  facet={\gls{float}}
}



\newglossaryentry{radial tooloffset event}
{
  type=mtc,
  category=model,
  name={RADIAL},
  description={A reference to a radial type tool offset variable.},
  kind={subtype,event},
  facet={\gls{float}}
}


\newglossaryentry{user event}
{
  type=mtc,
  category=model,
  name={USER},
  elementname=\cfont{User},
  description={The identifier of the person currently responsible for operating the piece of equipment.},
  kind={type,event},
  facet={\gls{string}}
}


\newglossaryentry{maintenance}
{
  type=mtc,
  category=model,
  name={MAINTENANCE},
  description={Action related to maintenance on the piece of equipment.},
  kind={subtype,enum,resettrigger}
}


\newglossaryentry{operator user event}
{
  type=mtc,
  category=model,
  name={OPERATOR},
  description={The identifier of the person currently responsible for operating the piece of equipment.},
  kind={subtype,event},
  facet={\gls{string}}
}


\newglossaryentry{setup user event}
{
  type=mtc,
  category=model,
  name={SET\_UP},
  description={The identifier of the person currently responsible for preparing a piece of equipment for production or restoring the piece of equipment to a neutral state after production.},
  kind={subtype,event},
  facet={\gls{string}}
}


\newglossaryentry{wire}
{
  type=mtc,
  category=model,
  name={WIRE},
  elementname=\cfont{Wire},
  description={A string like piece or filament of relatively rigid or flexible material provided in a variety of diameters.},
  kind={type,event,composition},
  facet={\gls{string}}
}


\newglossaryentry{workoffset event}
{
  type=mtc,
  category=model,
  name={WORK\_OFFSET},
  elementname=\cfont{WorkOffset},
  description={A reference to the offset variables for a work piece or part associated with a \gls{path} in a \gls{controller} type component.},
  kind={type,event},
  facet={\gls{float}}
}


\newglossaryentry{workholdingid event}
{
  type=mtc,
  category=model,
  name={WORKHOLDING\_ID},
  elementname=\cfont{WorkholdingId},
  description={The identifier for the current workholding or part clamp in use by a piece of equipment. \newline The \gls{valid data value} \must be a text string.},
  kind={type,event},
  facet={\gls{string}}
}



\newglossaryentry{communications condition}
{
  type=mtc,
  category=model,
  name={COMMUNICATIONS},
  elementname=\cfont{Communications},
  description={An indication that the piece of equipment has experienced a communications failure.},
  kind={type,condition}
}


\newglossaryentry{datarange condition}
{
  type=mtc,
  category=model,
  name={DATA\_RANGE},
  elementname=\cfont{DataRange},
  description={An indication that the value of the data associated with a measured value or a calculation is outside of an expected range.},
  kind={type,condition}
}



\newglossaryentry{hardware condition}
{
  type=mtc,
  category=model,
  name={HARDWARE},
  elementname=\cfont{Hardware},
  description={An indication of a fault associated with the hardware subsystem of the \gls{structural element}.},
  kind={type,condition}
}


\newglossaryentry{logicprogram condition}
{
  type=mtc,
  category=model,
  name={LOGIC\_PROGRAM},
  elementname=\cfont{LogicProgram},
  description={An indication that an error occurred in the logic program or programmable logic controller (PLC) associated with a piece of equipment.},
  kind={type,condition}
}


\newglossaryentry{motionprogram condition}
{
  type=mtc,
  category=model,
  name={MOTION\_PROGRAM},
  elementname=\cfont{MotionProgram},
  description={An indication that an error occurred in the motion program associated with a piece of equipment.},
  kind={type,condition}
}


\newglossaryentry{system condition}
{
  type=mtc,
  category=model,
  name={SYSTEM},
  elementname=\cfont{System},
  description={A general purpose indication associated with an electronic component of a piece of equipment or a controller that represents a fault that is not associated with the operator, program, or hardware.},
  kind={type,condition}
}


\newglossaryentry{event category}
{
  type=mtc,
  category=model,
  name={EVENT},
  kind={category},
  description={An \gls{event category} is a data item representing a discrete piece of information from the piece of equipment. }
}


\newglossaryentry{condition category}
{
  type=mtc,
  category=model,
  name={CONDITION},
  kind={category},
  description={A \gls{condition category} is a data item that communicates information about the health of a piece of equipment and its ability to function. }
}


\newglossaryentry{compositionid}
{
  type=mtc,
  category=model,
  name={compositionId},
  kind={attribute},
  description={The identifier attribute of the \gls{composition} element that the reported data is most closely associated. }
}


\newglossaryentry{actioncomplete}
{
  type=mtc,
  category=model,
  name={ACTION\_COMPLETE},
  kind={resettrigger},
  description={The value of the \gls{data entity} that is measuring an action or operation is to be reset upon completion of that action or operation.}
}


\newglossaryentry{annual}
{
  type=mtc,
  category=model,
  name={ANNUAL},
  kind={resettrigger},
  description={The value of the \gls{data entity} is to be reset at the end of a 12-month period.}
}


\newglossaryentry{day}
{
  type=mtc,
  category=model,
  name={DAY},
  kind={resettrigger},
  description={The value of the \gls{data entity} is to be reset at the end of a 24-hour period.}
}


\newglossaryentry{life}
{
  type=mtc,
  category=model,
  name={LIFE},
  kind={resettrigger},
  description={The value of the data item is not reset and accumulates for the entire life of the piece of equipment.}
}


\newglossaryentry{month}
{
  type=mtc,
  category=model,
  name={MONTH},
  kind={resettrigger},
  description={The value of the \gls{data entity} is to be reset at the end of a monthly period.}
}


\newglossaryentry{poweron}
{
  type=mtc,
  category=model,
  name={POWER\_ON},
  kind={resettrigger},
  description={The value of the \gls{data entity} is to be reset when power was applied to the piece of equipment after a planned or unplanned interruption of power has occurred.}
}


\newglossaryentry{shift}
{
  type=mtc,
  category=model,
  name={SHIFT},
  kind={resettrigger},
  description={The value of the \gls{data entity} is to be reset at the end of a work shift.}
}


\newglossaryentry{week}
{
  type=mtc,
  category=model,
  name={WEEK},
  kind={resettrigger},
  description={The value of the \gls{data entity} is to be reset at the end of a 7-day period.}
}


\newglossaryentry{warning}
{
  type=mtc,
  category=model,
  name={Warning},
  description={Warning value for a \gls{condition} element.}
}


\newglossaryentry{normal}
{
  type=mtc,
  category=model,
  name={Normal},
  description={Normal value for a \gls{condition} element.}
}


\newglossaryentry{fault}
{
  type=mtc,
  category=model,
  name={Fault},
  description={Fault value for a \gls{condition} element.}
}


\newglossaryentry{low}
{
  type=mtc,
  category=model,
  name={LOW},
  description={Low \gls{qualifier} value for a \gls{condition} element.}
}


\newglossaryentry{high}
{
  type=mtc,
  category=model,
  name={HIGH},
  description={High \gls{qualifier} value for a \gls{condition} element.}
}


\newglossaryentry{ampere}
{
  type=mtc,
  category=model,
  name={AMPERE},
  description={Amps},
  kind={units}
}


\newglossaryentry{celsius}
{
  type=mtc,
  category=model,
  name={CELSIUS},
  description={Degrees Celsius},
  kind={units}
}


\newglossaryentry{count}
{
  type=mtc,
  category=model,
  name={COUNT},
  description={A counted event},
  kind={units}
}


\newglossaryentry{decibel}
{
  type=mtc,
  category=model,
  name={DECIBEL},
  description={Sound Level},
  kind={units}
}


\newglossaryentry{degree}
{
  type=mtc,
  category=model,
  name={DEGREE},
  description={Angle in degrees},
  kind={units}
}


\newglossaryentry{degreepersecond}
{
  type=mtc,
  category=model,
  name={DEGREE/SECOND},
  description={Angular degrees per second},
  kind={units}
}


\newglossaryentry{degreepersecondsquared}
{
  type=mtc,
  category=model,
  name={DEGREE/SECOND$^2$},
  description={Angular acceleration in degrees per second squared},
  kind={units}
}


\newglossaryentry{hertz}
{
  type=mtc,
  category=model,
  name={HERTZ},
  description={Frequency measured in cycles per second},
  kind={units}
}


\newglossaryentry{joule}
{
  type=mtc,
  category=model,
  name={JOULE},
  description={A measurement of energy.},
  kind={units}
}


\newglossaryentry{kilogram}
{
  type=mtc,
  category=model,
  name={KILOGRAM},
  description={Kilograms},
  kind={units}
}



\newglossaryentry{literpersecond}
{
  type=mtc,
  category=model,
  name={LITER/SECOND},
  description={Liters per second},
  kind={units}
}


\newglossaryentry{microradian}
{
  type=mtc,
  category=model,
  name={MICRO\_RADIAN},
  description={Measurement of Tilt},
  kind={units}
}


\newglossaryentry{millimeter}
{
  type=mtc,
  category=model,
  name={MILLIMETER},
  description={Millimeters},
  kind={units}
}


\newglossaryentry{millimeterpersecond}
{
  type=mtc,
  category=model,
  name={MILLIMETER/SECOND},
  description={Millimeters per second},
  kind={units}
}


\newglossaryentry{millimeterpersecondsquared}
{
  type=mtc,
  category=model,
  name={MILLIMETER/SECOND$^2$},
  description={Acceleration in millimeters per second squared},
  kind={units}
}


\newglossaryentry{millimeter3d}
{
  type=mtc,
  category=model,
  name={MILLIMETER\_3D},
  description={A point in space identified by X, Y, and Z positions and represented by a space-delimited set of numbers each expressed in millimeters.},
  kind={units}
}


\newglossaryentry{newton}
{
  type=mtc,
  category=model,
  name={NEWTON},
  description={Force in Newtons},
  kind={units}
}


\newglossaryentry{newtonmeter}
{
  type=mtc,
  category=model,
  name={NEWTON\_METER},
  description={Torque, a unit for force times distance.},
  kind={units}
}


\newglossaryentry{ohm}
{
  type=mtc,
  category=model,
  name={OHM},
  description={Measure of Electrical Resistance},
  kind={units}
}


\newglossaryentry{pascal}
{
  type=mtc,
  category=model,
  name={PASCAL},
  description={Pressure in Newtons per square meter},
  kind={units}
}


\newglossaryentry{pascalsecond}
{
  type=mtc,
  category=model,
  name={PASCAL\_SECOND},
  description={Measurement of Viscosity},
  kind={units}
}


\newglossaryentry{percent}
{
  type=mtc,
  category=model,
  name={PERCENT},
  description={Percentage},
  kind={units}
}


\newglossaryentry{revolutionperminute}
{
  type=mtc,
  category=model,
  name={REVOLUTION/MINUTE},
  description={Revolutions per minute},
  kind={units}
}


\newglossaryentry{second}
{
  type=mtc,
  category=model,
  name={SECOND},
  description={A measurement of time.},
  kind={units}
}


\newglossaryentry{siemenspermeter}
{
  type=mtc,
  category=model,
  name={SIEMENS/METER},
  description={A measurement of Electrical Conductivity},
  kind={units}
}


\newglossaryentry{volt}
{
  type=mtc,
  category=model,
  name={VOLT},
  description={Volts},
  kind={units}
}



\newglossaryentry{watt}
{
  type=mtc,
  category=model,
  name={WATT},
  description={Watts},
  kind={units}
}


\newglossaryentry{wattsecond}
{
  type=mtc,
  category=model,
  name={WATT\_SECOND},
  description={Measurement of electrical energy, equal to one Joule},
  kind={units}
}


\newglossaryentry{centipoise}
{
  type=mtc,
  category=model,
  name={CENTIPOISE},
  description={A measure of Viscosity},
  kind={nativeUnits}
}


\newglossaryentry{degreeperminute}
{
  type=mtc,
  category=model,
  name={DEGREE/MINUTE},
  description={Rotational velocity in degrees per minute},
  kind={nativeUnits}
}


\newglossaryentry{fahrenheit}
{
  type=mtc,
  category=model,
  name={FAHRENHEIT},
  description={Temperature in Fahrenheit},
  kind={nativeUnits}
}


\newglossaryentry{foot}
{
  type=mtc,
  category=model,
  name={FOOT},
  description={Feet},
  kind={nativeUnits}
}


\newglossaryentry{footperminute}
{
  type=mtc,
  category=model,
  name={FOOT/MINUTE},
  description={Feet per minute},
  kind={nativeUnits}
}


\newglossaryentry{footpersecond}
{
  type=mtc,
  category=model,
  name={FOOT/SECOND},
  description={Feet per second},
  kind={nativeUnits}
}


\newglossaryentry{footpersecondsquared}
{
  type=mtc,
  category=model,
  name={FOOT/SECOND$^2$},
  description={Acceleration in feet per second squared},
  kind={nativeUnits}
}


\newglossaryentry{foot3d}
{
  type=mtc,
  category=model,
  name={FOOT\_3D},
  description={A point in space identified by X, Y, and Z positions and represented by a space-delimited set of numbers each expressed in feet.},
  kind={nativeUnits}
}


\newglossaryentry{gallonperminute}
{
  type=mtc,
  category=model,
  name={GALLON/MINUTE},
  description={Gallons per minute.},
  kind={nativeUnits}
}


\newglossaryentry{inch}
{
  type=mtc,
  category=model,
  name={INCH},
  description={Inches},
  kind={nativeUnits}
}


\newglossaryentry{inchperminute}
{
  type=mtc,
  category=model,
  name={INCH/MINUTE},
  description={Inches per minute},
  kind={nativeUnits}
}


\newglossaryentry{inchpersecond}
{
  type=mtc,
  category=model,
  name={INCH/SECOND},
  description={Inches per second},
  kind={nativeUnits}
}


\newglossaryentry{inchpersecondsquared}
{
  type=mtc,
  category=model,
  name={INCH/SECOND$^2$},
  description={Acceleration in inches per second squared},
  kind={nativeUnits}
}


\newglossaryentry{inch3d}
{
  type=mtc,
  category=model,
  name={INCH\_3D},
  description={A point in space identified by X, Y, and Z positions and represented by a space-delimited set of numbers each expressed in inches.},
  kind={nativeUnits}
}


\newglossaryentry{inchpound}
{
  type=mtc,
  category=model,
  name={INCH\_POUND},
  description={A measure of torque in inch pounds.},
  kind={nativeUnits}
}


\newglossaryentry{kelvin}
{
  type=mtc,
  category=model,
  name={KELVIN},
  description={A measurement of temperature},
  kind={nativeUnits}
}


\newglossaryentry{kilowatt}
{
  type=mtc,
  category=model,
  name={KILOWATT},
  description={A measurement in kilowatt.},
  kind={nativeUnits}
}


\newglossaryentry{kilowatthour}
{
  type=mtc,
  category=model,
  name={KILOWATT\_HOUR},
  description={Kilowatt hours which is 3.6 mega joules.},
  kind={nativeUnits}
}


\newglossaryentry{liter}
{
  type=mtc,
  category=model,
  name={LITER},
  description={Measurement of volume of a fluid},
  kind={units,nativeUnits}
}


\newglossaryentry{literperminute}
{
  type=mtc,
  category=model,
  name={LITER/MINUTE},
  description={Measurement of rate of flow of a fluid},
  kind={nativeUnits}
}


\newglossaryentry{millimeterperminute}
{
  type=mtc,
  category=model,
  name={MILLIMETER/MINUTE},
  description={Velocity in millimeters per minute},
  kind={nativeUnits}
}


\newglossaryentry{other}
{
  type=mtc,
  category=model,
  name={OTHER},
  description={Unsupported units},
  kind={nativeUnits}
}


\newglossaryentry{pound}
{
  type=mtc,
  category=model,
  name={POUND},
  description={US pounds},
  kind={nativeUnits}
}


\newglossaryentry{poundperinchsquared}
{
  type=mtc,
  category=model,
  name={POUND/INCH$^2$},
  description={Pressure in pounds per square inch (PSI).},
  kind={nativeUnits}
}


\newglossaryentry{radian}
{
  type=mtc,
  category=model,
  name={RADIAN},
  description={Angle in radians},
  kind={nativeUnits}
}


\newglossaryentry{radianperminute}
{
  type=mtc,
  category=model,
  name={RADIAN/MINUTE},
  description={Velocity in radians per minute.},
  kind={nativeUnits}
}


\newglossaryentry{radianpersecond}
{
  type=mtc,
  category=model,
  name={RADIAN/SECOND},
  description={Velocity in radians per second},
  kind={nativeUnits}
}


\newglossaryentry{radianpersecondsquared}
{
  type=mtc,
  category=model,
  name={RADIAN/SECOND$^2$},
  description={Rotational acceleration in radian per second squared},
  kind={nativeUnits}
}


\newglossaryentry{revolutionpersecond}
{
  type=mtc,
  category=model,
  name={REVOLUTION/SECOND},
  description={Rotational velocity in revolution per second},
  kind={nativeUnits}
}


\newglossaryentry{active value}
{
  type=mtc,
  category=model,
  name={ACTIVE},
  description={The value of the \gls{data entity} that is engaging.},
  kind={enum}
}


\newglossaryentry{inactive value}
{
  type=mtc,
  category=model,
  name={INACTIVE},
  description={The value of the \gls{data entity} that is not engaging.},
  kind={enum}
}


\newglossaryentry{available value}
{
  type=mtc,
  category=model,
  name={AVAILABLE},
  description={The value or status of an XML element when it is available.},
  kind={enum}
}


\newglossaryentry{tandem value}
{
  type=mtc,
  category=model,
  name={TANDEM},
  description={Elements are physically connected to each other and operate as a single unit.},
  kind={enum}
}


\newglossaryentry{synchronous value}
{
  type=mtc,
  category=model,
  name={SYNCHRONOUS},
  description={Physical or logical parts which are not physically connected to each other but are operating together.},
  kind={enum}
}


\newglossaryentry{master value}
{
  type=mtc,
  category=model,
  name={MASTER},
  description={It provides information or state values that influences the operation of other \gls{dataitem} of similar type.},
  kind={enum}
}


\newglossaryentry{slave value}
{
  type=mtc,
  category=model,
  name={SLAVE},
  description={The axis is a slave to the \gls{coupledaxes event}},
  kind={enum}
}


\newglossaryentry{home value}
{
  type=mtc,
  category=model,
  name={HOME},
  description={The component at its home position.},
  kind={enum}
}


\newglossaryentry{travel value}
{
  type=mtc,
  category=model,
  name={TRAVEL},
  description={The component is in motion.},
  kind={enum}
}


\newglossaryentry{parked value}
{
  type=mtc,
  category=model,
  name={PARKED},
  description={The component has been moved to a fixed position.},
  kind={enum}
}


\newglossaryentry{stopped value}
{
  type=mtc,
  category=model,
  name={STOPPED},
  description={The component is stopped.},
  kind={enum}
}


\newglossaryentry{open value}
{
  type=mtc,
  category=model,
  name={OPEN},
  description={A component is open to the point of a positive confirmation.},
  kind={enum}
}


\newglossaryentry{closed value}
{
  type=mtc,
  category=model,
  name={CLOSED},
  description={A component is closed to the point of a positive confirmation.},
  kind={enum}
}


\newglossaryentry{unlatched value}
{
  type=mtc,
  category=model,
  name={UNLATCHED},
  description={An intermediate position.},
  kind={enum}
}


\newglossaryentry{right value}
{
  type=mtc,
  category=model,
  name={RIGHT},
  description={The position of the Composition element is oriented to the right to the point of a positive confirmation.},
  kind={enum}
}


\newglossaryentry{left value}
{
  type=mtc,
  category=model,
  name={LEFT},
  description={The position of the Composition element is oriented to the left to the point of a positive confirmation.},
  kind={enum}
}


\newglossaryentry{transitioning value}
{
  type=mtc,
  category=model,
  name={TRANSITIONING},
  description={It is in an intermediate position of the Composition element.},
  kind={enum}
}


\newglossaryentry{on value}
{
  type=mtc,
  category=model,
  name={ON},
  description={On state or value.},
  kind={enum}
}


\newglossaryentry{off value}
{
  type=mtc,
  category=model,
  name={OFF},
  description={Off state or value.},
  kind={enum}
}


\newglossaryentry{up value}
{
  type=mtc,
  category=model,
  name={UP},
  description={Increase in the behavior of a \gls{structural element}.},
  kind={enum}
}


\newglossaryentry{down value}
{
  type=mtc,
  category=model,
  name={DOWN},
  description={Reduction in the behavior of a \gls{structural element}.},
  kind={enum}
}


\newglossaryentry{automatic value}
{
  type=mtc,
  category=model,
  name={AUTOMATIC},
  description={The \gls{controller} is configured to automatically execute a program.},
  kind={enum}
}


\newglossaryentry{manual value}
{
  type=mtc,
  category=model,
  name={MANUAL},
  description={Operations based on the instructions received from an external source.},
  kind={enum}
}


\newglossaryentry{manualdatainput value}
{
  type=mtc,
  category=model,
  name={MANUAL\_DATA\_INPUT},
  description={The operator can enter a series of operations for the controller to perform.},
  kind={enum}
}


\newglossaryentry{semiautomatic value}
{
  type=mtc,
  category=model,
  name={SEMI\_AUTOMATIC},
  description={The controller  executes a single set of instructions from an active program and then stops until given a command to execute the next set of instructions.},
  kind={enum}
}


\newglossaryentry{edit value}
{
  type=mtc,
  category=model,
  name={EDIT},
  description={The controller is currently functioning as a programming device and is not capable of executing an active program.},
  kind={enum}
}


\newglossaryentry{clockwise value}
{
  type=mtc,
  category=model,
  name={CLOCKWISE},
  description={A \gls{rotary direction event} component type rotating in a clockwise fashion using the right-hand rule.},
  kind={enum}
}


\newglossaryentry{counterclockwise value}
{
  type=mtc,
  category=model,
  name={COUNTER\_CLOCKWISE},
  description={A \gls{rotary direction event} component type rotating in a counter clockwise fashion using the right-hand rule.},
  kind={enum}
}


\newglossaryentry{positive value}
{
  type=mtc,
  category=model,
  name={POSITIVE},
  description={A \gls{linear subtype} type component is moving in the direction of increasing position value.},
  kind={enum}
}


\newglossaryentry{negative value}
{
  type=mtc,
  category=model,
  name={NEGATIVE},
  description={A \gls{linear subtype} type component is moving in the direction of decreasing position value.},
  kind={enum}
}


\newglossaryentry{armed value}
{
  type=mtc,
  category=model,
  name={ARMED},
  description={The emergency stop circuit is complete and the piece of equipment, component, or composition element is allowed to operate. },
  kind={enum}
}


\newglossaryentry{triggered value}
{
  type=mtc,
  category=model,
  name={TRIGGERED},
  description={The operation of the piece of equipment, component, or composition element is inhibited.},
  kind={enum}
}


\newglossaryentry{ready value}
{
  type=mtc,
  category=model,
  name={READY},
  description={A component is ready to engage.},
  kind={enum}
}


\newglossaryentry{interrupted value}
{
  type=mtc,
  category=model,
  name={INTERRUPTED},
  description={The action of a Component has been suspended due to an external signal.},
  kind={enum}
}


\newglossaryentry{feedhold value}
{
  type=mtc,
  category=model,
  name={FEED\_HOLD},
  description={Motion of a Component has been commanded to stop at its current position.},
  kind={enum}
}


\newglossaryentry{optionalstop value}
{
  type=mtc,
  category=model,
  name={OPTIONAL\_STOP},
  description={The controller’s program has been intentionally stopped},
  kind={enum}
}


\newglossaryentry{programstopped value}
{
  type=mtc,
  category=model,
  name={PROGRAM\_STOPPED},
  description={The execution of the \gls{controller}'s program has been stopped by a command from within the program.},
  kind={enum}
}


\newglossaryentry{programcompleted value}
{
  type=mtc,
  category=model,
  name={PROGRAM\_COMPLETED},
  description={The execution of the controller’s program has been stopped by a command from within the program.},
  kind={enum}
}


\newglossaryentry{production value}
{
  type=mtc,
  category=model,
  name={PRODUCTION},
  description={A \gls{structural element} is currently producing product.},
  kind={enum}
}


\newglossaryentry{setup value}
{
  type=mtc,
  category=model,
  name={SETUP},
  description={A \gls{structural element} is being prepared or modified to begin production of product.},
  kind={enum}
}


\newglossaryentry{teardown value}
{
  type=mtc,
  category=model,
  name={TEARDOWN},
  description={Typically, a \gls{structural element} has completed the production of a product and is being modified or returned to a neutral state such that it may then be prepared to begin production of a different product.},
  kind={enum}
}


\newglossaryentry{processdevelopment value}
{
  type=mtc,
  category=model,
  name={PROCESS\_DEVELOPMENT},
  description={A \gls{structural element} is being used to prove-out a new process.},
  kind={enum}
}


\newglossaryentry{disabled value}
{
  type=mtc,
  category=model,
  name={DISABLED},
  description={A component is currently not operational.},
  kind={enum}
}


\newglossaryentry{enabled value}
{
  type=mtc,
  category=model,
  name={ENABLED},
  description={A component is currently operational and performing as expected.},
  kind={enum}
}


\newglossaryentry{independent value}
{
  type=mtc,
  category=model,
  name={INDEPENDENT},
  description={The path is operating independently and without the influence of another path.},
  kind={enum}
}


\newglossaryentry{mirror value}
{
  type=mtc,
  category=model,
  name={MIRROR},
  description={The axes associated with the path are mirroring the motion of the \gls{master value} path.},
  kind={enum}
}


\newglossaryentry{yes value}
{
  type=mtc,
  category=model,
  name={YES},
  description={The \gls{endofbar event} has been reached.},
  kind={enum}
}


\newglossaryentry{no value}
{
  type=mtc,
  category=model,
  name={NO},
  description={The \gls{endofbar event} has not been reached.},
  kind={enum}
}


\newglossaryentry{notready value}
{
  type=mtc,
  category=model,
  name={NOT\_READY},
  description={A component is not ready to engage.},
  kind={enum}
}


\newglossaryentry{spindle value}
{
  type=mtc,
  category=model,
  name={SPINDLE},
  description={The axis is functioning as a spindle.},
  kind={enum}
}


\newglossaryentry{index value}
{
  type=mtc,
  category=model,
  name={INDEX},
  description={The axis is configured to index.},
  kind={enum}
}


\newglossaryentry{contour value}
{
  type=mtc,
  category=model,
  name={CONTOUR},
  description={The position of the axis is being interpolated.},
  kind={enum}
}


\newglossaryentry{mtconnectasset}
{
  type=mtc,
  category=model,
  name={MTConnectAsset},
  plural={MTConnectAssets},
  description={XML response document}
}


\newglossaryentry{asset information model}
{
  type=mtc,
  name={Asset Information Model},
  description={It associates each electronic MTConnectAssets document with a unique identifier and allows for some predefined mechanisms to find, create, request, updated, and delete these electronic documents in a way that provides for consistency across multiple pieces of equipment.}
}


\newglossaryentry{assetid}
{
  type=mtc,
  category=model,
  name={assetId},
  description={The unique identifier for the \gls{mtconnect asset}. The identifier \MUST be unique with respect to all other \glspl{asset} in an MTConnect installation. The identifier \SHOULD be globally unique with respect to all other \glspl{asset}.}
}


\newglossaryentry{version mtconnectassets}
{
  type=mtc,
  category=model,
  name={version},
  description={The protocol version number. This is the \gls{major} and \gls{minor} version number of the MTConnect Standard being used. For example, if the version number of the Standard used is \cfont{10.21.33}, the \gls{version mtconnectassets} will be \cfont{10.21}.}
}


\newglossaryentry{major}
{
  type=mtc,
  name={major},
  description={major version}
}


\newglossaryentry{minor}
{
  type=mtc,
  name={minor},
  description={minor version}
}


\newglossaryentry{creationtime mtconnectassets}
{
  type=mtc,
  category=model,
  name={creationTime},
  description={The time the response was created. }
}


\newglossaryentry{testindicator mtconnectassets}
{
  type=mtc,
  category=model,
  name={testIndicator},
  description={Optional flag that indicates the system is operating in test mode. This data is only for testing and indicates that the data is simulated. }
}


\newglossaryentry{instanceid mtconnectassets}
{
  type=mtc,
  category=model,
  name={instanceId},
  description={A number indicating which invocation of the \gls{agent}. This is used to differentiate between separate instances of the \gls{agent}. This value \MUST have a maximum value of $2^{64}-1$ and \MUST be stored in an unsigned 64-bit integer. }
}


\newglossaryentry{sender mtconnectassets}
{
  type=mtc,
  category=model,
  name={sender},
  description={The \gls{agent} identification information. }
}


\newglossaryentry{assetbuffersize mtconnectassets}
{
  type=mtc,
  category=model,
  name={assetBufferSize},
  description={The maximum number of \glspl{mtconnect asset} that will be retained by the \gls{agent}. The \gls{assetbuffersize mtconnectassets} \MUST be an unsigned positive integer value with a maximum value of $2^{32}-1$. }
}


\newglossaryentry{assetcount mtconnectassets}
{
  type=mtc,
  category=model,
  name={assetCount},
  description={The total number of \glspl{mtconnect asset} in an\gls{agent}. This \MUST be an unsigned positive integer value with a maximum value of $2^{32}-1$. This value \MUSTNOT be greater than \gls{assetbuffersize mtconnectassets}.}
}


\newglossaryentry{asset mtconnectassets}
{
  type= mtc,
  category=model,
  name= {Asset},
  description= {An abstract XML element. Replaced in the XML document by types of \gls{asset mtconnectassets} elements representing entities that are not pieces of equipment.}
}

\newglossaryentry{assets mtconnectassets}
{
  type= mtc,
  category=model,
  name= {Assets},
  description={XML container that consists of one or more types of \gls{asset mtconnectassets} XML elements. }
}


\newglossaryentry{deviceuuid}
{
  type=mtc,
  category=model,
  name={deviceUuid},
  description={The piece of equipments UUID that supplied this data. This is an optional element references to the UUID attribute given in the \gls{device} element. This can be any series of numbers and letters as defined by the XML type \gls{nmtoken}.}
}


\newglossaryentry{removed}
{
  type=mtc,
  category=model,
  name={removed},
  description={An indicator that the \gls{mtconnect asset} has been removed from the piece of equipment.}
}


\newglossaryentry{asset buffer}
{
  type=mtc,
  name={asset buffer},
  description={A buffer or queue for \gls{assets mtconnectassets}.}
}


\newglossaryentry{cuttingtool}
{
  type=mtc,
  category=model,
  name={CuttingTool},
  description={A \gls{cuttingtool} physically removes the material from the workpiece by shear deformation.}
}


\newglossaryentry{assetchanged event}
{
  type=mtc,
  category=model,
  name={ASSET\_CHANGED},
  elementname={AssetChanged},
  description={The value of the \gls{cdata} for the event \MUST be the \gls{assetid} of the asset that has been added or changed. There will not be a separate message for new assets.},
  kind={type,event},
  facet={\gls{string}}
}


\newglossaryentry{assetremoved event}
{
  type=mtc,
  category=model,
  name={ASSET\_REMOVED},
  elementname={AssetRemoved},
  description={The value of the \gls{cdata} for the event \MUST be the \gls{assetid} of the asset that has been removed. The asset will still be visible if requested with the \gls{includeremoved} parameter as described in the protocol section. When assets are removed they are not moved to the beginning of the most recently modified list. },
  kind={type,event},
  facet={\gls{string}}
}


\newglossaryentry{includeremoved}
{
  type=mtc,
  category=model,
  name={includeRemoved},
  description={A flag to include removed \gls{assets mtconnectassets} while requesting an \gls{agent} for an \glspl{mtconnectstream} \gls{response document}.}
}


\newglossaryentry{assettype}
{
  type=mtc,
  category=model,
  name={assetType},
  description={The type of asset that was updated.}
}


\newglossaryentry{cuttingtoolarchetype}
{
  type=mtc,
  category=model,
  name={CuttingToolArchetype},
  description={The \gls{cuttingtoolarchetype} represent the static Cutting Tool geometries and nominal values as one would expect from a tool catalog.}
}


\newglossaryentry{assetid cuttingtool}
{
  type=mtc,
  category=model,
  name={assetId},
  description={The unique identifier of the instance of this tool. This will be the same as the \gls{toolid cuttingtool} and \gls{serialnumber} in most cases. The \gls{assetid cuttingtool} \SHOULD be the combination of the \gls{toolid cuttingtool} and \gls{serialnumber} as in \gls{toolid cuttingtool}. \gls{serialnumber} or an equivalent implementation dependent identification scheme.}
}


\newglossaryentry{toolid cuttingtool}
{
  type=mtc,
  category=model,
  name={toolId},
  description={The identifier for a class of Cutting Tools. This is defined as an XML string type and is implementation dependent. }
}


\newglossaryentry{manufacturers cuttingtool}
{
  type=mtc,
  category=model,
  name={manufacturers},
  description={An optional attribute referring to the manufacturer(s) of this Cutting Tool, for this element, this will reference the Tool Item and Adaptive Items specifically. }
}


\newglossaryentry{cuttingitem}
{
  type=mtc,
  category=model,
  name={CuttingItem},
  description={A \gls{cuttingitem} is the portion of the tool that physically removes the material from the workpiece by shear deformation.}
}

\newglossaryentry{cuttingitems}
{
  type=mtc,
  category=model,
  name={CuttingItems},
  description={An optional set of individual Cutting Items.}
}

\newglossaryentry{cuttingtooldefinition}
{
  type=mtc,
  category=model,
  name={CuttingToolDefinition},
  description={Reference to an ISO 13399.}
}


\newglossaryentry{cutterstatus}
{
  type=mtc,
  category=model,
  name={CutterStatus},
  description={The status of this assembly.}
}


\newglossaryentry{status cutterstatus}
{
  type=mtc,
  category=model,
  name={Status},
  description={The status of the Cutting Tool.}
}


\newglossaryentry{toollife}
{
  type=mtc,
  category=model,
  name={ToolLife},
  description={The Cutting Tool life as related to this assembly.}
}


\newglossaryentry{location}
{
  type=mtc,
  category=model,
  name={Location},
  description={The Pot or Spindle this tool currently resides in.}
}


\newglossaryentry{reconditioncount}
{
  type=mtc,
  category=model,
  name={ReconditionCount},
  description={The number of times this cutter has been reconditioned.}
}


\newglossaryentry{cuttingtoollifecycle}
{
  type=mtc,
  category=model,
  name={CuttingToolLifeCycle},
  description={Data regarding the use of this tool.}
}


\newglossaryentry{format cuttingtooldefinition}
{
  type=mtc,
  category=model,
  name={format},
  description={Identifies the expected representation of the enclosed data.}
}


\newglossaryentry{xml format}
{
  type=mtc,
  category=model,
  name={XML},
  description={The default value for the definition. The content will be an XML document.}
}


\newglossaryentry{express format}
{
  type=mtc,
  category=model,
  name={EXPRESS},
  description={The document will confirm to the ISO 10303 Part 21 standard.}
}


\newglossaryentry{text format}
{
  type=mtc,
  category=model,
  name={TEXT},
  description={The document will be a text representation of the tool data.}
}


\newglossaryentry{undefined format}
{
  type=mtc,
  category=model,
  name={UNDEFINED},
  description={The document will be provided in an undefined format.}
}


\newglossaryentry{cuttingtooldefinition deprecated}
{
  type=mtc,
  category=model,
  name=\deprecated{CuttingToolDefinition},
  deprecated={true},
  description={\DEPRECATED for \gls{cuttingtool} in Version 1.3.0.   \newline \deprecated{Reference to an ISO 13399.}}
}


\newglossaryentry{cuttingtoolarchetypereference}
{
  type=mtc,
  category=model,
  name={CuttingToolArchetypeReference},
  description={The content of this XML element is the \gls{assetid cuttingtool} of the \gls{cuttingtoolarchetype} document. It \MAY also contain a source attribute that gives the URL of the archetype data as well.}
}


\newglossaryentry{new status}
{
  type=mtc,
  category=model,
  name={NEW},
  description={A new tool that has not been used or first use. Marks the start of the tool history.}
}



\newglossaryentry{allocated status}
{
  type=mtc,
  category=model,
  name={ALLOCATED},
  description={Indicates if this tool is has been committed to a piece of equipment for use and is not available for use in any other piece of equipment. If this is not present, this tool has not been allocated for this piece of equipment and can be used by another piece of equipment.}
}


\newglossaryentry{unallocated status}
{
  type=mtc,
  category=model,
  name={UNALLOCATED},
  description={Indicates this Cutting Tool has not been committed to a process and can be allocated.}
}


\newglossaryentry{measured status}
{
  type=mtc,
  category=model,
  name={MEASURED},
  description={The tool has been measured.}
}


\newglossaryentry{reconditioned status}
{
  type=mtc,
  category=model,
  name={RECONDITIONED},
  description={The Cutting Tool has been reconditioned. See \gls{reconditioncount} for the number of times this cutter has been reconditioned.}
}


\newglossaryentry{used status}
{
  type=mtc,
  category=model,
  name={USED},
  description={The Cutting Tool is in process and has remaining tool life.}
}


\newglossaryentry{expired status}
{
  type=mtc,
  category=model,
  name={EXPIRED},
  description={The Cutting Tool has reached the end of its useful life.}
}


\newglossaryentry{broken status}
{
  type=mtc,
  category=model,
  name={BROKEN},
  description={Premature tool failure.}
}


\newglossaryentry{notregistered status}
{
  type=mtc,
  category=model,
  name={NOT\_REGISTERED},
  description={This Cutting Tool cannot be used until it is entered into the system.}
}


\newglossaryentry{unknown status}
{
  type=mtc,
  category=model,
  name={UNKNOWN},
  description={The Cutting Tool is an indeterminate state. This is the default value.}
}


\newglossaryentry{countdirection}
{
  type=mtc,
  category=model,
  name={countDirection},
  description={Indicates if the tool life counts from zero to maximum or maximum to zero.}
}


\newglossaryentry{warning toollife}
{
  type=mtc,
  category=model,
  name={warning},
  description={The point at which a tool life warning will be raised.}
}


\newglossaryentry{limit}
{
  type=mtc,
  category=model,
  name={limit},
  description={The end of life limit for this tool.}
}


\newglossaryentry{initial}
{
  type=mtc,
  category=model,
  name={initial},
  description={The initial life of the tool when it is new.}
}


\newglossaryentry{minutes type}
{
  type=mtc,
  category=model,
  name={MINUTES},
  description={The tool life measured in minutes. All units for minimum, maximum, and nominal \MUST be provided in minutes.}
}


\newglossaryentry{wear type}
{
  type=mtc,
  category=model,
  name={WEAR},
  description={The tool life measured in tool wear. Wear \MUST be provided in millimeters as an offset to nominal. All units for minimum, maximum, and nominal \MUST be given as millimeter offsets as well. }
}



\newglossaryentry{positiveoverlap}
{
  type=mtc,
  category=model,
  name={positiveOverlap},
  description={The number of locations at higher index value from this location.}
}


\newglossaryentry{negativeoverlap}
{
  type=mtc,
  category=model,
  name={negativeOverlap},
  description={The number of location at lower index values from this location.}
}


\newglossaryentry{pot type}
{
  type=mtc,
  category=model,
  name={POT},
  description={The number of the pot in the tool handling system.}
}


\newglossaryentry{station type}
{
  type=mtc,
  category=model,
  name={STATION},
  description={The tool location in a horizontal turning machine.}
}


\newglossaryentry{crib type}
{
  type=mtc,
  category=model,
  name={CRIB},
  description={The location with regard to a tool crib.}
}


\newglossaryentry{maximumcount}
{
  type=mtc,
  category=model,
  name={maximumCount},
  description={The maximum number of times this tool may be reconditioned.}
}



\newglossaryentry{programtoolgroup}
{
  type=mtc,
  category=model,
  name={ProgramToolGroup},
  description={The tool group this tool is assigned in the part program.}
}


\newglossaryentry{programtoolnumber}
{
  type=mtc,
  category=model,
  name={ProgramToolNumber},
  description={The number of the tool as referenced in the part program.}
}


\newglossaryentry{processspindlespeed}
{
  type=mtc,
  category=model,
  name={ProcessSpindleSpeed},
  description={The constrained process spindle speed for this tool.}
}


\newglossaryentry{processfeedrate}
{
  type=mtc,
  category=model,
  name={ProcessFeedRate},
  description={The constrained process feed rate for this tool in mm/s.}
}


\newglossaryentry{connectioncodemachineside}
{
  type=mtc,
  category=model,
  name={ConnectionCodeMachineSide},
  description={Identifier for the capability to connect any component of the Cutting Tool together, except Assembly Items, on the machine side. Code: \cfont{CCMS}}
}


\newglossaryentry{measurement}
{
  type=mtc,
  category=model,
  name={Measurement},
  description={A measure of a \gls{structural element}.}
}

\newglossaryentry{measurements}
{
  type=mtc,
  category=model,
  name={Measurements},
  description={The \gls{measurements} element is a collection of one or more constrained scalar values associated with this Cutting Tool.}
}


\newglossaryentry{xs:any}
{
  type=mtc,
  category=model,
  name={xs:any},
  description={Any additional properties not in the current document model. \MUST be in separate XML namespace.}
}


\newglossaryentry{maximum attribute}
{
  type=mtc,
  category=model,
  name={maximum},
  kind={attribute},
  description={The upper bound for the value of a \gls{structural element}.}
}


\newglossaryentry{minimum attribute}
{
  type=mtc,
  category=model,
  name={minimum},
  kind={attribute},
  description={The lower bound for value of a \gls{structural element}.}
}


\newglossaryentry{nominal attribute}
{
  type=mtc,
  category=model,
  name={nominal},
  description={The nominal value for a \gls{structural element}.}
}




\newglossaryentry{commonmeasurement}
{
  type=mtc,
  category=model,
  name={CommonMeasurement},
  description={A subtype of \gls{measurement}.}
}


\newglossaryentry{assemblymeasurement}
{
  type=mtc,
  category=model,
  name={AssemblyMeasurement},
  description={A subtype of \gls{measurement}.}
}


\newglossaryentry{cuttingitemmeasurement}
{
  type=mtc,
  category=model,
  name={CuttingItemMeasurement},
  description={A subtype of \gls{measurement}.}
}


\newglossaryentry{code measurement}
{
  type=mtc,
  category=model,
  name={code},
  description={A shop specific code for this measurement. ISO 13399 codes \MAY be used for these codes as well.}
}



\newglossaryentry{bodydiametermax}
{
  type=mtc,
  category=model,
  name={BodyDiameterMax},
  code=\cfont{BDX},
  description={The largest diameter of the body of a Tool Item. },
  units=\cfont{\gls{millimeter}}
}


\newglossaryentry{bodylengthmax}
{
  type=mtc,
  category=model,
  name={BodyLengthMax},
  code=\cfont{LBX},
  description={The distance measured along the X axis from that point of the item closest to the workpiece, including the Cutting Item for a Tool Item but excluding a protruding locking mechanism for an Adaptive Item, to either the front of the flange on a flanged body or the beginning of the connection interface feature on the machine side for cylindrical or prismatic shanks.},
  units=\cfont{\gls{millimeter}}
}


\newglossaryentry{depthofcutmax}
{
  type=mtc,
  category=model,
  name={DepthOfCutMax},
  code=\cfont{APMX},
  description={The maximum engagement of the cutting edge or edges with the workpiece measured perpendicular to the feed motion. },
  units=\cfont{\gls{millimeter}}
}


\newglossaryentry{cuttingdiametermax}
{
  type=mtc,
  category=model,
  name={CuttingDiameterMax},
  code=\cfont{DC},
  description={The maximum diameter of a circle on which the defined point Pk of each of the master inserts is located on a Tool Item. The normal of the machined peripheral surface points towards the axis of the Cutting Tool. },
  units=\cfont{\gls{millimeter}}
}


\newglossaryentry{flangediametermax}
{
  type=mtc,
  category=model,
  name={FlangeDiameterMax},
  code=\cfont{DF},
  description={The dimension between two parallel tangents on the outside edge of a flange. },
  units=\cfont{\gls{millimeter}}
}


\newglossaryentry{overalltoollength}
{
  type=mtc,
  category=model,
  name={OverallToolLength},
  code=\cfont{OAL},
  description={The largest length dimension of the Cutting Tool including the master insert where applicable.  },
  units=\cfont{\gls{millimeter}}
}


\newglossaryentry{shankdiameter}
{
  type=mtc,
  category=model,
  name={ShankDiameter},
  code=\cfont{DMM},
  description={The dimension of the diameter of a cylindrical portion of a Tool Item or an Adaptive Item that can participate in a connection. },
  units=\cfont{\gls{millimeter}}
}


\newglossaryentry{shankheight}
{
  type=mtc,
  category=model,
  name={ShankHeight},
  code=\cfont{H},
  description={The dimension of the height of the shank. },
  units=\cfont{\gls{millimeter}}
}


\newglossaryentry{shanklength}
{
  type=mtc,
  category=model,
  name={ShankLength},
  code=\cfont{LS},
  description={The dimension of the length of the shank. },
  units=\cfont{\gls{millimeter}}
}


\newglossaryentry{usablelengthmax}
{
  type=mtc,
  category=model,
  name={UsableLengthMax},
  code=\cfont{LUX},
  description={Maximum length of a Cutting Tool that can be used in a particular cutting operation including the non-cutting portions of the tool.},
  units=\cfont{\gls{millimeter}}
}


\newglossaryentry{protrudinglength}
{
  type=mtc,
  category=model,
  name={ProtrudingLength},
  code=\cfont{LPR},
  description={The dimension from the yz-plane to the furthest point of the Tool Item or Adaptive Item measured in the -X direction. },
  units=\cfont{\gls{millimeter}}
}


\newglossaryentry{weight}
{
  type=mtc,
  category=model,
  name={Weight},
  code=\cfont{WT},
  description={The total weight of the Cutting Tool in grams. The force exerted by the mass of the Cutting Tool. },
  units=\cfont{GRAM}
}


\newglossaryentry{functionallength}
{
  type=mtc,
  category=model,
  name={FunctionalLength},
  code=\cfont{LF},
  description={The distance from the gauge plane or from the end of the shank to the furthest point on the tool, if a gauge plane does not exist, to the cutting reference point determined by the main function of the tool. The \gls{cuttingtool} functional length will be the length of the entire tool, not a single Cutting Item. Each \gls{cuttingitem} can have an independent \gls{functionallength} represented in its measurements. },
  units=\cfont{\gls{millimeter}}
}


\newglossaryentry{count cuttingitems}
{
  type=mtc,
  category=model,
  name={count},
  description={The number of Cutting Items. }
}


\newglossaryentry{indices cuttingitem}
{
  type=mtc,
  category=model,
  name={indices},
  description={The number or numbers representing the individual Cutting Item or items on the tool. }
}


\newglossaryentry{itemid cuttingitem}
{
  type=mtc,
  category=model,
  name={itemId},
  description={The manufacturer identifier of this Cutting Item. }
}


\newglossaryentry{manufacturers cuttingitem}
{
  type=mtc,
  category=model,
  name={manufacturers},
  description={The manufacturers of the Cutting Item. }
}


\newglossaryentry{grade cuttingitem}
{
  type=mtc,
  category=model,
  name={grade},
  description={The material composition for this Cutting Item.}
}



\newglossaryentry{locus cuttingitem}
{
  type=mtc,
  category=model,
  name={Locus},
  description={A free form description of the location on the Cutting Tool.}
}


\newglossaryentry{itemlife cuttingitem}
{
  type=mtc,
  category=model,
  name={ItemLife},
  description={The life of this Cutting Item.}
}


\newglossaryentry{cuttingreferencepoint}
{
  type=mtc,
  category=model,
  name={CuttingReferencePoint},
  code=\cfont{CRP},
  description={The theoretical sharp point of the Cutting Tool from which the major functional dimensions are taken. },
  units=\cfont{\gls{millimeter}}
}


\newglossaryentry{cuttingedgelength}
{
  type=mtc,
  category=model,
  name={CuttingEdgeLength},
  code=\cfont{L},
  description={The theoretical length of the cutting edge of a Cutting Item over sharp corners.},
  units=\cfont{\gls{millimeter}}
}


\newglossaryentry{driveangle}
{
  type=mtc,
  category=model,
  name={DriveAngle},
  code=\cfont{DRVA},
  description={Angle between the driving mechanism locator on a Tool Item and the main cutting edge. },
  units=\cfont{\gls{degree}}
}


\newglossaryentry{flangediameter}
{
  type=mtc,
  category=model,
  name={FlangeDiameter},
  code=\cfont{DF},
  description={The dimension between two parallel tangents on the outside edge of a flange. },
  units=\cfont{\gls{millimeter}}
}


\newglossaryentry{functionalwidth}
{
  type=mtc,
  category=model,
  name={FunctionalWidth},
  code=\cfont{WF},
  description={The distance between the cutting reference point and the rear backing surface of a turning tool or the axis of a boring bar.},
  units=\cfont{\gls{millimeter}}
}


\newglossaryentry{incribedcirclediameter}
{
  type=mtc,
  category=model,
  name={IncribedCircleDiameter},
  code=\cfont{IC},
  description={The diameter of a circle to which all edges of a equilateral and round regular insert are tangential. },
  units=\cfont{\gls{millimeter}}
}


\newglossaryentry{pointangle}
{
  type=mtc,
  category=model,
  name={PointAngle},
  code=\cfont{SIG},
  description={The angle between the major cutting edge and the same cutting edge rotated by 180 degrees about the tool axis.},
  units=\cfont{\gls{degree}}
}


\newglossaryentry{toolcuttingedgeangle}
{
  type=mtc,
  category=model,
  name={ToolCuttingEdgeAngle},
  code=\cfont{KAPR},
  description={The angle between the tool cutting edge plane and the tool feed plane measured in a plane parallel the xy-plane. },
  units=\cfont{\gls{degree}}
}


\newglossaryentry{toolleadangle}
{
  type=mtc,
  category=model,
  name={ToolLeadAngle},
  code=\cfont{PSIR},
  description={The angle between the tool cutting edge plane and a plane perpendicular to the tool feed plane measured in a plane parallel the xy-plane. },
  units=\cfont{\gls{degree}}
}


\newglossaryentry{toolorientation}
{
  type=mtc,
  category=model,
  name={ToolOrientation},
  code=\cfont{N/A},
  description={The angle of the tool with respect to the workpiece for a given process. The value is application specific. },
  units=\cfont{\gls{degree}}
}


\newglossaryentry{wiperedgelength}
{
  type=mtc,
  category=model,
  name={WiperEdgeLength},
  code=\cfont{BS},
  description={The measure of the length of a wiper edge of a Cutting Item.},
  units=\cfont{\gls{millimeter}}
}


\newglossaryentry{stepdiameterlength}
{
  type=mtc,
  category=model,
  name={StepDiameterLength},
  code=\cfont{SDLx},
  description={The length of a portion of a stepped tool that is related to a corresponding cutting diameter measured from the cutting reference point of that cutting diameter to the point on the next cutting edge at which the diameter starts to change.},
  units=\cfont{\gls{millimeter}}
}


\newglossaryentry{stepincludedangle}
{
  type=mtc,
  category=model,
  name={StepIncludedAngle},
  code=\cfont{STAx},
  description={The angle between a major edge on a step of a stepped tool and the same cutting edge rotated 180 degrees about its tool axis.},
  units=\cfont{\gls{degree}}
}


\newglossaryentry{cuttingdiameter}
{
  type=mtc,
  category=model,
  name={CuttingDiameter},
  code=\cfont{DCx},
  description={The diameter of a circle on which the defined point Pk located on this Cutting Tool. The normal of the machined peripheral surface points towards the axis of the Cutting Tool.},
  units=\cfont{\gls{millimeter}}
}


\newglossaryentry{cuttingheight}
{
  type=mtc,
  category=model,
  name={CuttingHeight},
  code=\cfont{HF},
  description={The distance from the basal plane of the Tool Item to the cutting point. },
  units=\cfont{\gls{millimeter}}
}


\newglossaryentry{cornerradius}
{
  type=mtc,
  category=model,
  name={CornerRadius},
  code=\cfont{RE},
  description={The nominal radius of a rounded corner measured in the X Y-plane. },
  units=\cfont{\gls{millimeter}}
}


\newglossaryentry{functionallength cuttingitem}
{
  type=mtc,
  category=model,
  name={FunctionalLength},
  code=\cfont{LFx},
  description={The distance from the gauge plane or from the end of the shank of the Cutting Tool, if a gauge plane does not exist, to the cutting reference point determined by the main function of the tool. This measurement will be with reference to the Cutting Tool and \MUSTNOT exist without a Cutting Tool.},
  units=\cfont{\gls{millimeter}}
}


\newglossaryentry{chamferflatlength}
{
  type=mtc,
  category=model,
  name={ChamferFlatLength},
  code=\cfont{BCH},
  description={The flat length of a chamfer. },
  units=\cfont{\gls{millimeter}}
}


\newglossaryentry{chamferwidth}
{
  type=mtc,
  category=model,
  name={ChamferWidth},
  code=\cfont{CHW},
  description={The width of the chamfer. },
  units=\cfont{\gls{millimeter}}
}


\newglossaryentry{insertwidth}
{
  type=mtc,
  category=model,
  name={InsertWidth},
  code=\cfont{W1},
  description={W1 is used for the insert width when an inscribed circle diameter is not practical.},
  units=\cfont{\gls{millimeter}}
}


\newglossaryentry{publish}
{
  type=mtc,
  name={Publish},
  description={Sending of messages in a \gls{publishsubscribe} pattern.}
}


\newglossaryentry{subscribe}
{
  type=mtc,
  name={Subscribe},
  description={Receiving messages in a \gls{publishsubscribe} pattern.}
}



\newglossaryentry{fail value}
{
  type=mtc,
  category=model,
  name={FAIL},
  description={Failure before completion of an action.},
  kind={enum}
}


\newglossaryentry{complete value}
{
  type=mtc,
  category=model,
  name={COMPLETE},
  description={Completion of an action.},
  kind={enum}
}


\newglossaryentry{doorinterface}
{
  type=mtc,
  category=model,
  name={DoorInterface},
  description={\gls{doorinterface} provides the set of information used to coordinate the operations between two pieces of equipment, one of which controls the operation of a door. }
}


\newglossaryentry{chuckinterface}
{
  type=mtc,
  category=model,
  name={ChuckInterface},
  description={\gls{chuckinterface} provides the set of information used to coordinate the operations between two pieces of equipment, one of which controls the operation of a chuck.  }
}


\newglossaryentry{barfeederinterface}
{
  type=mtc,
  category=model,
  name={BarFeederInterface},
  description={\gls{barfeederinterface} provides the set of information used to coordinate the operations between a Bar Feeder and another piece of equipment.  }
}


\newglossaryentry{materialhandlerinterface}
{
  type=mtc,
  category=model,
  name={MaterialHandlerInterface},
  description={\gls{materialhandlerinterface} provides the set of information used to coordinate the operations between a piece of equipment and another associated piece of equipment used to automatically handle various types of materials or services associated with the original piece of equipment. }
}


\newglossaryentry{request subtype interface}
{
  type=mtc,
  category=model,
  name={REQUEST},
  description={A subtype of an \gls{interface} \gls{dataitem} type to communicate a request. }
}


\newglossaryentry{response subtype interface}
{
  type=mtc,
  category=model,
  name={RESPONSE},
  description={A subtype of an \gls{interface} \gls{dataitem} type to communicate a response.}
}


\newglossaryentry{opendoor event}
{
  type=mtc,
  category=model,
  name={OPEN\_DOOR},
  elementname=\cfont{OpenDoor},
  description={Service to open a door. },
  kind={type,event},
  facet={\gls{string}}
}


\newglossaryentry{closedoor event}
{
  type=mtc,
  category=model,
  name={CLOSE\_DOOR},
  elementname=\cfont{CloseDoor},
  description={Service to close a door.},
  kind={type,event},
  facet={\gls{string}}
}


\newglossaryentry{openchuck event}
{
  type=mtc,
  category=model,
  name={OPEN\_CHUCK},
  elementname=\cfont{OpenChuck},
  description={Service to open a chuck. },
  kind={type,event},
  facet={\gls{string}}
}


\newglossaryentry{closechuck event}
{
  type=mtc,
  category=model,
  name={CLOSE\_CHUCK},
  elementname=\cfont{CloseChuck},
  description={Service to close a chuck.},
  kind={type,event},
  facet={\gls{string}}
}


\newglossaryentry{materialfeed event}
{
  type=mtc,
  category=model,
  name={MATERIAL\_FEED},
  elementname=\cfont{MaterialFeed},
  description={Service to advance material or feed product to a piece of equipment from a continuous or bulk source. },
  kind={type,event},
  facet={\gls{string}}
}


\newglossaryentry{materialchange event}
{
  type=mtc,
  category=model,
  name={MATERIAL\_CHANGE},
  elementname=\cfont{MaterialChange},
  description={Service to change the type of material or product being loaded or fed to a piece of equipment.},
  kind={type,event},
  facet={\gls{string}}
}


\newglossaryentry{materialretract event}
{
  type=mtc,
  category=model,
  name={MATERIAL\_RETRACT},
  elementname=\cfont{MaterialRetract},
  description={Service to remove or retract material or product.},
  kind={type,event},
  facet={\gls{string}}
}


\newglossaryentry{partchange event}
{
  type=mtc,
  category=model,
  name={PART\_CHANGE},
  elementname=\cfont{PartChange},
  description={Service to change the part or product associated with a piece of equipment to a different part or product.  },
  kind={type,event},
  facet={\gls{string}}
}


\newglossaryentry{materialload event}
{
  type=mtc,
  category=model,
  name={MATERIAL\_LOAD},
  elementname=\cfont{MaterialLoad},
  description={Service to load a piece of material or product.},
  kind={type,event},
  facet={\gls{string}}
}


\newglossaryentry{materialunload event}
{
  type=mtc,
  category=model,
  name={MATERIAL\_UNLOAD},
  elementname=\cfont{MaterialUnload},
  description={Service to unload a piece of material or product.},
  kind={type,event},
  facet={\gls{string}}
}


\newglossaryentry{integer}
{
  type=mtc,
  name={integer},
  description={Integer data type.},
  kind={facet}
}


\newglossaryentry{string}
{
  type=mtc,
  name={string},
  description={String data type.},
  kind={facet}
}


\newglossaryentry{float}
{
  type=mtc,
  name={float},
  description={Float data type.},
  kind={facet}
}


\newglossaryentry{boolean}
{
  type=mtc,
  name={boolean},
  description={Boolean data type.},
  kind={facet}
}


\newglossaryentry{datetime}
{
  type=mtc,
  name={datetime},
  description={DateTime (yyyy-mm-ddthh:mm:ss.ffff) data type.},
  kind={facet}
}


\newglossaryentry{array3d}
{
  type=mtc,
  name={array3d},
  description={A three dimensional \gls{float} array.},
  kind={facet},
  facet={\gls{float}}
}

\newglossaryentry{arraystring}
{
  type=mtc,
  name={arraystring},
  description={An array of \glspl{string}.},
  kind={facet},
  facet={\gls{string}}
}

\longnewglossaryentry{abstract element}
{
  name= {Abstract Element}
}
{
  An element that defines a set of common characteristics that are shared by a group of elements.
  
  An abstract element cannot appear in a document. In a specific implementation of a schema, an abstract element is replaced by a derived element that is itself not an abstract element. The characteristics for the derived element are inherited from the abstract element. 
  
  Appears in the documents in the following form: abstract.
}


\longnewglossaryentry{adapter}
{
  name= {Adapter}
}
{
  An optional piece of hardware or software that transforms information provided by a piece of equipment into a form that can be received by an MTConnect Agent.

  Appears in the documents in the following form: adapter.
}


\longnewglossaryentry{agent}
{
  name= {Agent}
}
{
  Refers to an MTConnect Agent. 
  
  Software that collects data published from one or more piece(s) of equipment, organizes that data in a structured manner, and responds to requests for data from client software systems by providing a structured response in the form of a Response Document that is constructed using the semantic data models defined in the Standard. 
  Appears in the documents in the following form: MTConnect Agent or Agent.
}


\longnewglossaryentry{application programming interface}
{
  name= {Application Programming Interface}
}
{
  A set of methods to provide communications between software applications.

  The API defined in the MTConnect Standard describes the methods for providing the Request/Response Information Exchange between an MTConnect Agent and client software applications.
  
  Appears in the documents in the following forms: Application Programming Interface or API.
}


\longnewglossaryentry{archetype}
{
  name={Archetype}
}
{
  \ulheading{General Description of an MTConnect Asset}:
  
  Archetype is a class of MTConnect Assets that provides the requirements, constraints, and common properties for a type of MTConnect Asset.

  Appears in the documents in the following form: Archetype.

  \ulheading{Used as an XML term describing an MTConnect Asset}:
  
  In an XML representation of the Assets Information Model, Archetype is an abstract element that is replaced by a specific type of Asset \gls{archetype mtconnectassets}. 
  
  Appears in the documents in the following form: \gls{archetype mtconnectassets}
}


\newglossaryentry{archetype mtconnectassets}
{
  type=mtc,
  category=model,
  name={Archetype},
  description={Model of a domain concept.}
}

\longnewglossaryentry{asset}
{
  name= {Asset},
  plural= {Assets}
}
{
  \ulheading{General meaning}:

  Typically referred to as an MTConnect Asset.

  An MTConnect Asset is something that is used in the manufacturing process, but is not permanently associated with a single piece of equipment, can be removed from the piece of equipment without compromising its function, and can be associated with other pieces of equipment during its lifecycle.

  \ulheading{Used to identify a storage area in an MTConnect Agent}:

  See description of Buffer.

  \ulheading{Used as an Information Model}:

  Used to describe an Information Model that contains the rules and terminology that describe information that may be included in electronic documents representing MTConnect Assets.

  The Assets Information Model defines the structure for the Assets Response Document.

  Individual Information Models describe the structure of the Asset Documents represent each type of MTConnect Asset. Appears in the documents in the following form: Assets Information Model or (asset type) Information Model.
  
  \ulheading{Used when referring to an MTConnect Asset}:

  Refers to the information related to an MTConnect Asset or a group of MTConnect Assets.

  Appears in the documents in the following form: \gls{asset} or \glspl{asset}.

  \ulheading{Used as an XML container or element}:

  \begin{itemize}
      \item When used as an XML container that consists of one or more types of Asset XML elements.
      
      Appears in the documents in the following form: Assets.
      \item When used as an abstract XML element. It is replaced in the XML document by types of Asset elements representing individual Asset entities.
      
      Appears in the documents in the following form: \gls{asset mtconnectassets}.
  \end{itemize}

  \ulheading{Used to describe information stored in an MTConnect Agent}:

  Identifies an electronic document published by a data source and stored in the assets buffer of an MTConnect Agent.
  
  Appears in the documents in the following form: Asset Document.
  
  \ulheading{Used as an XML representation of an MTConnect Response document}:

  Identifies an electronic document encoded in XML and published by an MTConnect Agent in response to a Request for information from a client software application relating to MTConnect Assets.

  Appears in the documents in the following form: \glspl{mtconnectasset}.

  \ulheading{Used as an MTConnect Request}:

  Represents a specific type of communications request between a client software application and an MTConnect Agent regarding MTConnect Assets.

  Appears in the documents in the following form: Asset Request.

  \ulheading{Used as part of an HTTP Request}:

  Used in the path portion of an HTTP Request Line, by a client software application, to initiate an Asset Request to an MTConnect Agent to publish an MTConnectAssets document.

  Appears in the documents in the following form: asset.
}


\longnewglossaryentry{attribute}
{
  name={Attribute},
  plural={Attributes}
}
{
  A term that is used to provide additional information or properties for an element.

  Appears in the documents in the following form: attribute.
}


\longnewglossaryentry{base functional structure}
{
  name={Base Functional Structure}
}
{
  A consistent set of functionalities defined by the MTConnect Standard. This functionality includes the protocol(s) used to communicate data to a client software application, the semantic data models defining how that data is organized into Response Documents, and the encoding of those Response Documents.

  Appears in the documents in the following form: Base Functional Structure.
}

\longnewglossaryentry{buffer}
{
  name={buffer}
}
{
  \ulheading{General meaning}:

  A section of an MTConnect Agent that provides storage for information published from pieces of equipment.

  \ulheading{Used relative to Streaming Data}:

  A section of an MTConnect Agent that provides storage for information relating to individual pieces of Streaming Data. 
  
  Appears in the documents in the following form: buffer.

  \ulheading{Used relative to MTConnect Assets}:

  A section of an MTConnect Agent that provides storage for Asset Documents.

  Appears in the documents in the following form: assets buffer.
}

\longnewglossaryentry{cdata}
{
  name={CDATA}
}
{
  \ulheading{General meaning}:

  An abbreviation for Character Data.

  CDATA is used to describe a value (text or data) published as part of an XML element.

  For example, "This is some text" is the CDATA in the XML element:
  
  <Message ...>This is some text</Message>

  Appears in the documents in the following form: CDATA
}


\longnewglossaryentry{child element}
{
  name={Child Element},
  plural={Child Elements}
}
{
  A portion of a data modeling structure that illustrates the relationship between an element and the higher-level Parent Element within which it is contained.
  
  Appears in the documents in the following form: Child Element.
}


\longnewglossaryentry{client}
{
  name={client}
}
{
  A process or set of processes that send Requests for information to an MTConnect Agent; e.g. software applications or a function that implements the Request portion of an Interface Interaction Model.

  Appears in the documents in the following form: client.
}

\longnewglossaryentry{component term}
{
  name={Component}
}
{
  \ulheading{General meaning}:

  A Structural Element that represents a physical or logical part or subpart of a piece of equipment.

  Appears in the documents in the following form: Component.

  \ulheading{Used in Information Models}:

  A data modeling element used to organize the data being retrieved from a piece of equipment.
  \begin{itemize}
      \item When used as an XML container to organize Lower Level Component elements. 
      
      Appears in the documents in the following form: Components.
      \item When used as an abstract XML element. Component is replaced in a data model by a type of Component element. Component is also an XML container used to organize Lower Level Component elements, Data Entities, or both.
      
      Appears in the documents in the following form: Component.
  \end{itemize}
}

\longnewglossaryentry{composition term}
{
  name={Composition}
}
{
  \ulheading{General meaning}:

  Data modeling elements that describe the lowest level basic structural or functional building blocks contained within a Component element.

  Appears in the documents in the following form: Composition Element

  \ulheading{Used in Information Models}:

  A data modeling element used to organize the data being retrieved from a piece of equipment.
  \begin{itemize}
      \item When used as an XML container to organize Composition elements. 
      
      Appears in the documents in the following form: Compositions
      \item When used as an abstract XML element. Composition is replaced in a data model by a type of Composition Element. 
      
      Appears in the documents in the following form: Composition.
  \end{itemize}
}


\longnewglossaryentry{condition term}
{
  name={Condition}
}
{
	\ulheading{General meaning}:

	An indicator of the health of a piece of equipment or a Component and its ability to function.

	\ulheading{Used as a modeling element}:

	A data modeling element used to organize and communicate information relative to the health of a piece of equipment or Component.

	Appears in the documents in the following form: Condition or as Condition Element(s).

	\ulheading{Used in Information Models}:

	An XML element used to represent Condition Elements.

    \begin{itemize}
	\item When used as an XML container to organize Lower Level Condition elements.

	Appears in the documents in the following form: Condition.

	\item When used as a Lower Level element, the form Condition is an abstract type XML element.  This Lower Level element is a Data Entity.  Condition is replaced in a data model by type of Condition element.

	Appears in the documents in the following form: Condition.
	\end{itemize}

	Note: The form Condition is used to represent both above uses.
}


\longnewglossaryentry{controlled vocabulary}
{
  name={Controlled Vocabulary},
  plural={Controlled Vocabularies}
}
{
	A restricted set of values that may be published as the Valid Data Value for a Data Entity.

	Appears in the documents in the following form: Controlled Vocabulary.
}


\longnewglossaryentry{current term}
{
  name={Current}
}
{
	\ulheading{General meaning}:

	Meaning 1:  A term describing the most recent occurrence of something.

	Meaning 2:  A term used to describe movement; e.g. electric current or air current.

	Appears in the documents in the following form: current

	\ulheading{Used in reference to an MTConnect Agent}:

	A reference to the most recent information available to an MTConnect Agent.

	Appears in the documents in the following form: current.

	\ulheading{Used as an MTConnect Request}:

	A specific type of communications request between a client software application and an MTConnect Agent regarding Streaming Data.  

	Appears in the documents in the following form: Current Request.

	\ulheading{Used as part of an HTTP Request}:

	Used in the path portion of an HTTP Request Line, by a client software application, to initiate a Current Request to an MTConnect Agent to publish an MTConnectStreams document.

	Appears in the documents in the following form: current.
}


\longnewglossaryentry{data dictionary}
{
  name={Data Dictionary}
}
{
	Listing of standardized terms and definitions used in MTConnect Information Models.

	Appears in the documents in the following form: data dictionary.
}


\longnewglossaryentry{data entity}
{
  name={Data Entity},
  plural={Data Entities}
}
{
	A primary data modeling element that represents all elements that either describe data items that may be reported by an MTConnect Agent or the data items that contain the actual data published by an Agent.

	Appears in the documents in the following form: Data Entity.
}


\longnewglossaryentry{data item}
{
  name={Data Item}
}
{
	\ulheading{General meaning}:

	Descriptive information or properties and characteristics associated with a Data Entity.

	Appears in the documents in the following form: data item.

	\ulheading{Used in an XML representation of a Data Entity}:

    \begin{itemize}
	\item When used as an XML container to organize DataItem elements.

	Appears in the documents in the following form: DataItems.

	\item When used to represent a specific Data Entity, the form DataItem is an XML element.  

	Appears in the documents in the following form: DataItem.
    \end{itemize}
}


\longnewglossaryentry{data source}
{
  name={Data Source}
}
{
	Any piece of equipment that can produce data that is published to an MTConnect Agent.

	Appears in the documents in the following form: data source.
}


\longnewglossaryentry{data streaming}
{
  name={Data Streaming}
}
{
	A method for an MTConnect Agent to provide a continuous stream of information in response to a single Request from a client software application.

	Appears in the documents in the following form: Data Streaming.
}


\longnewglossaryentry{deprecated}
{
  name={Deprecated}
}
{
	An indication that specific content in an MTConnect Document is currently usable but is regarded as being obsolete or superseded.  It is recommended that deprecated content should be avoided.

	Appears in the documents in the following form: DEPRECATED.
}


\longnewglossaryentry{deprecation warning}
{
  name={Deprecation Warning}
}
{
	An indicator that specific content in an MTConnect Document may be changed to DEPRECATED in a future release of the standard.

	Appears in the documents in the following form: DEPRECATION WARNING.
}


\longnewglossaryentry{device information model}
{
  name={Device Information Model},
  plural={Devices Information Model}
}
{
	A set of rules and terms that describes the physical and logical configuration for a piece of equipment and the data that may be reported by that equipment.    

	Appears in the documents in the following form: Devices Information Model.
}


\longnewglossaryentry{device term}
{
  name={Device}
}
{
	A part of an information model representing a piece of equipment.  
	\ulheading{Used in an XML representation of a Response Document}:

    \begin{itemize}
	\item When used as an XML container to organize Device elements.

	Appears in the documents in the following form: Devices.

	\item When used as an XML container to represent a specific piece of equipment and is composed of a set of Structural Elements that organize and provide relevance to data published from that piece of equipment.

	Appears in the documents in the following form: Device.
	\end{itemize}
}


\longnewglossaryentry{document}
{
  name={Document}
}
{
	\ulheading{General meaning}:

	A piece of written, printed, or electronic matter that provides information.

	\ulheading{Used to represent an MTConnect Document}:

	Refers to printed or electronic document(s) that represent a Part(s) of the MTConnect Standard.

	Appears in the documents in the following form: MTConnect Document.

	\ulheading{Used to represent a specific representation of an MTConnect Document}:

	Refers to electronic document(s) associated with an MTConnect Agent that are encoded using XML; Response Documents or Asset Documents.

	Appears in the documents in the following form: MTConnect XML Document.

	\ulheading{Used to describe types of information stored in an MTConnect Agent}:

	In an implementation, the electronic documents that are published from a data source and stored by an MTConnect Agent.

	Appears in the documents in the following form: Asset Document.

	\ulheading{Used to describe information published by an MTConnect Agent}:

	A document published by an MTConnect Agent based upon one of the semantic data models defined in the MTConnect Standard in response to a request from a client.  

	Appears in the documents in the following form: Response Document.
}


\longnewglossaryentry{document body}
{
  name={Document Body}
}
{
	The portion of the content of an MTConnect Response Document that is defined by the relative MTConnect Information Model. The Document Body contains the Structural Elements and Data Entities reported in a Response Document.

	Appears in the documents in the following form: Document Body.
}


\longnewglossaryentry{document header}
{
  name={Document Header}
}
{
	The portion of the content of an MTConnect Response Document that provides information from an MTConnect Agent defining version information, storage capacity, protocol, and other information associated with the management of the data stored in or retrieved from the Agent.
	
	Appears in the documents in the following form: Document Header.
}


\longnewglossaryentry{element}
{
  name={Element},
  plural={Elements}
}
{
	Refers to an XML element.

	An XML element is a logical portion of an XML document or schema that begins with a start-tag and ends with a corresponding end-tag.  

	The information provided between the start-tag and end-tag may contain attributes, other elements (sub-elements), and/or CDATA.

	Note:  Also, an XML element may consist of an empty-element tag.  Refer to Appendix B for more information on element tags.

	Appears in the documents in the following form: element.
}


\longnewglossaryentry{element name}
{
  name={Element Name},
  plural={Element Names}
}
{
	A descriptive identifier contained in both the start-tag and end-tag of an XML element that provides the name of the element.

	Appears in the documents in the following form: element name.

	\ulheading{Used to describe the name for a specific XML element}:

	Reference to the name provided in the start-tag, end-tag, or empty-element tag for an XML element.

	Appears in the documents in the following form: Element Name.
}


\longnewglossaryentry{equipment}
{
  name={Equipment}
}
{
	Represents anything that can publish information and is used in the operations of a manufacturing facility shop floor.  Examples of equipment are machine tools, ovens, sensor units, workstations, software applications, and bar feeders.

	Appears in the documents in the following form: equipment or piece of equipment.
}


\longnewglossaryentry{error information model}
{
  name={Error Information Model}
}
{
	The rules and terminology that describes the Response Document returned by an MTConnect Agent when it encounters an error while interpreting a Request for information from a client software application or when an Agent experiences an error while publishing the Response to a Request for information.

	Appears in the documents in the following form: Error Information Model.
}


\longnewglossaryentry{event term}
{
  name={Event},
  plural={Events}
}
{
	\ulheading{General meaning}:

	The occurrence of something that happens or takes place.

	Appears in the documents in the following form: event.

	\ulheading{Used as a type of Data Entity}:

	An identification that represents a change in state of information associated with a piece of equipment or an occurrence of an action.  Event also provides a means to publish a message from a piece of equipment.

	Appears in the documents in the following form: Event.

	\ulheading{Used as a category attribute for a Data Entity}:

	Used as a value for the category attribute for an XML dataItem element.

	Appears in the documents in the following form: EVENT.

	\ulheading{Used as an XML container or element}:

    \begin{itemize}
	\item When used as an XML container that consists of one or more types of Event XML elements.

	Appears in the documents in the following form: Events.

	\item When used as an abstract XML element.  It is replaced in the XML document by types of Event elements.

	Appears in the documents in the following form: Event.
    \end{itemize}
}


\longnewglossaryentry{extensible}
{
  name={Extensible}
}
{
	The ability for an implementer to extend MTConnect Information Models by adding content not currently addressed in the MTConnect Standard.
}


\longnewglossaryentry{fault state}
{
  name={Fault State},
  plural={Fault States}
}
{
	In the MTConnect Standard, a term that indicates the reported status of a Condition category Data Entity.   

	Appears in the documents in the following form: Fault State.
}


\longnewglossaryentry{heartbeat}
{
  name={Heartbeat}
}
{
	\ulheading{General meaning}:

	A function that indicates to a client application that the communications connection to an MTConnect Agent is still viable during times when there is no new data available to report – often referred to as a "keep alive" message.

	Appears in the documents in the following form: heartbeat.

	\ulheading{When used as part of an HTTP Request}:

	The form heartbeat is used as a parameter in the query portion of an HTTP Request Line.

	Appears in the documents in the following form: heartbeat.
}


\longnewglossaryentry{http}
{
  name={HTTP}
}
{
	Hyper-Text Transport Protocol.  The protocol used by all web browsers and web applications.

	Note:  HTTP is an IETF standard and is defined in RFC 7230.  See https://tools.ietf.org/html/rfc7230 for more information.
}


\longnewglossaryentry{http error message}
{
  name={HTTP Error Message}
}
{
	In the MTConnect Standard, a response provided by an MTConnect Agent indicating that an HTTP Request is incorrectly formatted or identifies that the requested data is not available from the Agent.  

	Appears in the documents in the following form: HTTP Error Message.
}


\longnewglossaryentry{http header}
{
  name={HTTP Header}
}
{
	In the MTConnect Standard, the content of the Header portion of either an HTTP Request from a client software application or an HTTP Response from an MTConnect Agent.

	Appears in the documents in the following form: HTTP Header.
}


\longnewglossaryentry{http method}
{
  name={HTTP Method}
}
{
	In the MTConnect Standard, a portion of a command in an HTTP Request that indicates the desired action to be performed on the identified resource; often referred to as verbs.
}


\longnewglossaryentry{http request}
{
  name={HTTP Request}
}
{
	In the MTConnect Standard, a communications command issued by a client software application to an MTConnect Agent requesting information defined in the HTTP Request Line.

	Appears in the documents in the following form: HTTP Request.
}


\longnewglossaryentry{http request line}
{
  name={HTTP Request Line}
}
{
	In the MTConnect Standard, the first line of an HTTP Request describing a specific Response Document to be published by an MTConnect Agent.

	Appears in the documents in the following form: HTTP Request Line.
}


\longnewglossaryentry{http response}
{
  name={HTTP Response}
}
{
	In the MTConnect Standard, the information published from an MTConnect Agent in reply to an HTTP Request.  An HTTP Response may be either a Response Document or an HTTP Error Message.

	Appears in the documents in the following form: HTTP Response.
}


\longnewglossaryentry{http server}
{
  name={HTTP Server}
}
{
	In the MTConnect Standard, a software program that accepts HTTP Requests from client software applications and publishes HTTP Responses as a reply to those Requests.

	Appears in the documents in the following form: HTTP Server.
}


\longnewglossaryentry{http status code}
{
  name={HTTP Status Code}
}
{
	In the MTConnect Standard, a numeric code contained in an HTTP Response that defines a status category associated with the Response – either a success status or a category of an HTTP error.  

	Appears in the documents in the following form: HTTP Status Code.
}


\longnewglossaryentry{id term}
{
  name={id}
}
{
	\ulheading{General meaning}:

	An identifier used to distinguish a piece of information.

	Appears in the documents in the following form: id.

	\ulheading{Used as an XML attribute}:

	When used as an attribute for an XML element - Structural Element, Data Entity, or Asset.  id provides a unique identity for the element within an XML document.

	Appears in the documents in the following form: id.
}


\longnewglossaryentry{implementation}
{
  name={Implementation}
}
{
	A specific instantiation of the MTConnect Standard.
}


\longnewglossaryentry{information model}
{
  name={Information Model},
  plural={Information Models}
}
{
	The rules, relationships, and terminology that are used to define how information is structured.

	For example, an information model is used to define the structure for each MTConnect Response Document; the definition of each piece of information within those documents and the relationship between pieces of information.

	Appears in the documents in the following form: Information Model.
}


\longnewglossaryentry{instance}
{
  name={Instance},
  plural={Instances}
}
{
	Describes a set of Streaming Data in an MTConnect Agent.  Each time an Agent is restarted with an empty buffer, data placed in the buffer represents a new instance of the Agent.

	Appears in the documents in the following form: instance.
}


\longnewglossaryentry{interaction model}
{
  name={Interaction Model}
}
{
	The definition of information exchanged to support the interactions between pieces of equipment collaborating to complete a task.

	Appears in the documents in the following form: Interaction Model.
}


\longnewglossaryentry{interface}
{
  name={Interface},
  plural={Interfaces}
}
{
	\ulheading{General meaning}:

	The exchange of information between pieces of equipment and/or software systems.

	Appears in the documents in the following form: interface.

	\ulheading{Used as an Interaction Model}:

	An Interaction Model that describes a method for inter-operations between pieces of equipment.

	Appears in the documents in the following form: Interface.

	\ulheading{Used as an XML container or element}:

    \begin{itemize}
	\item When used as an XML container that consists of one or more types of Interface XML elements.

	Appears in the documents in the following form: Interfaces.

	\item When used as an abstract XML element.  It is replaced in the XML document by types of Interface elements.

	Appears in the documents in the following form: Interface.
	\end{itemize}
}


\longnewglossaryentry{message term}
{
  name={Message}
}
{
	\ulheading{General meaning}:

	The content of a communication process.

	Appears in the documents in the following form: message.

	\ulheading{Used relative to an MTConnect Agent}:

	Describes the information that is exchanged between an MTConnect Agent and a client software application.  A Message may contain either a Request from a client software application or a Response from an MTConnect Agent.

	Appears in the documents in the following form: Message.

	\ulheading{Used as a type of Data Entity}:

	Describes a type of Data Entity in the Devices Information Model that can contain any text string of information or native code to be transferred from a piece of equipment.

	Appears in the documents in the following form: MESSAGE.

	\ulheading{Used as an Element Name}:

	An Element Name for a Data Entity in the Streams Information Model that can contain any text string of information or native code to be transferred from a piece of equipment.

	Appears in the documents in the following form:  Message.
}


\longnewglossaryentry{metadata}
{
  name={Metadata}
}
{
	Data that provides information about other data.

	For example, Equipment Metadata defines both the Structural Elements that represent the physical and logical parts and sub-parts of each piece of equipment, the relationships between those parts and sub-parts, and the definitions of the Data Entities associated with that piece of equipment.

	Appears in the documents in the following form: Metadata or Equipment Metadata.
}


\longnewglossaryentry{mtconnect agent}
{
  name={MTConnect Agent}
}
{
	See definition for Agent.
}


\longnewglossaryentry{mtconnectassets response document}
{
  name={MTConnectAssets Response Document}
}
{
	An electronic document published by an MTConnect Agent in response to a Request for information from a client software application relating to MTConnect Assets.

	Appears in the documents in the following form: MTConnectAssets Response Document.
}


\longnewglossaryentry{mtconnectdevices response document}
{
  name={MTConnectDevices Response Document}
}
{
	An electronic document published by an MTConnect Agent in response to a Request for information from a client software application that includes metadata for one or more pieces of equipment.

	Appears in the documents in the following form: MTConnectDevices Response Document.
}


\longnewglossaryentry{mtconnecterrors response document}
{
  name={MTConnectErrors Response Document}
}
{
	An electronic document published by an MTConnect Agent whenever it encounters an error while interpreting a Request for information from a client software application or when an Agent experiences an error while publishing the Response to a Request for information.

	Appears in the documents in the following form: MTConnectErrors Response Document.
}


\longnewglossaryentry{mtconnect request}
{
  name={MTConnect Request}
}
{
	A communication request for information issued from a client software application to an MTConnect Agent.

	Appears in the documents in the following form: MTConnect Request.
}


\longnewglossaryentry{mtconnectstreams response document}
{
  name={MTConnectStreams Response Document}
}
{
	An electronic document published by an MTConnect Agent in response to a Request for information from a client software application that includes Streaming Data from the Agent.

	Appears in the documents in the following form: MTConnectStreams Response Document.
}


\longnewglossaryentry{nmtoken}
{
  name={NMTOKEN}
}
{
	The data type for XML identifiers.

	Note: The identifier must start with a letter, an underscore "\_" or a colon.  The next character must be a letter, a number, or one of the following ".", "-", "\_", ":".  The identifier must not have any spaces or special characters.

	Appears in the documents in the following form: NMTOKEN.
}


\longnewglossaryentry{parameter}
{
  name={Parameter}
}
{
	\ulheading{General Meaning}:

	A variable that must be given a value during the execution of a program or a communications command.

	\ulheading{When used as part of an HTTP Request}:

	Represents the content (keys and associated values) provided in the Query portion of an HTTP Request Line that identifies specific information to be returned in a Response Document.

	Appears in the documents in the following form: parameter.
}


\longnewglossaryentry{parent element}
{
  name={Parent Element}
}
{
	An XML element used to organize Lower Level child elements that share a common relationship to the Parent Element.

	Appears in the documents in the following form: Parent Element.
}


\longnewglossaryentry{persistence}
{
  name={Persistence}
}
{
	A method for retaining or restoring information.
}


\longnewglossaryentry{probe term}
{
  name={Probe}
}
{
	\ulheading{General meaning of a physical entity}:

	An instrument commonly used for measuring the physical geometrical characteristics of an object.

    \begin{itemize}
	\item \ulheading{Used to describe a measurement device}:

	The form probe is used to define a measurement device that provides position information.

	Appears in the documents in the following form: probe. 

	\item \ulheading{Used within a Data Entity}:

	The form PROBE is used to designate a subtype for the Data Entity PATH\_POSITION indicating a measurement position relating to a probe unit.

	Appears in the documents in the following form: PROBE.
    \end{itemize}

	\ulheading{General meaning for communications with an MTConnect Agent}:

	Probe is used to define a type of communication request. 

    \begin{itemize}
	\item \ulheading{Used as a type of communication request}:

	The form Probe Request represents a specific type of communications request between a client software application and an MTConnect Agent regarding metadata for one or more pieces of equipment.

	Appears in the documents in the following form: Probe Request.

	\item \ulheading{Used in an HTTP Request Line}:

	The form probe is used to designate a Probe Request in the <Path> portion of an HTTP Request Line.

	Appears in the documents in the following form: probe.
    \end{itemize}
}


\longnewglossaryentry{protocol}
{
  name={Protocol}
}
{
	A set of rules that allow two or more entities to transmit information from one to the other.
}


\longnewglossaryentry{publishsubscribe}
{
  name={Publish/Subscribe}
}
{
	In the MTConnect Standard, a communications messaging pattern that may be used to publish Streaming Data from an MTConnect Agent.  When a Publish/Subscribe communication method is established between a client software application and an MTConnect Agent, the Agent will repeatedly publish a specific MTConnectStreams document at a defined period.

	Appears in the documents in the following form: Publish/Subscribe.
}


\longnewglossaryentry{query}
{
  name={Query}
}
{
	\ulheading{General Meaning}:

	A portion of a request for information that more precisely defines the specific information to be published in response to the request. 

	Appears in the documents in the following form: Query.

	\ulheading{Used in an HTTP Request Line}:

	The form query includes a string of parameters that define filters used to refine the content of a Response Document published in response to an HTTP Request.

	Appears in the documents in the following form: query.
}


\longnewglossaryentry{requestresponse messaging structure}
{
  name={Request /Response Messaging Structure}
}
{
	A communications pattern that supports the transfer of information between an MTConnect Agent and a client software application. In a Request/Response information exchange, a client software application requests specific information from an MTConnect Agent. An MTConnect Agent responds to the Request by publishing a Response Document.   

	Appears in the documents in the following form: Request/Response Messaging Structure.
}


\longnewglossaryentry{request}
{
  name={Request}
}
{
	A communications method where a client software application transmits a message to an MTConnect Agent.  That message instructs the Agent to respond with specific information.

	Appears in the documents in the following form: Request.
}


\longnewglossaryentry{requester}
{
  name={Requester}
}
{
	An entity that initiates a Request for information in a communications exchange.

	Appears in the documents in the following form: Requester.
}


\longnewglossaryentry{responder}
{
  name={Responder}
}
{
	An entity that responds to a Request for information in a communications exchange.

	Appears in the documents in the following form: Responder.
}


\longnewglossaryentry{response document}
{
  name={Response Document}
}
{
	See definition of Document.
}


\longnewglossaryentry{rest}
{
  name={REST}
}
{
	Stands for REpresentational State Transfer:  A software architecture where a client software application and server move through a series of state transitions based solely on the request from the client and the response from the server. 

	Appears in the documents in the following form: REST.
}


\longnewglossaryentry{root element}
{
  name={Root Element}
}
{
	The first Structural Element provided in a Response Document encoded using XML.  The Root Element is an XML container and is the Parent Element for all other XML elements in the document.  The Root Element appears immediately following the XML Declaration.

	Appears in the documents in the following form: Root Element.
}


\longnewglossaryentry{sample term}
{
  name={Sample},
  plural={Samples}
}
{
	\ulheading{General meaning}:

	The collection of one or more pieces of information.  

	\ulheading{Used when referring to the collection of information}:

	When referring to the collection of a piece of information from a data source.

	Appears in the documents in the following form: sample.

	\ulheading{Used as an MTConnect Request}:

	When representing a specific type of communications request between a client software application and an MTConnect Agent regarding Streaming Data.  

	Appears in the documents in the following form: Sample Request.

	\ulheading{Used as part of an HTTP Request}:

	Used in the path portion of an HTTP Request Line, by a client software application, to initiate a Sample Request to an MTConnect Agent to publish an MTConnectStreams document.

	Appears in the documents in the following form: sample.

	\ulheading{Used to describe a Data Entity}:

	Used to define a specific type of Data Entity.  A Sample type Data Entity reports the value for a continuously variable or analog piece of information.

	Appears in the documents in the following form: Sample or Samples.

	\ulheading{Used as an XML container or element}:

    \begin{itemize}
	\item When used as an XML container that consists of one or more types of Sample XML elements.

	Appears in the documents in the following form: Samples.

	\item When used as an abstract XML element.  It is replaced in the XML document by types of Sample elements representing individual Sample type of Data Entity.

	Appears in the documents in the following form: Sample.
    \end{itemize}
}


\longnewglossaryentry{schema}
{
  name={Schema}
}
{
	\ulheading{General meaning}:

	The definition of the structure, rules, and vocabularies used to define the information published in an electronic document.

	Appears in the documents in the following form: schema.

	\ulheading{Used in association with an MTConnect Response Document}:

	Identifies a specific schema defined for an MTConnect Response Document.

	Appears in the documents in the following form: schema.
}


\longnewglossaryentry{semantic data model}
{
  name={Semantic Data Model}
}
{
	A methodology for defining the structure and meaning for data in a specific logical way.  

	It provides the rules for encoding electronic information such that it can be interpreted by a software system.  

	Appears in the documents in the following form: semantic data model.
}


\longnewglossaryentry{sequence number}
{
  name={sequence number},
  plural={sequence numbers}
}
{
	The primary key identifier used to manage and locate a specific piece of Streaming Data in an MTConnect Agent.

	Sequence number is a monotonically increasing number within an instance of an MTConnect Agent.

	Appears in the documents in the following form: sequence number.
}


\longnewglossaryentry{standard}
{
  name={Standard}
}
{
	\ulheading{General meaning}:

	A document established by consensus that provides rules, guidelines, or characteristics for activities or their results (as defined in ISO/IEC Guide 2:2004).

	Used when referring to the MTConnect Standard. 

	The MTConnect Standard is a standard that provides the definition and semantic data structure for information published by pieces of equipment.

	Appears in the documents in the following form: Standard or MTConnect Standard.
}


\longnewglossaryentry{streaming data}
{
  name={Streaming Data}
}
{
	The values published by a piece of equipment for the Data Entities defined by the Equipment Metadata.

	Appears in the documents in the following form: Streaming Data.
}


\longnewglossaryentry{streams information model}
{
  name={Streams Information Model}
}
{
	The rules and terminology (semantic data model) that describes the Streaming Data returned by an MTConnect Agent from a piece of equipment in response to a Sample Request or a Current Request.

	Appears in the documents in the following form: Streams Information Model.
}


\longnewglossaryentry{structural element}
{
  name={Structural Element},
  plural={Structural Elements}
}
{
	\ulheading{General meaning}:

	An XML element that organizes information that represents the physical and logical parts and sub-parts of a piece of equipment.

	Appears in the documents in the following form: Structural Element.

	\ulheading{Used to indicate hierarchy of Components}:

	When used to describe a primary physical or logical construct within a piece of equipment. 

	Appears in the documents in the following form: Top Level Structural Element.

	When used to indicate a Child Element which provides additional detail describing the physical or logical structure of a Top Level Structural Element.

	Appears in the documents in the following form: Lover Level Structural Element.
}


\longnewglossaryentry{subtype term}
{
  name={Subtype}
}
{
	\ulheading{General meaning}:

	A secondary or subordinate type of categorization or classification of information.

	In software and data modeling, a subtype is a type of data that is related to another higher-level type of data.

	Appears in the documents in the following form: subtype.

	\ulheading{Used as an attribute for a Data Entity}:

	Used as an attribute that provides a sub-categorization for the type attribute for a piece of information.

	Appears in the documents in the following form: subType.
}


\longnewglossaryentry{time stamp}
{
  name={Time Stamp}
}
{
	\ulheading{General meaning}:

	The best available estimate of the time that the value(s) for published or recorded information was measured or determined.

	Appears in the documents as "time stamp".

	\ulheading{Used as an attribute for recorded or published data}:

	An attribute that identifies the time associated with a Data Entity as stored in an MTConnect Agent.

	Appears in the documents in the following form: timestamp.
}


\longnewglossaryentry{type term}
{
  name={Type}
}
{
	\ulheading{General meaning}:

	A classification or categorization of information.

	In software and data modeling, a type is a grouping function to identify pieces of information that share common characteristics. 

	Appears in the documents in the following form: type.

	\ulheading{Used as an attribute for a Data Entity}:

	Used as an attribute that provides a categorization for piece of information that share common characteristics.

	Appears in the documents in the following form: type.
}


\longnewglossaryentry{uri}
{
  name={URI}
}
{
	Stands for Universal Resource Identifier.  

	See http://www.w3.org/TR/uri-clarification/\#RFC3986  
}


\longnewglossaryentry{url}
{
  name={URL},
  kind={facet}
}
{
	Stands for Uniform Resource Locator.  

	See http://www.w3.org/TR/uri-clarification/\#RFC3986
}


\longnewglossaryentry{urn}
{
  name={URN}
}
{
	Stands for Uniform Resource Name.  

	See http://www.w3.org/TR/uri-clarification/\#RFC3986  
}


\longnewglossaryentry{utcgmt}
{
  name={UTC/GMT}
}
{
	Stands for Coordinated Universal Time/Greenwich Mean Time.  

	UTC/GMT is the primary time standard by which the world regulates clocks and time.

	The time stamp for all information reported in an MTConnect Response document is provided in UTC/GMT format.
}


\longnewglossaryentry{uuid term}
{
  name={UUID}
}
{
	\ulheading{General meaning}:

	Stands for Universally Unique Identifier. (Can also be referred to as a GUID in some literature – Globally Unique Identifier).

	Note:  Defined in RFC 4122 of the IETF.  See https://www.ietf.org/rfc/rfc4122.txt for more information.

	Appears in the documents in the following form: UUID.

	\ulheading{Used as an attribute for an XML element}:

	Used as an attribute that provides a unique identity for a piece of information reported by an MTConnect Agent.

	Appears in the documents in the following form: uuid.
}


\longnewglossaryentry{valid data value}
{
  name={Valid Data Value},
  plural={Valid Data Values}
}
{
	One or more acceptable values or constrained values that can be reported for a Data Entity.

	Appears in the documents in the following form: Valid Data Value(s).
}


\longnewglossaryentry{w3c}
{
  name={W3C}
}
{
	Stands for World Wide Web Consortium.

	W3C is an international community of organizations and the public work together to develop internet standards.  

	W3C Standards are used as a guide within the MTConnect Standard.
}


\longnewglossaryentry{warning term}
{
  name={WARNING}
}
{
	\ulheading{General Meaning}:

	A statement or action that indicates a possible danger, problem, or other unexpected situation.

	\ulheading{Used relative to changes in an MTConnect Document}:

	Used to indicate that specific content in an MTConnect Document may be changed in a future release of the standard.

	Appears in the documents in the following form: WARNING.

	\ulheading{Used as a Valid Data Value for a Condition}:

	Used as a Valid Data Value for a Condition type Data Entity.

	Appears in the documents in the following form: WARNING.

	\ulheading{Used as an Element Name for a Data Entity}:

	Used as the Element Name for a Condition type Data Entity in an MTConnectStreams Response Document.

	Appears in the documents in the following form: Warning.
}


\longnewglossaryentry{xml}
{
  name={XML}
}
{
	Stands for EXtensible Markup Language. 

	XML defines a set of rules for encoding documents that both a human-readable and machine-readable.

	XML is the language used for all code examples in the MTConnect Standard.

	Refer to http://www.w3.org/XML for more information about XML.
}


\longnewglossaryentry{xml container}
{
  name={XML Container}
}
{
	In the MTConnect Standard, a type of XML element.

	An XML container is used to organize other XML elements that are logically related to each other.   A container may have either Data Entities or other Structural Elements as Child Elements.
}


\longnewglossaryentry{xml document}
{
  name={XML Document}
}
{
	An XML document is a structured text file encoded using XML.

	An XML document is an instantiation of an XML schema.  It has a single root XML element, conforms to the XML specification, and is structured based upon a specific schema.

	MTConnect Response Documents may be encoded as an XML document.
}


\longnewglossaryentry{xml schema}
{
  name={XML Schema}
}
{
	In the MTConnect Standard, an instantiation of a schema defining a specific document encoded in XML.
}


\longnewglossaryentry{xpath}
{
  name={XPATH}
}
{
	\ulheading{General meaning}:

	XPATH is a command structure that describes a way for a software system to locate information in an XML document.  

	XPATH uses an addressing syntax based on a path through the document's logical structure. 

	See http://www.w3.org/TR/xpath for more information on XPATH.

	Appears in the documents in the following form: XPATH.
}










\newglossaryentry{abstimeseries}
{
  type=mtc,
  category=code,
  name= {AbsTimeSeries},
  description= {}
}


\newglossaryentry{abstractconfiguration}
{
  name= {AbstractConfiguration},
  category=code,
  description= {}
}


\newglossaryentry{actuator}
{
  type=mtc,
  category=code,
  name={Actuator},
  kind={component},
  description={}
}


\newglossaryentry{adapter}
{
  type=mtc,
  name= {Adapter},
  description= {An application that provides data from a piece of equipment to an MTConnect Agent}
}


\newglossaryentry{agent}
{
  type=mtc,
  name= {Agent},
  description= {The Middleware Broker and Protocol Server for the MTConnect Standard using a REST HTTP Interface}
}


\newglossaryentry{asset}
{
  type= mtc,
  name= {Asset},
  plural= {Assets},
  description= {A complex information model relating to an entity in the manufacturing process that does not directly supply data}
}


\newglossaryentry{attribute}
{
  name= {Attribute},
  plural= {Attributes},
  description= {}
}


\newglossaryentry{auxiliary}
{
  type=mtc,
  category=code,
  name= {Auxiliary},
  kind={component},
  plural= {Auxiliaries},
  kindplural={component},
  description= {},
  descriptionplural={An XML container used to organize information for \gls{lower level} elements representing functional sub-systems that provide supplementary or extended capabilities for a piece of equipment, but they are not required for the basic operation of the equipment.}
}


\newglossaryentry{availability}
{
  type= mtc,
  category=code,
  name= {AVAILABILITY},
  description= {}
}


\newglossaryentry{available}
{
  type= mtc,
  category=code,
  name= {AVAILABLE},
  description= {}
}


\newglossaryentry{axis}
{
  type=mtc,
  category=code,
  name= {Axis},
  kind={component},
  plural={Axes},
  kindplural={component},
  description= {},
  descriptionplural={An XML container used to organize the \glspl{structural element} of a piece of equipment that perform linear or rotational motion.}
}


\newglossaryentry{electric}
{
  type=mtc,
  category=code,
  name={Electric},
  kind={systems},
  description={}
}


\newglossaryentry{loader}
{
  type=mtc,
  category=code,
  name={Loader},
  kind={auxiliaries},
  description={}
}


\newglossaryentry{wastedisposal}
{
  type=mtc,
  category=code,
  name={WasteDisposal},
  kind={auxiliaries},
  description={}
}


\newglossaryentry{toolingdelivery}
{
  type=mtc,
  category=code,
  name={Loader},
  kind={ToolingDelivery},
  description={}
}


\newglossaryentry{environmental}
{
  type=mtc,
  category=code,
  name={Environmental},
  kind={auxiliaries},
  description={}
}


\newglossaryentry{barfeeder}
{
  type=mtc,
  category=code,
  name={BarFeeder},
  kind={auxiliaries},
  description={}
}


\newglossaryentry{user}
{
  type=mtc,
  category=code,
  name={User},
  description={}
}


\newglossaryentry{units}
{
  type=mtc,
  category=code,
  name={units},
  kind={attribute},
  description={}
}


\newglossaryentry{block}
{
  type=mtc,
  category=code,
  name={Block},
  description={}
}


\newglossaryentry{buffer}
{
  type=mtc,
  name=buffer,
  description={a set of entities that are stored in memory often limited in size}
}


\newglossaryentry{buffersize}
{
  type=mtc,
  name=buffersize,
  description={}
}


\newglossaryentry{calibrationdate}
{
  type=mtc,
  category=code,
  name={CalibrationDate},
  kind={element},
  description={}
}


\newglossaryentry{nextcalibrationdate}
{
  type=mtc,
  category=code,
  name={NextCalibrationDate},
  kind={element},
  description={}
}


\newglossaryentry{calibrationinitials}
{
  type=mtc,
  category=code,
  name={CalibrationInitials},
  kind={element},
  description={}
}


\newglossaryentry{category}
{
  type=mtc,
  category=code,
  name={category},
  kind={attribute},
  description={}
}


\newglossaryentry{cdata}
{
  type=mtc,
  category=code,
  name=CDATA,
  description={The blocks of text that are not parsed by the parser, but are otherwise recognized as markup. The predefined entities such as \&lt;, \&gt;, and \&amp; require typing and are generally difficult to read in the markup}
}


\newglossaryentry{channel}
{
  type=mtc,
  category=code,
  name={Channel},
  kind={element},
  attributes={\gls{number},\gls{name}},
  elements={\gls{description},\gls{calibrationdate},\gls{nextcalibrationdate},\gls{calibrationinitials}},
  description={\gls{channel} represents each \gls{sensing element} connected to a \gls{sensor unit}.},
  plural={Channels},
  kindplural={element},
  descriptionplural={When \gls{sensor} represents multiple \glspl{sensing element}, each \gls{sensing element} is represented by a \gls{channel} for the \gls{sensor}. \glspl{channel} is an XML container used to organize information for the \glspl{sensing element}. }
}


\newglossaryentry{character data}
{
  type=mtc,
  category=code,
  name={CharacterData},
  description={See \gls{cdata}}
}


\newglossaryentry{child element}
{
  type=mtc,
  name={Child Element},
  plural={Child Elements},
  description={}
}


\newglossaryentry{component}
{
  type=mtc,
  category=code,
  plural={Components},
  kindplural={element},
  elementsplural={\gls{device}},
  name={Component},
  kind={element},
  attributes={\gls{id},\gls{nativename},\gls{sampleinterval},\gls{uuid},\gls{name}},
  elements={\gls{description},\gls{configuration},\glspl{dataitem},\glspl{component},\glspl{composition},\glspl{reference}},
  description={An abstract XML element. Replaced in the XML document by types of \gls{component} elements representing physical parts and logical functions of a piece of equipment.},
  descriptionplural={} 
}


\newglossaryentry{component componentstream}
{
  type=mtc,
  category=code,
  name={component},
  description={}
}


\newglossaryentry{componentid}
{
  type=mtc,
  category=code,
  name={componentId},
  kind={attribute},
  description={}
}


\newglossaryentry{componentref}
{
  type=mtc,
  category=code,
  name={ComponentRef},
  kind={reference},
  description={} 
}


\newglossaryentry{componentstream}
{
  type=mtc,
  category=code,
  name={ComponentStream},
  description={A XML container type element that organizes data returned from an MTConnect Agent in response to a current or sample HTTP request.} 
}


\newglossaryentry{composition}
{
  type=mtc,
  category=code,
  name={Composition},
  kind={element},
  attributes={\gls{id},\gls{uuid},\gls{name},\gls{type}},
  elements={\gls{descripiton}},
  description={},
  plural={Compositions},
  kindplural={element},
  elementsplural={\gls{composition}},
  descriptionplural={}
}


\newglossaryentry{condition}
{
  type=mtc,
  category=code,
  name={Condition},
  plural={Condition},
  description={},
  descriptionplural={}
}


\newglossaryentry{constraint}
{
  type=mtc,
  name={Constraint},
  kind={element},
  category=code,
  description={},
  plural={Constraints},
  elementsplural={\gls{value},\gls{maximum},\gls{minimum},\gls{nominal}},
  kindplural={element},
  descriptionplural={}
}


\newglossaryentry{controlled vocabulary}
{
  type=mtc,
  name={Controlled Vocabulary},
  plural= {Controlled Vocabularies},
  description= {A restricted set of values that may be published as the \gls{valid data value} for a \gls{data entity}.}
}


\newglossaryentry{configuration}
{
  type=mtc,
  category=code,
  name={Configuration},
  description={An XML element that contains technical information about a piece of equipment describing its physical layout or functional characteristics.},
  kind={element}
}


\newglossaryentry{current}
{
  type=mtc,
  category=code,
  name={Current},
  description={}
}


\newglossaryentry{current request}
{
  type=mtc,
  name={Current Request},
  description={An \glstext{http} request to the \gls{agent} for returning latest known values for the \gls{dataitem} as an \glspl{mtconnectstream} \glstext{xml} document}
}


\newglossaryentry{current httprequest}
{
  type=mtc,
  category=code,
  name={current},
  description={}
}


\newglossaryentry{data entity}
{
  type=mtc,
  name={Data Entity},
  plural={Data Entities},
  description={A primary data modeling element that represents all elements that either describe data items that may be reported by an \gls{agent} or the data items that contain the actual data published by an \gls{agent}.}
}


\newglossaryentry{dataitem}
{
  type=mtc,
  name={DataItem},
  kind={element},
  attributes={\gls{name},\gls{id},\gls{type},\gls{subtype},\gls{statistic},\gls{units},\gls{nativeunits},\gls{nativescale},\gls{category},\gls{coordinatesystem},\gls{compositionid},\gls{samplerate},\gls{representation},\gls{significantdigits}},
  elements={\gls{dataitem}},
  category=code,
  plural={DataItems},
  kindplural={element},
  description={},
  descriptionplural={}
}


\newglossaryentry{dataitemid}
{
  type=mtc,
  category=code,
  name={dataItemId},
  kind={attribute},
  description={}
}


\newglossaryentry{dataitemref}
{
  type=mtc,
  category=code,
  name={DataItemRef},
  kind={reference},
  description={}
}


\newglossaryentry{description}
{
  type=mtc,
  name={Description},
  category=code,
  description={},
  kind={element},
  attributes={\gls{manufacturer},\gls{model},\gls{serialnumber},\gls{station}}
}


\newglossaryentry{device}
{
  type=mtc,
  category=code,
  name={Device},
  description={The primary container element for each piece of equipment. \gls{device} is organized within the \glspl{device}  container.},
  kind={element},
  attributes={\gls{id},\gls{nativename},\gls{sampleinterval},\gls{uuid},\gls{name}},
  elements={\gls{description},\gls{configuration},\glspl{dataitem},\glspl{component},\glspl{composition},\glspl{reference}},
  plural={Devices},
  descriptionplural={The first, or highest level, \gls{structural element} in a \glspl{mtconnectdevice} document.},
  elementsplural={\gls{device}},
  kindplural={element}
}


\newglossaryentry{device information model}
{
  type=mtc,
  name={Device Information Model},
  plural={Devices Information Model},
  description={A set of rules and terms that describes the physical and logical configuration for a piece of equipment and the data that may be reported by that equipment.}
}


\newglossaryentry{devicestream}
{
  type=mtc,
  category=code,
  name={DeviceStream},
  description={A XML container element provided in the \glspl{stream} container in the \glspl{mtconnectstream} document.}
}


\newglossaryentry{device stream}
{
  type=mtc,
  category=code,
  name={Device Stream},
  description={}
}


\newglossaryentry{discrete representation}
{
  type=mtc,
  category=code,
  name={DISCRETE},
  kind={representation},
  description={A \gls{data entity} where each discrete occurrence of the data may have the same value as the previous occurrence of the data.}
}


\newglossaryentry{duration}
{
  type=mtc,
  category=code,
  name={duration},
  description={}
}


\newglossaryentry{element}
{
  name={Element},
  plural={Elements},
  description= {An element that defines a set of common characteristics that are shared by a group of elements.}
}


\newglossaryentry{element name}
{
  name={Element Name},
  plural={Element Names},
  description= {An element that defines a set of common characteristics that are shared by a group of elements.}
}


\newglossaryentry{event}
{
  type=mtc,
  category=code,
  name={Event},
  plural={Events},
  description={A XML element which provides the information and data reported from a piece of equipment for those \gls{dataitem} elements defined with a \gls{category} attribute of \gls{event category} in the \glspl{mtconnectdevice} document.},
  descriptionplural={A XML container type element that organizes the data reported in the \glspl{mtconnectstream} document for \gls{dataitem} elements defined in the \glspl{mtconnectdevice} document with a \gls{category} attribute of \gls{event category}.}
}


\newglossaryentry{fault state}
{
  type=mtc,
  name={Fault State},
  plural={Fault States},
  description={}
}


\newglossaryentry{filter}
{
  type=mtc,
  category=code,
  kind={element},
  name={Filter},
  description={},
  plural={Filters},
  kindplural={element},
  elementsplural={\gls{filter}}
}


\newglossaryentry{firmwareversion}
{
  type=mtc,
  category=code,
  name={FirmwareVersion},
  kind={element},
  description={}
}


\newglossaryentry{header}
{
  type=mtc,
  category=code,
  name={Header},
  description={}
}


\newglossaryentry{id}
{
  type=mtc,
  category=code,
  name={id},
  description={The unique identifier for this element.}
}


\newglossaryentry{idref}
{
  type=mtc,
  category=code,
  name={idRef},
  description={}
}


\newglossaryentry{initialvalue}
{
  name={InitialValue},
  category=code,
  type=mtc,
  kind={element},
  description={}
}


\newglossaryentry{interfaces}
{
  type=mtc,
  name=interfaces,
  description={\citetitle{MTCPart5} provides an interaction model for coordinating activities between manufacturing devices}
}


\newglossaryentry{interface component}
{
  type=mtc,
  category=code,
  name={Interface},
  kind={component},
  plural={Interfaces},
  kindplural={component},
  description={}
}


\newglossaryentry{interface}
{
  type=mtc,
  name={Interface},
  plural={Interfaces},
  description={}
}


\newglossaryentry{linear}
{
  type=mtc,
  category=code,
  name={Linear},
  kind={axes,component},
  description={}
}


\newglossaryentry{lower camel case}
{
  type=mtc,
  name={Lower Camel Case},
  description={the first word is lowercase and the remaining words are capitalized and all spaces between words are removed.}
}


\newglossaryentry{lower level}
{
  type=mtc,
  name={Lower Level},
  description={A nested element that is below a higher level element}
}


\newglossaryentry{machine}
{
  type=mtc,
  category=code,
  name={MACHINE},
  kind={coordinatesystem},
  description={}
}


\newglossaryentry{manufacturer}
{
  type=mtc,
  category=code,
  name={manufacturer},
  description={}
}


\newglossaryentry{maximum}
{
  type=mtc,
  category=code,
  name={Maximum},
  kind={element},
  description={}
}


\newglossaryentry{minimum}
{
  type=mtc,
  category=code,
  name={Minimum},
  kind={element},
  description={}
}


\newglossaryentry{message}
{
  type=mtc,
  category=code,
  name={Message},
  description={}
}


\newglossaryentry{minimumdelta}
{
  type=mtc,
  category=code,
  name={MINIMUM\_DELTA},
  kind={filter},
  description={}
}


\newglossaryentry{mixes-in}
{
  type=mtc,
  name={<<Mixes In>>},
  description={A software architecture pattern represented as a \gls{stereotype} that combines the properties of one class with another. Similar to a sub-class, it allows for inheritance in single inheritance type systems.}
}


\newglossaryentry{modbus}
{
  type=mtc,
  name=MODBUS,
  description={Modbus is a communication protocol developed by Modicon systems and is a method used for transmitting information over serial lines between electronic devices}
}


\newglossaryentry{model}
{
  type=mtc,
  category=code,
  name={model},
  description={}
}


\newglossaryentry{mtconnect}
{
  type=mtc,
  name={MTConnect},
  description={}
}


\newglossaryentry{mtconnect asset}
{
  type=mtc,
  name={MTConnect Asset},
  description={See \gls{asset}}
}


\newglossaryentry{mtconnectdevice}
{
  type=mtc,
  category=code,
  name={MTConnectDevice},
  plural={MTConnectDevices},
  description={}
}


\newglossaryentry{mtconnect device}
{
  type=mtc,
  name={MTConnect Device},
  plural={MTConnect Devices},
  description={}
}


\newglossaryentry{mtconnectstream}
{
  type=mtc,
  category=code,
  name={MTConnectStream},
  plural={MTConnectStreams},
  description={}
}


\newglossaryentry{mtconnect stream}
{
  type=mtc,
  name={MTConnect Stream},
  plural={MTConnect Streams},
  description={}
}


\newglossaryentry{mtconnect standard}
{
  name={MTConnect Standard},
  description={}
}


\newglossaryentry{name}
{
  type=mtc,
  category=code,
  name={name},
  description={}
}


\newglossaryentry{nativename}
{
  type=mtc,
  category=code,
  name={nativeName},
  description={}
}


\newglossaryentry{nativescale}
{
  type=mtc,
  category=code,
  name={nativeScale},
  kind={attribute},
  description={}
}


\newglossaryentry{nativecode}
{
  type=mtc,
  category=code,
  name={nativeCode},
  description={}
}


\newglossaryentry{nativeseverity}
{
  type=mtc,
  category=code,
  name={nativeSeverity},
  description={}
}


\newglossaryentry{nativeunits}
{
  type=mtc,
  category=code,
  name={nativeUnits},
  description={}
}


\newglossaryentry{nmtoken}
{
  name={NMTOKEN},
  description={}
}


\newglossaryentry{nominal}
{
  type=mtc,
  category=code,
  name={Nominal},
  kind={element},
  description={}
}


\newglossaryentry{occurrence}
{
  name={Occurence},
  description={Occurrence defines the number of times the content defined in the tables \MAY be provided in the usage case specified},
  plural={Occurences}
}


\newglossaryentry{number}
{
  type=mtc,
  category=code,
  name={number},
  description={}
}


\newglossaryentry{ontology}
{
  type=mtc,
  name=ontology,
  description={logical structure of the terms used to describe a domain of knowledge, including both the definitions of the applicable terms and their relationships ISO 20534:2018}
}


\newglossaryentry{parent element}
{
  type=mtc,
  name={Parent Element},
  description={}
}


\newglossaryentry{pascal case}
{
  type=mtc,
  name={Pascal Case},
  description= {The first letter of each word is capitalized and the remaining letters are in lowercase. All space is removed between letters}
}


\newglossaryentry{path}
{
  type=mtc,
  category=code,
  name={Path},
  kind={controller},
  description= {}
}


\newglossaryentry{period}
{
  type=mtc,
  category=code,
  name={PERIOD},
  kind={filter},
  description={}
}


\newglossaryentry{port}
{
  type=mtc,
  name={Port},
  description={}
}


\newglossaryentry{position}
{
  type=mtc,
  name={Position},
  description={}
}


\newglossaryentry{probe}
{
  type=mtc,
  category=code,
  name={Probe},
  description={}
}


\newglossaryentry{probe request}
{
  type=mtc,
  name={Probe Request},
  description={An \glstext{http} request to the \gls{agent} for returning metadata as an MTConnectDevices \glstext{xml} document}
}


\newglossaryentry{qname}
{
  type=mtc,
  category=code,
  name={QName},
  description={A \gls{qname}, or qualified name, is the fully qualified name of an element, attribute, or identifier in an XML document. A  \gls{qname} concisely associates the URI of an XML namespace with the local name of an element, attribute, or identifier in that namespace.}
}


\newglossaryentry{qualifier}
{
  type=mtc,
  category=code,
  name={qualifier},
  description={}
}


\newglossaryentry{realization}
{
  type=mtc,
  name={Realization},
  description={Realization is a specialized abstraction relationship between two sets of model elements, one representing a specification (the supplier) and the other represents an implementation of the latter (the client). Realization can be used to model stepwise refinement, optimizations, transformations, templates, model synthesis, framework composition, etc.}
}


\newglossaryentry{reference}
{
  type=mtc,
  category=code,
  name={Reference},
  kind={element},
  attributes={\gls{idref},\gls{name}},
  description={},
  plural={References},
  kindplural={element},
  elementsplural={\gls{reference}},
  descriptionplural={}
}


\newglossaryentry{representation}
{
  type=mtc,
  category=code,
  name={representation},
  kind={attribute},
  description={Description of a means to interpret data consisting of multiple data points or samples reported as a single value.  \newline \gls{representation} is an optional attribute.  \newline \gls{representation} will define a unique format for each set of data.  \newline \gls{representation} for \gls{timeseries representation}, \gls{discrete representation}, and \gls{value representation} are defined below in Section {\color{red} 7.2.2.12}.  \newline If \gls{representation} is not specified, it \MUST be determined to be \gls{value representation}.}
}


\newglossaryentry{resettrigger}
{
  type=mtc,
  category=code,
  name={ResetTrigger},
  kind={element},
  description={\gls{resettrigger} is an optional XML element that identifies the type of event that may cause a reset to occur. It is additional information regarding the meaning of the data that establishes an understanding of the time frame that the data represents so that the data may be correctly understood by a client software application.}
}


\newglossaryentry{resettriggered}
{
  type=mtc,
  category=code,
  name={resetTriggered},
  description={For those \gls{dataitem} elements that report data that may be periodically reset to an initial value, \gls{resettriggered} identifies when a reported value has been reset and what has caused that reset to occur.  \newline resetTriggered is an optional attribute.  \newline \gls{resettriggered} \MUST only be provided for the specific occurrence of a Data Entity reported in the MTConnectStreams document when the reset occurred and \MUSTNOT be provided for any other occurrence of the Data Entity reported in a MTConnectStreams document.}
}


\newglossaryentry{resource}
{
  type=mtc,
  category=code,
  name={Resource},
  kind={component},
  plural={Resources},
  kindplural={component},
  description={},
  descriptionplural={An XML container used to organize information for \gls{lower level} elements representing types of items, materials, and personnel that support the operation of a piece of equipment or work to be performed at a location. \glspl{resource} also represents materials or other items consumed or transformed by a piece of equipment for production of parts or other types of goods.}
}


\newglossaryentry{material}
{
  type=mtc,
  category=code,
  name={Material},
  kind={resources},
  plural={Materials},
  kindplural={resources},
  description={}
}


\newglossaryentry{stock}
{
  type=mtc,
  category=code,
  name={Stock},
  kind={materials},
  description={}
}


\newglossaryentry{personnel}
{
  type=mtc,
  category=code,
  name={Personnel},
  kind={resources},
  description={}
}


\newglossaryentry{response}
{
  type=mtc,
  name={Response},
  description={}
}


\newglossaryentry{request}
{
  type=mtc,
  name={Request},
  description={}
}


\newglossaryentry{rotary}
{
  type=mtc,
  category=code,
  name={Rotary},
  kind={axes,component},
  description={}
}


\newglossaryentry{sample}
{
  type=mtc,
  category=code,
  name={Sample},
  plural={Samples},
  description={A XML element that provides the information and data reported from a piece of equipment for those \gls{dataitem} elements defined with a \gls{category} attribute of \gls{sample category} in the \glspl{mtconnectdevice} document. },
  descriptionplural={A XML container type element that organizes the data reported in the \glspl{mtconnectstream} document for \gls{dataitem} elements defined in the \glspl{mtconnectdevice} document with a \gls{category} attribute of \gls{sample category}.}
}


\newglossaryentry{sample category}
{
  type=mtc,
  category=code,
  name={SAMPLE},
  description={SAMPLE.}
}


\newglossaryentry{sample httprequest}
{
  type=mtc,
  category=code,
  name={sample},
  description={}
}


\newglossaryentry{samplecount}
{
  type=mtc,
  category=code,
  name={sampleCount},
  description={}
}


\newglossaryentry{sampleinterval}
{
  type=mtc,
  category=code,
  name={sampleInterval},
  description={}
}


\newglossaryentry{sample request}
{
  type=mtc,
  name={Sample Request},
  description= {A request from the \gls{agent} for a stream of time series data.}
}


\newglossaryentry{samplerate}
{
  type=mtc,
  category=code,
  name={sampleRate},
  kind={attribute},
  description={}
}


\newglossaryentry{sensing element}
{
  type=mtc,
  name={sensing element},
  description={},
  plural={sensing elements}
}


\newglossaryentry{sensor element}
{
  type=mtc,
  name={sensor element},
  plural={sensor elements},
  description={}
}


\newglossaryentry{sensor}
{
  type=mtc,
  category=code,
  name={Sensor},
  kind={auxiliaries,component},
  plural={Sensors},
  description= {}
}


\newglossaryentry{sensor unit}
{
  type=mtc,
  name={sensor unit},
  plural={sensor units},
  description= {}
}


\newglossaryentry{sensorconfiguration}
{
  type=mtc,
  name={SensorConfiguration},
  kind={element},
  elements={\gls{firmwareversion},\gls{calibrationdate},\gls{nextcalibrationdate},\gls{calibrationinitials},\glspl{channel}},
  description= {}
}


\newglossaryentry{serialnumber}
{
  type=mtc,
  category=code,
  name={serialNumber},
  description={}
}


\newglossaryentry{sequence}
{
  type=mtc,
  category=code,
  name={sequence},
  description={}
}


\newglossaryentry{sequence number}
{
  type=mtc,
  name={sequence number},
  plural={sequence numbers},
  description={}
}


\newglossaryentry{significantdigits}
{
  type=mtc,
  category=code,
  name={significantDigits},
  kind={attribute},
  description={}
}


\newglossaryentry{slot}
{
  type=mtc,
  name=Slot,
  description={Slot is UML element which specifies that an instance has a value or values for a specific structural feature. UML specification also says that a slot gives the value or values of a structural feature of the instance. An instance can have one slot per structural feature of its classifiers, including inherited features.}
}


\newglossaryentry{station}
{
  type=mtc,
  category=code,
  name={station},
  plural={stations},
  description={}
}


\newglossaryentry{source}
{
  type=mtc,
  category=code,
  name={Source},
  kind={element},
  attributes={\gls{componentid},\gls{dataitemid},\gls{compositionid}},
  description={}
}


\newglossaryentry{statistic}
{
  type=mtc,
  category=code,
  name={statistic},
  kind={attribute},
  description={}
}


\newglossaryentry{stereotype}
{
  type=mtc,
  name=stereotype,
  description={A profile class which defines how an existing metaclass may be extended as part of a profile. It enables the use of a platform or domain specific terminology or notation in place of, or in addition to, the ones used for the extended metaclass. A stereotype is denoted by <<[name]>>.}
}


\newglossaryentry{stream}
{
  type=mtc,
  category=code,
  name={Stream},
  plural={Streams},
  description={},
  descriptionplural= {The first, or highest, level XML container element in an \glspl{mtconnectstream} \gls{response} Document provided by an \gls{agent} in response to a \gls{sample httprequest} or \gls{current httprequest} HTTP \gls{request}.}
}


\newglossaryentry{streams information model}
{
  type=mtc,
  name={Streams Information Model},
  description={}
}


\newglossaryentry{structural element}
{
  type=mtc,
  name={Structural Element},
  plural={Structural Elements},
  description={An XML element that organizes information that represents the physical and logical parts and sub-parts of a piece of equipment.}
}


\newglossaryentry{subtype}
{
  type=mtc,
  category=code,
  name={subType},
  description={}
}


\newglossaryentry{system}
{
  type=mtc,
  category=code,
  name={System},
  kind={component},
  plural={Systems},
  kindplural={component},
  description={},
  descriptionplural={An XML container used to organize information for \gls{lower level} elements representing the major sub-systems that are permanently integrated into a piece of equipment.}
}


\newglossaryentry{time series}
{
  type=mtc,
  name={Time Series},
  description={A \gls{dataitem} representation of a contiguous vector of values supporting high frequency data rates}
}


\newglossaryentry{timeseries representation}
{
  type=mtc,
  category=code,
  name={TIME\_SERIES},
  kind={representation},
  description={A series of sampled data. }
}


\newglossaryentry{timestamp}
{
  type=mtc,
  category=code,
  name={timestamp},
  description={}
}


\newglossaryentry{top level}
{
  type=mtc,
  name={Top Level},
  description={}
}


\newglossaryentry{type}
{
  type=mtc,
  category=code,
  name={type},
  plural={types},
  description={}
}


\newglossaryentry{umlassociation}
{
  type=mtc,
  name=UMLAssociation,
  description={A relationship between UML Classes}
}


\newglossaryentry{unavailable}
{
  type=mtc,
  category=code,
  name={UNAVAILABLE},
  description={}
}


\newglossaryentry{uuid}
{
  type=mtc,
  category=code,
  name={uuid},
  description={}
}


\newglossaryentry{valid data value}
{
  type=mtc,
  name={Valid Data Value},
  description={}
}


\newglossaryentry{value}
{
  type=mtc,
  category=code,
  name={Value},
  kind={element},
  description={}
}


\newglossaryentry{value representation}
{
  type=mtc,
  category=code,
  name={VALUE},
  kind={representation},
  description={The measured value of the sample data.}
}


\newglossaryentry{work}
{
  type=mtc,
  category=code,
  name={WORK},
  kind={coordinatesystem},
  description={}
}


\newglossaryentry{xs:float}
{
  type=mtc,
  name={xs:float},
  description={}
}


\newglossaryentry{xs:lang}
{
  type=mtc,
  category=code,
  name={xs:lang},
  description={}
}


\newglossaryentry{controller}
{
  type=mtc,
  category=code,
  name={Controller},
  kind={component},
  description= {}
}


\newglossaryentry{coordinatesystem}
{
  type=mtc,
  category=code,
  name={coordinateSystem},
  kind={attribute},
  description={}
}


\newglossaryentry{actuator type}
{
  type=mtc,
  category=code,
  name={ACTUATOR},
  kind={composition},
  description={A mechanism for moving or controlling a mechanical part of a piece of equipment.   \newline It takes energy usually provided by air, electric current, or liquid and converts the energy into some kind of motion.  }
}


\newglossaryentry{amplifier}
{
  type=mtc,
  category=code,
  name={AMPLIFIER},
  kind={composition},
  description={An electronic component or circuit for amplifying power, electric current, or voltage.}
}


\newglossaryentry{ballscrew}
{
  type=mtc,
  category=code,
  name={BALLSCREW},
  kind={composition},
  description={A mechanical structure for transforming rotary motion into linear motion.}
}


\newglossaryentry{belt}
{
  type=mtc,
  category=code,
  name={BELT},
  kind={composition},
  description={An endless flexible band used to transmit motion for a piece of equipment or to convey materials and objects.}
}


\newglossaryentry{brake}
{
  type=mtc,
  category=code,
  name={BRAKE},
  kind={composition},
  description={A mechanism for slowing or stopping a moving object by the absorption or transfer of the energy of momentum, usually by means of friction, electrical force, or magnetic force.}
}


\newglossaryentry{chain}
{
  type=mtc,
  category=code,
  name={CHAIN},
  kind={composition},
  description={An interconnected series of objects that band together and are used to transmit motion for a piece of equipment or to convey materials and objects.}
}


\newglossaryentry{chopper}
{
  type=mtc,
  category=code,
  name={CHOPPER},
  kind={composition},
  description={A mechanism used to break material into smaller pieces.}
}


\newglossaryentry{chuck}
{
  type=mtc,
  category=code,
  name={CHUCK},
  kind={composition},
  description={A mechanism that holds a part, stock material, or any other item in place.}
}


\newglossaryentry{chuck component}
{
  type=mtc,
  category=code,
  name={Chuck},
  kind={axes,component},
  description={}
}


\newglossaryentry{chute}
{
  type=mtc,
  category=code,
  name={CHUTE},
  kind={composition},
  description={An inclined channel for conveying material.}
}


\newglossaryentry{circuitbreaker}
{
  type=mtc,
  category=code,
  name={CIRCUIT\_BREAKER},
  kind={composition},
  description={A mechanism for interrupting an electric circuit.}
}


\newglossaryentry{clamp}
{
  type=mtc,
  category=code,
  name={CLAMP},
  kind={composition},
  description={A mechanism used to strengthen, support, or fasten objects in place.}
}


\newglossaryentry{compressor}
{
  type=mtc,
  category=code,
  name={COMPRESSOR},
  kind={composition},
  description={A pump or other mechanism for reducing volume and increasing pressure of gases in order to condense the gases to drive pneumatically powered pieces of equipment.}
}


\newglossaryentry{door}
{
  type=mtc,
  category=code,
  name={DOOR},
  kind={composition},
  description={A mechanical mechanism or closure that can cover a physical access portal into a piece of equipment allowing or restricting access to other parts of the equipment.}
}


\newglossaryentry{door component}
{
  type=mtc,
  category=code,
  name={Door},
  kind={component},
  description={}
}


\newglossaryentry{drain}
{
  type=mtc,
  category=code,
  name={DRAIN},
  kind={composition},
  description={A mechanism that allows material to flow for the purpose of drainage from, for example, a vessel or tank.}
}


\newglossaryentry{encoder}
{
  type=mtc,
  category=code,
  name={ENCODER},
  kind={composition},
  description={A mechanism used to measure rotary position.}
}


\newglossaryentry{fan}
{
  type=mtc,
  category=code,
  name={FAN},
  kind={composition},
  description={Any mechanism for producing a current of air.}
}


\newglossaryentry{filter type}
{
  type=mtc,
  category=code,
  name={FILTER},
  kind={composition},
  description={Any substance or structure through which liquids or gases are passed to remove suspended impurities or to recover solids.}
}


\newglossaryentry{gripper}
{
  type=mtc,
  category=code,
  name={GRIPPER},
  kind={composition},
  description={A mechanism that holds a part, stock material, or any other item in place.}
}


\newglossaryentry{hopper}
{
  type=mtc,
  category=code,
  name={HOPPER},
  kind={composition},
  description={A chamber or bin in which materials are stored temporarily, typically being filled through the top and dispensed through the bottom.}
}


\newglossaryentry{hydraulic}
{
  type=mtc,
  category=code,
  name={Hydraulic},
  kind={systems},
  description={}
}


\newglossaryentry{pneumatic}
{
  type=mtc,
  category=code,
  name={Pneumatic},
  kind={systems},
  description={}
}


\newglossaryentry{coolant}
{
  type=mtc,
  category=code,
  name={Coolant},
  kind={systems},
  description={}
}


\newglossaryentry{lubrication}
{
  type=mtc,
  category=code,
  name={Lubrication},
  kind={systems},
  description={}
}


\newglossaryentry{enclosure}
{
  type=mtc,
  category=code,
  name={Enclosure},
  kind={systems},
  description={}
}


\newglossaryentry{protective}
{
  type=mtc,
  category=code,
  name={Protective},
  kind={systems},
  description={}
}


\newglossaryentry{processpower}
{
  type=mtc,
  category=code,
  name={ProcessPower},
  kind={systems},
  description={}
}


\newglossaryentry{feeder}
{
  type=mtc,
  category=code,
  name={Feeder},
  kind={systems},
  description={}
}


\newglossaryentry{dielectric}
{
  type=mtc,
  category=code,
  name={Dielectric},
  kind={systems},
  description={}
}


\newglossaryentry{linearpositionfeedback}
{
  type=mtc,
  category=code,
  name={LINEAR\_POSITION\_FEEDBACK},
  kind={composition},
  description={A mechanism that measures linear motion or position.}
}


\newglossaryentry{motor}
{
  type=mtc,
  category=code,
  name={MOTOR},
  kind={composition},
  description={A mechanism that converts electrical, pneumatic, or hydraulic energy into mechanical energy.}
}


\newglossaryentry{oil}
{
  type=mtc,
  category=code,
  name={OIL},
  kind={composition},
  description={A viscous liquid.}
}


\newglossaryentry{powersupply}
{
  type=mtc,
  category=code,
  name={POWER\_SUPPLY},
  kind={composition},
  description={A unit that provides power to electric mechanisms.}
}


\newglossaryentry{pulley}
{
  type=mtc,
  category=code,
  name={PULLEY},
  kind={composition},
  description={A mechanism or wheel that turns in a frame or block and serves to change the direction of or to transmit force.}
}


\newglossaryentry{pump}
{
  type=mtc,
  category=code,
  name={PUMP},
  kind={composition},
  description={An apparatus raising, driving, exhausting, or compressing fluids or gases by means of a piston, plunger, or set of rotating vanes.}
}


\newglossaryentry{sensingelement}
{
  type=mtc,
  category=code,
  name={SENSING\_ELEMENT},
  kind={composition},
  description={A mechanism that provides a signal or measured value.}
}


\newglossaryentry{storagebattery}
{
  type=mtc,
  category=code,
  name={STORAGE\_BATTERY},
  kind={composition},
  description={A component consisting of one or more cells, in which chemical energy is converted into electricity and used as a source of power. }
}


\newglossaryentry{switch}
{
  type=mtc,
  category=code,
  name={SWITCH},
  kind={composition},
  description={A mechanism for turning on or off an electric current or for making or breaking a circuit.}
}


\newglossaryentry{tank}
{
  type=mtc,
  category=code,
  name={TANK},
  kind={composition},
  description={A receptacle or container for holding material.}
}


\newglossaryentry{tensioner}
{
  type=mtc,
  category=code,
  name={TENSIONER},
  kind={composition},
  description={A mechanism that provides or applies a stretch or strain to another mechanism.}
}


\newglossaryentry{transformer}
{
  type=mtc,
  category=code,
  name={TRANSFORMER},
  kind={composition},
  description={A mechanism that transforms electric energy from a source to a secondary circuit.}
}


\newglossaryentry{valve}
{
  type=mtc,
  category=code,
  name={VALVE},
  kind={composition},
  description={Any mechanism for halting or controlling the flow of a liquid, gas, or other material through a passage, pipe, inlet, or outlet.}
}


\newglossaryentry{water}
{
  type=mtc,
  category=code,
  name={WATER},
  kind={composition},
  description={A fluid.}
}


\newglossaryentry{wire}
{
  type=mtc,
  category=code,
  name={WIRE},
  kind={composition},
  description={A string like piece or filament of relatively rigid or flexible material provided in a variety of diameters.}
}


\newglossaryentry{average}
{
  type=mtc,
  category=code,
  name={AVERAGE},
  kind={statistic},
  description={Mathematical Average value calculated for the data item during the calculation period.}
}


\newglossaryentry{kurtosis}
{
  type=mtc,
  category=code,
  name={KURTOSIS},
  kind={statistic},
  description={A measure of the peakedness of a probability distribution; i.e., the shape of the distribution curve.}
}


\newglossaryentry{maximum statistic}
{
  type=mtc,
  category=code,
  name={MAXIMUM},
  kind={statistic},
  description={Maximum or peak value recorded for the data item during the calculation period.}
}


\newglossaryentry{median}
{
  type=mtc,
  category=code,
  name={MEDIAN},
  kind={statistic},
  description={The middle number of a series of numbers.}
}


\newglossaryentry{minimum statistic}
{
  type=mtc,
  category=code,
  name={MINIMUM},
  kind={statistic},
  description={Minimum value recorded for the data item during the calculation period.}
}


\newglossaryentry{mode}
{
  type=mtc,
  category=code,
  name={MODE},
  kind={statistic},
  description={The number in a series of numbers that occurs most often.}
}


\newglossaryentry{range}
{
  type=mtc,
  category=code,
  name={RANGE},
  kind={statistic},
  description={Difference between the Maximum and Minimum value of a data item during the calculation period.  Also represents Peak-to-Peak measurement in a waveform.}
}


\newglossaryentry{rootmeansquare}
{
  type=mtc,
  category=code,
  name={ROOT\_MEAN\_SQUARE},
  kind={statistic},
  description={Mathematical Root Mean Square (RMS) value calculated for the data item during the calculation period.}
}


\newglossaryentry{standarddeviation}
{
  type=mtc,
  category=code,
  name={STANDARD\_DEVIATION},
  kind={statistic},
  description={Statistical Standard Deviation value calculated for the data item during the calculation period.}
}


\newglossaryentry{acceleration sample}
{
  type=mtc,
  category=code,
  name={ACCELERATION},
  elementname=\cfont{Acceleration},
  description={The measurement of the rate of change of velocity.},
  units=\cfont{\gls{millimeterpersecondsquared}},
  kind={type,sample},
  facet={\gls{float}}
}


\newglossaryentry{accumulatedtime sample}
{
  type=mtc,
  category=code,
  name={ACCUMULATED\_TIME},
  elementname=\cfont{AccumulatedTime},
  description={The measurement of accumulated time for an activity or event.},
  units=\cfont{\gls{second}},
  kind={type,sample},
  facet={\gls{float}}
}


\newglossaryentry{amperage sample}
{
  type=mtc,
  category=code,
  name={AMPERAGE},
  elementname=\cfont{Amperage},
  description={The measurement of electrical current.},
  units=\cfont{\gls{ampere}},
  kind={type,sample},
  facet={\gls{float}}
}


\newglossaryentry{actual amperage sample}
{
  type=mtc,
  category=code,
  name={ACTUAL},
  description={The measured amperage being delivered from a power source.},
  units=\cfont{\gls{ampere}},
  kind={subtype,sample},
  facet={\gls{float}}
}


\newglossaryentry{alternating amperage sample}
{
  type=mtc,
  category=code,
  name={ALTERNATING},
  description={The measurement of alternating current.   If not specified further in statistic, defaults to RMS current  },
  units=\cfont{\gls{ampere}},
  kind={subtype,sample},
  facet={\gls{float}}
}


\newglossaryentry{direct amperage sample}
{
  type=mtc,
  category=code,
  name={DIRECT},
  description={The measurement of DC current},
  units=\cfont{\gls{ampere}},
  kind={subtype,sample},
  facet={\gls{float}}
}


\newglossaryentry{target amperage sample}
{
  type=mtc,
  category=code,
  name={TARGET},
  description={The desired or preset amperage to be delivered from a power source.},
  units=\cfont{\gls{ampere}},
  kind={subtype,sample},
  facet={\gls{float}}
}


\newglossaryentry{angle sample}
{
  type=mtc,
  category=code,
  name={ANGLE},
  elementname=\cfont{Angle},
  description={The measurement of angular position.},
  units=\cfont{\gls{degree}},
  kind={type,sample},
  facet={\gls{float}}
}


\newglossaryentry{actual angle sample}
{
  type=mtc,
  category=code,
  name={ACTUAL},
  description={The actual angular position as read from the physical component.},
  units=\cfont{\gls{degree}},
  kind={subtype,sample},
  facet={\gls{float}}
}


\newglossaryentry{commanded angle sample}
{
  type=mtc,
  category=code,
  name={COMMANDED},
  description={A calculated value for angular position computed by the \gls{controller} type component},
  units=\cfont{\gls{degree}},
  kind={subtype,sample},
  facet={\gls{float}}
}


\newglossaryentry{angularacceleration sample}
{
  type=mtc,
  category=code,
  name={ANGULAR\_ACCELERATION},
  elementname=\cfont{AngularAcceleration},
  description={The measurement rate of change of angular velocity.},
  units=\cfont{\gls{degreepersecondsquared}},
  kind={type,sample},
  facet={\gls{float}}
}


\newglossaryentry{angularvelocity sample}
{
  type=mtc,
  category=code,
  name={ANGULAR\_VELOCITY},
  elementname=\cfont{AngularVelocity},
  representation=\cfont{AngularVelocityTimeSeries},
  description={The measurement of the rate of change of angular position.},
  units=\cfont{\gls{degreepersecond}},
  kind={type,sample},
  facet={\gls{float}}
}


\newglossaryentry{axisfeedrate sample}
{
  type=mtc,
  category=code,
  name={AXIS\_FEEDRATE},
  elementname=\cfont{AxisFeedrate},
  description={The measurement of the feedrate of a linear axis.},
  units=\cfont{\gls{millimeterpersecond}},
  kind={type,sample},
  facet={\gls{float}}
}


\newglossaryentry{actual axisfeedrate sample}
{
  type=mtc,
  category=code,
  name={ACTUAL},
  description={The measured value of the feedrate of a linear axis.},
  units=\cfont{\gls{millimeterpersecond}},
  kind={subtype,sample},
  facet={\gls{float}}
}


\newglossaryentry{commanded axisfeedrate sample}
{
  type=mtc,
  category=code,
  name={COMMANDED},
  description={The feedrate of a linear axis as specified by the \gls{controller} type component.The COMMANDED feedrate is a calculated value that includes adjustments and overrides.},
  units=\cfont{\gls{millimeterpersecond}},
  kind={subtype,sample},
  facet={\gls{float}}
}


\newglossaryentry{jog axisfeedrate sample}
{
  type=mtc,
  category=code,
  name={JOG},
  description={The feedrate specified by a logic or motion program, by a pre-set value, or set by a switch as the feedrate for a linear axis when operating in a manual state or method (jogging).  },
  units=\cfont{\gls{millimeterpersecond}},
  kind={subtype,sample},
  facet={\gls{float}}
}


\newglossaryentry{override axisfeedrate sample}
{
  type=mtc,
  category=code,
  name={OVERRIDE},
  description={The operators overridden value. Percent of commanded.  DEPRECATED in Version 1.3.   See \gls{event category} category data items.},
  units=\cfont{\gls{percent}},
  kind={subtype,sample},
  facet={\gls{float}}
}


\newglossaryentry{programmed axisfeedrate sample}
{
  type=mtc,
  category=code,
  name={PROGRAMMED},
  description={The feedrate specified by a logic or motion program or set by a switch for a linear axis.},
  units=\cfont{\gls{millimeterpersecond}},
  kind={subtype,sample},
  facet={\gls{float}}
}


\newglossaryentry{rapid axisfeedrate sample}
{
  type=mtc,
  category=code,
  name={RAPID},
  description={The feedrate specified by a logic or motion program, by a pre-set value, or set by a switch as the feedrate for a linear axis when operating in a rapid positioning mode.},
  units=\cfont{\gls{millimeterpersecond}},
  kind={subtype,sample},
  facet={\gls{float}}
}


\newglossaryentry{clocktime sample}
{
  type=mtc,
  category=code,
  name={CLOCK\_TIME},
  elementname=\cfont{ClockTime},
  description={The value provided by a timing device at a specific point in time.},
  units=\cfont{yyyy-mm-ddthh:mm:ss.ffff},
  kind={type,sample},
  facet={\gls{datetime}}
}


\newglossaryentry{concentration sample}
{
  type=mtc,
  category=code,
  name={CONCENTRATION},
  elementname=\cfont{Concentration},
  description={The measurement of the percentage of one component within a mixture of components},
  units=\cfont{\gls{percent}},
  kind={type,sample},
  facet={\gls{float}}
}


\newglossaryentry{conductivity sample}
{
  type=mtc,
  category=code,
  name={CONDUCTIVITY},
  elementname=\cfont{Conductivity},
  description={The measurement of the ability of a material to conduct electricity.},
  units=\cfont{\gls{siemenspermeter}},
  kind={type,sample},
  facet={\gls{float}}
}


\newglossaryentry{displacement sample}
{
  type=mtc,
  category=code,
  name={DISPLACEMENT},
  elementname=\cfont{Displacement},
  description={The measurement of the change in position of an object.},
  units=\cfont{\gls{millimeter}},
  kind={type,sample},
  facet={\gls{float}}
}


\newglossaryentry{electricalenergy sample}
{
  type=mtc,
  category=code,
  name={ELECTRICAL\_ENERGY},
  elementname=\cfont{ElectricalEnergy},
  description={The measurement of electrical energy consumption by a component.},
  units=\cfont{\gls{wattsecond}},
  kind={type,sample},
  facet={\gls{float}}
}


\newglossaryentry{equipmenttimer sample}
{
  type=mtc,
  category=code,
  name={EQUIPMENT\_TIMER},
  elementname=\cfont{EquipmentTimer},
  description={The measurement of the amount of time a piece of equipment or a sub-part of a piece of equipment has performed specific activities.},
  units=\cfont{\gls{second}},
  kind={type,sample},
  facet={\gls{float}}
}


\newglossaryentry{delay equipmenttimer sample}
{
  type=mtc,
  category=code,
  name={DELAY},
  description={Measurement of the time that a piece of equipment is waiting for an event or an action to occur.},
  units=\cfont{\gls{second}},
  kind={subtype,sample},
  facet={\gls{float}}
}


\newglossaryentry{loaded equipmenttimer sample}
{
  type=mtc,
  category=code,
  name={LOADED},
  description={Measurement of the time that the sub-parts of a piece of equipment are under load. \newline Example: For traditional machine tools, this is a measurement of the time that the cutting tool is assumed to be engaged with the part.},
  units=\cfont{\gls{second}},
  kind={subtype,sample},
  facet={\gls{float}}
}


\newglossaryentry{operating equipmenttimer sample}
{
  type=mtc,
  category=code,
  name={OPERATING},
  description={Measurement of the time that the major sub-parts of a piece of equipment are powered or performing any activity whether producing a part or product or not.   \newline Example: For traditional machine tools, this includes \gls{working equipmentmode event}, plus idle time.},
  units=\cfont{\gls{second}},
  kind={subtype,sample},
  facet={\gls{float}}
}


\newglossaryentry{powered equipmenttimer sample}
{
  type=mtc,
  category=code,
  name={POWERED},
  description={The measurement of time that primary power is applied to the piece of equipment and, as a minimum, the controller or logic portion of the piece of equipment is powered and functioning or components that are required to remain on are powered.Example: Heaters for an extrusion machine that are required to be powered even when the equipment is turned off.},
  units=\cfont{\gls{second}},
  kind={subtype,sample},
  facet={\gls{float}}
}


\newglossaryentry{working equipmenttimer sample}
{
  type=mtc,
  category=code,
  name={WORKING},
  description={Measurement of the time that a piece of equipment is performing any activity  the equipment is active and performing a function under load or not. \newline Example: For traditional machine tools, this includes LOADED, plus rapid moves, tool changes, etc.},
  units=\cfont{\gls{second}},
  kind={subtype,sample},
  facet={\gls{float}}
}


\newglossaryentry{filllevel sample}
{
  type=mtc,
  category=code,
  name={FILL\_LEVEL},
  elementname=\cfont{FillLevel},
  description={The measurement of the amount of a substance remaining compared to the planned maximum amount of that substance.},
  units=\cfont{\gls{percent}},
  kind={type,sample},
  facet={\gls{float}}
}


\newglossaryentry{flow sample}
{
  type=mtc,
  category=code,
  name={FLOW},
  elementname=\cfont{Flow},
  description={The measurement of the rate of flow of a fluid.},
  units=\cfont{\gls{literpersecond}},
  kind={type,sample},
  facet={\gls{float}}
}


\newglossaryentry{frequency sample}
{
  type=mtc,
  category=code,
  name={FREQUENCY},
  elementname=\cfont{Frequency},
  description={The measurement of the number of occurrences of a repeating event per unit time.},
  units=\cfont{\gls{hertz}},
  kind={type,sample},
  facet={\gls{float}}
}


\newglossaryentry{globalposition sample}
{
  type=mtc,
  category=code,
  name=\deprecated{GLOBAL\_POSITION},
  elementname=\deprecated{\cfont{GlobalPosition}},
  description={DEPRECATED in Version 1.1},
  units={},
  kind={type,sample},
  facet={\gls{float}}
}


\newglossaryentry{length sample}
{
  type=mtc,
  category=code,
  name={LENGTH},
  elementname=\cfont{Length},
  description={The measurement of the length of an object.},
  units=\cfont{\gls{millimeter}},
  kind={type,sample},
  facet={\gls{float}}
}


\newglossaryentry{remaining length sample}
{
  type=mtc,
  category=code,
  name={REMAINING},
  description={The remaining total length of an object.},
  units=\cfont{\gls{millimeter}},
  kind={subtype,sample},
  facet={\gls{float}}
}


\newglossaryentry{standard length sample}
{
  type=mtc,
  category=code,
  name={STANDARD},
  description={The standard or original length of an object.},
  units=\cfont{\gls{millimeter}},
  kind={subtype,sample},
  facet={\gls{float}}
}


\newglossaryentry{useable length sample}
{
  type=mtc,
  category=code,
  name={USEABLE},
  description={The remaining useable length of an object.},
  units=\cfont{\gls{millimeter}},
  kind={subtype,sample},
  facet={\gls{float}}
}


\newglossaryentry{level sample}
{
  type=mtc,
  category=code,
  name={LEVEL},
  elementname=\cfont{Level},
  description={DEPRECATED in Version 1.2.  See \gls{filllevel sample}},
  units={},
  kind={type,sample},
  facet={\gls{float}}
}


\newglossaryentry{linearforce sample}
{
  type=mtc,
  category=code,
  name={LINEAR\_FORCE},
  elementname=\cfont{LinearForce},
  description={The measurement of the push or pull introduced by an actuator or exerted on an object.},
  units=\cfont{\gls{newton}},
  kind={type,sample},
  facet={\gls{float}}
}


\newglossaryentry{load sample}
{
  type=mtc,
  category=code,
  name={LOAD},
  elementname=\cfont{Load},
  description={The measurement of the actual versus the standard rating of a piece of equipment.},
  units=\cfont{\gls{percent}},
  kind={type,sample},
  facet={\gls{float}}
}


\newglossaryentry{mass sample}
{
  type=mtc,
  category=code,
  name={MASS},
  elementname=\cfont{Mass},
  description={The measurement of the mass of an object(s) or an amount of material.},
  units=\cfont{\gls{kilogram}},
  kind={type,sample},
  facet={\gls{float}}
}


\newglossaryentry{pathfeedrate sample}
{
  type=mtc,
  category=code,
  name={PATH\_FEEDRATE},
  elementname=\cfont{PathFeedrate},
  description={The measurement of the feedrate for the axes, or a single axis, associated with a \glselementname{path} component a vector.},
  units=\cfont{\gls{millimeterpersecond}},
  kind={type,sample},
  facet={\gls{float}}
}


\newglossaryentry{actual pathfeedrate sample}
{
  type=mtc,
  category=code,
  name={ACTUAL},
  description={The measured value of the feedrate of the axes, or a single axis, associated with a path component.},
  units=\cfont{\gls{millimeterpersecond}},
  kind={subtype,sample},
  facet={\gls{float}}
}


\newglossaryentry{commanded pathfeedrate sample}
{
  type=mtc,
  category=code,
  name={COMMANDED},
  description={The feedrate as specified by the \gls{controller} type component for the axes, or a single axis, associated with a Path component.The COMMANDED feedrate is a calculated value that includes adjustments and overrides.},
  units=\cfont{\gls{millimeterpersecond}},
  kind={subtype,sample},
  facet={\gls{float}}
}


\newglossaryentry{jog pathfeedrate sample}
{
  type=mtc,
  category=code,
  name={JOG},
  description={The feedrate specified by a logic or motion program, by a pre-set value, or set by a switch as the feedrate for the axes, or a single axis, associated with a Path when operating in a manual state or method (jogging).  },
  units=\cfont{\gls{millimeterpersecond}},
  kind={subtype,sample},
  facet={\gls{float}}
}


\newglossaryentry{override pathfeedrate sample}
{
  type=mtc,
  category=code,
  name={OVERRIDE},
  description={The operators overridden value.  Percent of commanded. DEPRECATED in Version 1.3.   See \gls{event category} category \glspl{dataitem}.},
  units=\cfont{\gls{percent}},
  kind={subtype,sample},
  facet={\gls{float}}
}


\newglossaryentry{programmed pathfeedrate sample}
{
  type=mtc,
  category=code,
  name={PROGRAMMED},
  description={The feedrate specified by a logic or motion program or set by a switch as the feedrate for the axes, or a single axis, associated with a Path.},
  units=\cfont{\gls{millimeterpersecond}},
  kind={subtype,sample},
  facet={\gls{float}}
}


\newglossaryentry{rapid pathfeedrate sample}
{
  type=mtc,
  category=code,
  name={RAPID},
  description={The feedrate specified by a logic or motion program, by a pre-set value, or set by a switch as the feedrate for the axes, or a single axis, associated with a Path when operating in a rapid positioning mode.},
  units=\cfont{\gls{millimeterpersecond}},
  kind={subtype,sample},
  facet={\gls{float}}
}


\newglossaryentry{pathposition sample}
{
  type=mtc,
  category=code,
  name={PATH\_POSITION},
  elementname=\cfont{PathPosition},
  description={A measured or calculated position of a control point associated with a \gls{controller} element, or \gls{path} element if provided, of a piece of equipment.},
  units=\cfont{\gls{millimeter3d}},
  kind={type,sample},
  facet={\gls{array3d}}
}


\newglossaryentry{actual pathposition sample}
{
  type=mtc,
  category=code,
  name={ACTUAL},
  description={The measured position of the current program control point as reported by the piece of equipment.},
  units=\cfont{\gls{millimeter3d}},
  kind={subtype,sample},
  facet={\gls{array3d}}
}


\newglossaryentry{commanded pathposition sample}
{
  type=mtc,
  category=code,
  name={COMMANDED},
  description={The position computed by the \gls{controller} type component.},
  units=\cfont{\gls{millimeter3d}},
  kind={subtype,sample},
  facet={\gls{array3d}}
}


\newglossaryentry{probe pathposition sample}
{
  type=mtc,
  category=code,
  name={PROBE},
  description={The position provided by a measurement probe.},
  units=\cfont{\gls{millimeter3d}},
  kind={subtype,sample},
  facet={\gls{array3d}}
}


\newglossaryentry{target pathposition sample}
{
  type=mtc,
  category=code,
  name={TARGET},
  description={The desired end position for a movement or a series of movements. Multiple discrete movements may need to be completed to achieve the final \gls{target position sample} position.  },
  units=\cfont{\gls{millimeter3d}},
  kind={subtype,sample},
  facet={\gls{array3d}}
}


\newglossaryentry{ph sample}
{
  type=mtc,
  category=code,
  name={PH},
  elementname=\cfont{PH},
  description={The measurement of the acidity or alkalinity.},
  units=\cfont{\gls{ph}},
  kind={type,sample},
  facet={\gls{float}}
}


\newglossaryentry{position sample}
{
  type=mtc,
  category=code,
  name={POSITION},
  elementname=\cfont{Position},
  description={A measured or calculated position of a \gls{component} element as reported by a piece of equipment.},
  units=\cfont{\gls{millimeter}},
  kind={type,sample},
  facet={\gls{float}}
}


\newglossaryentry{actual position sample}
{
  type=mtc,
  category=code,
  name={ACTUAL},
  description={The physical measured position of the control point for a Component.},
  units=\cfont{\gls{millimeter}},
  kind={subtype,sample},
  facet={\gls{float}}
}


\newglossaryentry{commanded position sample}
{
  type=mtc,
  category=code,
  name={COMMANDED},
  description={A position calculated by the \gls{controller} type component for a discrete movement.},
  units=\cfont{\gls{millimeter}},
  kind={subtype,sample},
  facet={\gls{float}}
}


\newglossaryentry{programmed position sample}
{
  type=mtc,
  category=code,
  name={PROGRAMMED},
  description={The position of the control point for a Component specified by a logic or motion program },
  units=\cfont{\gls{millimeter}},
  kind={subtype,sample},
  facet={\gls{float}}
}


\newglossaryentry{target position sample}
{
  type=mtc,
  category=code,
  name={TARGET},
  description={The desired end position of the control point for a Component resulting from a movement or a series of movements.  Multiple discrete movements may need to be completed to achieve the final \gls{target position sample} position.},
  units=\cfont{\gls{millimeter}},
  kind={subtype,sample},
  facet={\gls{float}}
}


\newglossaryentry{powerfactor sample}
{
  type=mtc,
  category=code,
  name={POWER\_FACTOR},
  elementname=\cfont{PowerFactor},
  description={The measurement of the ratio of real power flowing to a load to the apparent power in that AC circuit.},
  units=\cfont{\gls{percent}},
  kind={type,sample},
  facet={\gls{float}}
}


\newglossaryentry{pressure sample}
{
  type=mtc,
  category=code,
  name={PRESSURE},
  elementname=\cfont{Pressure},
  description={The measurement of force per unit area exerted by a gas or liquid.},
  units=\cfont{\gls{pascal}},
  kind={type,sample},
  facet={\gls{float}}
}


\newglossaryentry{processtimer sample}
{
  type=mtc,
  category=code,
  name={PROCESS\_TIMER},
  elementname=\cfont{ProcessTimer},
  description={The measurement of the amount of time a piece of equipment has performed different types of activities associated with the process being performed at that piece of equipment.},
  units=\cfont{\gls{second}},
  kind={type,sample},
  facet={\gls{float}}
}


\newglossaryentry{delay processtimer sample}
{
  type=mtc,
  category=code,
  name={DELAY},
  description={Measurement of the time that a process is waiting and unable to perform its intended function.},
  units=\cfont{\gls{second}},
  kind={subtype,sample},
  facet={\gls{float}}
}


\newglossaryentry{process processtimer sample}
{
  type=mtc,
  category=code,
  name={PROCESS},
  description={The measurement of the time from the beginning of production of a part or product on a piece of equipment until the time that production is complete for that part or product on that piece of equipment.  This includes the time that the piece of equipment is running, producing parts or products, or in the process of producing parts.},
  units=\cfont{\gls{second}},
  kind={subtype,sample},
  facet={\gls{float}}
}


\newglossaryentry{resistance sample}
{
  type=mtc,
  category=code,
  name={RESISTANCE},
  elementname=\cfont{Resistance},
  description={The measurement of the degree to which a substance opposes the passage of an electric current.},
  units=\cfont{\gls{ohm}},
  kind={type,sample},
  facet={\gls{float}}
}


\newglossaryentry{rotaryvelocity sample}
{
  type=mtc,
  category=code,
  name={ROTARY\_VELOCITY},
  elementname=\cfont{RotaryVelocity},
  description={The measurement of the rotational speed of a rotary axis.},
  units=\cfont{\gls{revolutionperminute}},
  kind={type,sample},
  facet={\gls{float}}
}


\newglossaryentry{actual rotaryvelocity sample}
{
  type=mtc,
  category=code,
  name={ACTUAL},
  description={The measured value of rotational speed that the rotary axis is spinning. },
  units=\cfont{\gls{revolutionperminute}},
  kind={subtype,sample},
  facet={\gls{float}}
}


\newglossaryentry{commanded rotaryvelocity sample}
{
  type=mtc,
  category=code,
  name={COMMANDED},
  description={The rotational speed as specified by the \gls{controller} type component.The COMMANDED velocity is a calculated value that includes adjustments and overrides.},
  units=\cfont{\gls{revolutionperminute}},
  kind={subtype,sample},
  facet={\gls{float}}
}


\newglossaryentry{override rotaryvelocity sample}
{
  type=mtc,
  category=code,
  name={OVERRIDE},
  description={The operators overridden value.  Percent of commanded. DEPRECATED in Version 1.3.   See \gls{event category} category \glspl{dataitem}.},
  units=\cfont{\gls{percent}},
  kind={subtype,sample},
  facet={\gls{float}}
}


\newglossaryentry{programmed rotaryvelocity sample}
{
  type=mtc,
  category=code,
  name={PROGRAMMED},
  description={The rotational velocity specified by a logic or motion program or set by a switch},
  units=\cfont{\gls{revolutionperminute}},
  kind={subtype,sample},
  facet={\gls{float}}
}


\newglossaryentry{soundlevel sample}
{
  type=mtc,
  category=code,
  name={SOUND\_LEVEL},
  elementname=\cfont{SoundLevel},
  description={The measurement of a sound level or sound pressure level relative to atmospheric pressure.},
  units=\cfont{\gls{decibel}},
  kind={type,sample},
  facet={\gls{float}}
}


\newglossaryentry{ascale soundlevel sample}
{
  type=mtc,
  category=code,
  name={A\_SCALE},
  description={A Scale weighting factor.   This is the default weighting factor if no factor is specified},
  units=\cfont{\gls{decibel}},
  kind={subtype,sample},
  facet={\gls{float}}
}


\newglossaryentry{bscale soundlevel sample}
{
  type=mtc,
  category=code,
  name={B\_SCALE},
  description={B Scale weighting factor},
  units=\cfont{\gls{decibel}},
  kind={subtype,sample},
  facet={\gls{float}}
}


\newglossaryentry{cscale soundlevel sample}
{
  type=mtc,
  category=code,
  name={C\_SCALE},
  description={C Scale weighting factor},
  units=\cfont{\gls{decibel}},
  kind={subtype,sample},
  facet={\gls{float}}
}


\newglossaryentry{dscale soundlevel sample}
{
  type=mtc,
  category=code,
  name={D\_SCALE},
  description={D Scale weighting factor},
  units=\cfont{\gls{decibel}},
  kind={subtype,sample},
  facet={\gls{float}}
}


\newglossaryentry{noscale soundlevel sample}
{
  type=mtc,
  category=code,
  name={NO\_SCALE},
  description={No weighting factor on the frequency scale},
  units=\cfont{\gls{decibel}},
  kind={subtype,sample},
  facet={\gls{float}}
}


\newglossaryentry{spindlespeed sample}
{
  type=mtc,
  category=code,
  name={\deprecated{SPINDLE\_SPEED}},
  elementname=\deprecated{\cfont{SpindleSpeed}},
  description={DEPRECATED in Version 1.2.  Replaced by \gls{rotaryvelocity sample}},
  units={},
  kind={type,sample},
  facet={\gls{float}}
}


\newglossaryentry{actual spindlespeed sample}
{
  type=mtc,
  category=code,
  name={ACTUAL},
  description={The rotational speed of a rotary axis.  ROTARY\_MODE \must be SPINDLE.},
  units=\cfont{\gls{revolutionperminute}},
  kind={subtype,sample},
  facet={\gls{float}}
}


\newglossaryentry{commanded spindlespeed sample}
{
  type=mtc,
  category=code,
  name={COMMANDED},
  description={The rotational speed the as specified by the \gls{controller} type Component.},
  units=\cfont{\gls{revolutionperminute}},
  kind={subtype,sample},
  facet={\gls{float}}
}


\newglossaryentry{override spindlespeed sample}
{
  type=mtc,
  category=code,
  name={OVERRIDE},
  description={The operators overridden value.  Percent of commanded. },
  units=\cfont{\gls{percent}},
  kind={subtype,sample},
  facet={\gls{float}}
}


\newglossaryentry{strain sample}
{
  type=mtc,
  category=code,
  name={STRAIN},
  elementname=\cfont{Strain},
  description={The measurement of the amount of deformation per unit length of an object when a load is applied.},
  units=\cfont{\gls{percent}},
  kind={type,sample},
  facet={\gls{float}}
}


\newglossaryentry{temperature sample}
{
  type=mtc,
  category=code,
  name={TEMPERATURE},
  elementname=\cfont{Temperature},
  description={The measurement of temperature.},
  units=\cfont{\gls{celsius}},
  kind={type,sample},
  facet={\gls{float}}
}


\newglossaryentry{tension sample}
{
  type=mtc,
  category=code,
  name={TENSION},
  elementname=\cfont{Tension},
  description={The measurement of a force that stretches or elongates an object.},
  units=\cfont{\gls{newton}},
  kind={type,sample},
  facet={\gls{float}}
}


\newglossaryentry{tilt sample}
{
  type=mtc,
  category=code,
  name={TILT},
  elementname=\cfont{Tilt},
  description={The measurement of angular displacement. },
  units=\cfont{\gls{microradian}},
  kind={type,sample},
  facet={\gls{float}}
}


\newglossaryentry{torque sample}
{
  type=mtc,
  category=code,
  name={TORQUE},
  elementname=\cfont{Torque},
  description={The measurement of the turning force exerted on an object or by an object.},
  units=\cfont{\gls{newtonmeter}},
  kind={type,sample},
  facet={\gls{float}}
}


\newglossaryentry{velocity sample}
{
  type=mtc,
  category=code,
  name={VELOCITY},
  elementname=\cfont{Velocity},
  description={The measurement of the rate of change of position of a \gls{component}.},
  units=\cfont{\gls{millimeterpersecond}},
  kind={type,sample},
  facet={\gls{float}}
}


\newglossaryentry{viscosity sample}
{
  type=mtc,
  category=code,
  name={VISCOSITY},
  elementname=\cfont{Viscosity},
  description={The measurement of a fluids resistance to flow.},
  units=\cfont{\gls{pascalsecond}},
  kind={type,sample},
  facet={\gls{float}}
}


\newglossaryentry{voltampere sample}
{
  type=mtc,
  category=code,
  name={VOLT\_AMPERE},
  elementname=\cfont{VoltAmpere},
  description={The measurement of the apparent power in an electrical circuit, equal to the product of root-mean-square (RMS) voltage and RMS current (commonly referred to as VA).},
  units=\cfont{\gls{voltampere}},
  kind={type,sample},
  facet={\gls{float}}
}


\newglossaryentry{voltamperereactive sample}
{
  type=mtc,
  category=code,
  name={VOLT\_AMPERE\_REACTIVE},
  elementname=\cfont{VoltAmpereReactive},
  description={The measurement of reactive power in an AC electrical circuit (commonly referred to as VAR).},
  units=\cfont{\gls{voltamperereactive}},
  kind={type,sample},
  facet={\gls{float}}
}


\newglossaryentry{voltage sample}
{
  type=mtc,
  category=code,
  name={VOLTAGE},
  elementname=\cfont{Voltage},
  description={The measurement of electrical potential between two points.},
  units=\cfont{\gls{volt}},
  kind={type,sample},
  facet={\gls{float}}
}


\newglossaryentry{actual voltage sample}
{
  type=mtc,
  category=code,
  name={ACTUAL},
  description={The measured voltage being delivered from a power source.},
  units=\cfont{\gls{volt}},
  kind={subtype,sample},
  facet={\gls{float}}
}


\newglossaryentry{alternating voltage sample}
{
  type=mtc,
  category=code,
  name={ALTERNATING},
  description={The measurement of alternating voltage.   If not specified further in statistic, defaults to RMS voltage  },
  units=\cfont{\gls{volt}},
  kind={subtype,sample},
  facet={\gls{float}}
}


\newglossaryentry{direct voltage sample}
{
  type=mtc,
  category=code,
  name={DIRECT},
  description={The measurement of DC voltage},
  units=\cfont{\gls{volt}},
  kind={subtype,sample},
  facet={\gls{float}}
}


\newglossaryentry{target voltage sample}
{
  type=mtc,
  category=code,
  name={TARGET},
  description={The desired or preset voltage to be delivered from a power source.},
  units=\cfont{\gls{volt}},
  kind={subtype,sample},
  facet={\gls{float}}
}


\newglossaryentry{wattage sample}
{
  type=mtc,
  category=code,
  name={WATTAGE},
  elementname=\cfont{Wattage},
  description={The measurement of power flowing through or dissipated by an electrical circuit or piece of equipment.},
  units=\cfont{\gls{watt}},
  kind={type,sample},
  facet={\gls{float}}
}


\newglossaryentry{actual wattage sample}
{
  type=mtc,
  category=code,
  name={ACTUAL},
  description={The measured wattage being delivered from a power source.},
  units=\cfont{\gls{watt}},
  kind={subtype,sample},
  facet={\gls{float}}
}


\newglossaryentry{target wattage sample}
{
  type=mtc,
  category=code,
  name={TARGET},
  description={The desired or preset wattage to be delivered from a power source.},
  units=\cfont{\gls{watt}},
  kind={subtype,sample},
  facet={\gls{float}}
}


\newglossaryentry{activeaxes event}
{
  type=mtc,
  category=code,
  name={ACTIVE\_AXES},
  elementname=\cfont{ActiveAxes},
  description={The set of axes currently associated with a \gls{path} or \gls{controller} \gls{structural element}.},
  kind={type,event},
  facet={\gls{arraystring}}
}


\newglossaryentry{actuatorstate event}
{
  type=mtc,
  category=code,
  name={ACTUATOR\_STATE},
  elementname=\cfont{ActuatorState},
  description={Represents the operational state of an apparatus for moving or controlling a mechanism or system.},
  kind={type,event},
  facet={\gls{string}},
  enumeration={\gls{active value},\gls{inactive value}}
}


\newglossaryentry{alarm event}
{
  type=mtc,
  category=code,
  name=\deprecated{ALARM},
  elementname=\deprecated{\cfont{Alarm}},
  description={DEPRECATED: Replaced with \gls{condition category} category data items in Version 1.1.0.},
  kind={type,event},
  facet={\gls{string}}
}


\newglossaryentry{availability event}
{
  type=mtc,
  category=code,
  name={AVAILABILITY},
  elementname=\cfont{Availability},
  description={Represents the \gls{agent}s ability to communicate with the data source.},
  kind={type,event},
  facet={\gls{string}},
  enumeration={\gls{available value},\gls{unavailable value}}
}


\newglossaryentry{axiscoupling event}
{
  type=mtc,
  category=code,
  name={AXIS\_COUPLING},
  elementname=\cfont{AxisCoupling},
  description={Describes the way the axes will be associated to each other. This is used in conjunction with \gls{coupledaxes event} to indicate the way they are interacting.},
  kind={type,event},
  facet={\gls{string}},
  enumeration={\gls{tandem value},\gls{synchronous value},\gls{master value},\gls{slave value}}
}


\newglossaryentry{axisfeedrateoverride event}
{
  type=mtc,
  category=code,
  name={AXIS\_FEEDRATE\_OVERRIDE},
  elementname=\cfont{AxisFeedrateOverride},
  description={The value of a signal or calculation issued to adjust the feedrate of an individual linear type axis.},
  kind={type,event},
  facet={\gls{float}}
}


\newglossaryentry{jog axisfeedrateoverride event}
{
  type=mtc,
  category=code,
  name={JOG},
  description={The value of a signal or calculation issued to adjust the feedrate of an individual linear type axis when that axis is being operated in a manual state or method (jogging).   \newline When the JOG subtype of AXIS\_FEEDRATE\_OVERRIDE is applied, the resulting commanded feedrate for the axis is limited to the value of the original JOG subtype of the AXIS\_FEEDRATE multiplied by the value of the JOG subtype of AXIS\_FEEDRATE\_OVERRIDE.},
  kind={subtype,event},
  facet={\gls{float}}
}


\newglossaryentry{programmed axisfeedrateoverride event}
{
  type=mtc,
  category=code,
  name={PROGRAMMED},
  description={The value of a signal or calculation issued to adjust the feedrate of an individual linear type axis that has been specified by a logic or motion program or set by a switch. \newline When the PROGRAMMED subtype of AXIS\_FEEDRATE\_OVERRIDE is applied, the resulting commanded feedrate for the axis is limited to the value of the original PROGRAMMED subtype of the AXIS\_FEEDRATE multiplied by the value of the PROGRAMMED subtype of AXIS\_FEEDRATE\_OVERRIDE.},
  kind={subtype,event},
  facet={\gls{float}}
}


\newglossaryentry{rapid axisfeedrateoverride event}
{
  type=mtc,
  category=code,
  name={RAPID},
  description={The value of a signal or calculation issued to adjust the feedrate of an individual linear type axis that is operating in a rapid positioning mode. \newline When the RAPID subtype of AXIS\_FEEDRATE\_OVERRIDE is applied, the resulting commanded feedrate for the axis is limited to the value of the original RAPID subtype of the AXIS\_FEEDRATE multiplied by the value of the RAPID subtype of AXIS\_FEEDRATE\_OVERRIDE.},
  kind={subtype,event},
  facet={\gls{float}}
}


\newglossaryentry{axisinterlock event}
{
  type=mtc,
  category=code,
  name={AXIS\_INTERLOCK},
  elementname=\cfont{AxisInterlock},
  description={An indicator of the state of the axis lockout function when power has been removed and the axis is allowed to move freely.},
  kind={type,event},
  facet={\gls{string}},
  enumeration={\gls{active value},\gls{inactive value}}
}


\newglossaryentry{axisstate event}
{
  type=mtc,
  category=code,
  name={AXIS\_STATE},
  elementname=\cfont{AxisState},
  description={An indicator of the controlled state of a \gls{linear} or \gls{rotary} component representing an axis.},
  kind={type,event},
  facet={\gls{string}},
  enumeration={\gls{home value},\gls{travel value},\gls{parked value},\gls{stopped value}}
}


\newglossaryentry{block event}
{
  type=mtc,
  category=code,
  name={BLOCK},
  elementname=\cfont{Block},
  description={The line of code or command being executed by a \gls{controller} \gls{structural element}.},
  kind={type,event},
  facet={\gls{string}}
}


\newglossaryentry{blockcount event}
{
  type=mtc,
  category=code,
  name={BLOCK\_COUNT},
  elementname=\cfont{BlockCount},
  description={The total count of the number of blocks of program code that have been executed since execution started.},
  kind={type,event},
  facet={\gls{integer}}
}


\newglossaryentry{chuckinterlock event}
{
  type=mtc,
  category=code,
  name={CHUCK\_INTERLOCK},
  elementname=\cfont{ChuckInterlock},
  description={An indication of the state of an interlock function or control logic state intended to prevent the associated \gls{chuck} component from being operated.},
  kind={type,event},
  facet={\gls{string}},
  enumeration={\gls{active value},\gls{inactive value}}
}


\newglossaryentry{manualunclamp chuckinterlock event}
{
  type=mtc,
  category=code,
  name={MANUAL\_UNCLAMP},
  description={An indication of the state of an operator controlled interlock that can inhibit the ability to initiate an unclamp action of an electronically controlled chuck.  The \gls{valid data value} \must be ACTIVE or INACTIVE. \newline When MANUAL\_UNCLAMP is ACTIVE, it is expected that a chuck cannot be unclamped until MANUAL\_UNCLAMP is set to INACTIVE. },
  kind={subtype,event},
  facet={\gls{string}},
  enumeration={\gls{active value},\gls{inactive value}}
}


\newglossaryentry{chuckstate event}
{
  type=mtc,
  category=code,
  name={CHUCK\_STATE},
  elementname=\cfont{ChuckState},
  description={An indication of the operating state of a mechanism that holds a part or stock material during a manufacturing process. It may also represent a mechanism that holds any other mechanism in place within a piece of equipment.},
  kind={type,event},
  facet={\gls{string}},
  enumeration={\gls{open value},\gls{closed value},\gls{unlatched value}}
}


\newglossaryentry{code event}
{
  type=mtc,
  category=code,
  name=\deprecated{CODE},
  elementname=\deprecated{\cfont{Code}},
  description={DEPRECATED in Version 1.1.},
  kind={type,event},
  facet={\gls{string}}
}


\newglossaryentry{compositionstate event}
{
  type=mtc,
  category=code,
  name={COMPOSITION\_STATE},
  elementname=\cfont{CompositionState},
  description={An indication of the operating condition of a mechanism represented by a \gls{composition} type element.},
  kind={type,event},
  facet={\gls{string}}
}


\newglossaryentry{action compositionstate event}
{
  type=mtc,
  category=code,
  name={ACTION},
  description={An indication of the operating state of a mechanism represented by a \gls{composition} type component.The operating state indicates whether the \gls{composition} element is activated or disabled. The \gls{valid data value} \must be ACTIVE or INACTIVE.},
  kind={subtype,event},
  facet={\gls{string}},
  enumeration={\gls{active value},\gls{inactive value}}
}


\newglossaryentry{lateral compositionstate event}
{
  type=mtc,
  category=code,
  name={LATERAL},
  description={An indication of the position of a mechanism that may move in a lateral direction.   The mechanism is represented by a \gls{composition} type component. \newline The position information indicates whether the \gls{composition} element is positioned to the right, to the left, or is in transition.  \newline The \gls{valid data value} \must be RIGHT, LEFT, or TRANSITIONING.},
  kind={subtype,event},
  facet={\gls{string}},
  enumeration={\gls{right value},\gls{left value},\gls{transitioning value}}
}


\newglossaryentry{motion compositionstate event}
{
  type=mtc,
  category=code,
  name={MOTION},
  description={An indication of the open or closed state of a mechanism.   The mechanism is represented by a \gls{composition} type component. \newline The operating state indicates whether the state of the \gls{composition} element is open, closed, or unlatched.   \newline The \gls{valid data value} \must be OPEN, UNLATCHED, or CLOSED.},
  kind={subtype,event},
  facet={\gls{string}},
  enumeration={\gls{open value},\gls{closed value},\gls{unlatched value}}
}


\newglossaryentry{switched compositionstate event}
{
  type=mtc,
  category=code,
  name={SWITCHED},
  description={An indication of the activation state of a mechanism represented by a \gls{composition} type component.The activation state indicates whether the \gls{composition} element is activated or not.The \gls{valid data value} \must be ON or OFF.},
  kind={subtype,event},
  facet={\gls{string}},
  enumeration={\gls{on value},\gls{off value}}
}


\newglossaryentry{vertical compositionstate event}
{
  type=mtc,
  category=code,
  name={VERTICAL},
  description={An indication of the position of a mechanism that may move in a vertical direction. The mechanism is represented by a \gls{composition} type component. \newline The position information indicates whether the \gls{composition} element is positioned to the top, to the bottom, or is in transition.  \newline The \gls{valid data value} \must be UP, DOWN, or TRANSITIONING.},
  kind={subtype,event},
  facet={\gls{string}},
  enumeration={\gls{up value},\gls{down value},\gls{transitioning value}}
}


\newglossaryentry{controllermode event}
{
  type=mtc,
  category=code,
  name={CONTROLLER\_MODE},
  elementname=\cfont{ControllerMode},
  description={The current operating mode of the \gls{controller} component.},
  kind={type,event},
  facet={\gls{string}},
  enumeration={\gls{automatic value},\gls{manual value},\gls{manualdatainput value},\gls{semiautomatic value},\gls{edit value}}
}


\newglossaryentry{controllermodeoverride event}
{
  type=mtc,
  category=code,
  name={CONTROLLER\_MODE\_OVERRIDE},
  elementname=\cfont{ControllerModeOverride},
  description={A setting or operator selection that changes the behavior of a piece of equipment.},
  kind={type,event},
  facet={\gls{string}},
  enumeration={\gls{on value},\gls{off value}}
}


\newglossaryentry{dryrun controllermodeoverride event}
{
  type=mtc,
  category=code,
  name={DRY\_RUN},
  description={A setting or operator selection used to execute a test mode to confirm the execution of machine functions.  The \gls{valid data value} \must be ON or OFF. \newline When DRY\_RUN is ON, the equipment performs all of its normal functions, except no part or product is produced.  If the equipment has a spindle, spindle operation is suspended.},
  kind={subtype,event},
  facet={\gls{string}},
  enumeration={\gls{on value},\gls{off value}}
}


\newglossaryentry{machineaxislock controllermodeoverride event}
{
  type=mtc,
  category=code,
  name={MACHINE\_AXIS\_LOCK},
  description={A setting or operator selection that changes the behavior of the controller on a piece of equipment.  The \gls{valid data value} \must be ON or OFF. \newline When MACHINE\_AXIS\_LOCK is ON, program execution continues normally, but no equipment motion occurs },
  kind={subtype,event},
  facet={\gls{string}},
  enumeration={\gls{on value},\gls{off value}}
}


\newglossaryentry{optionalstop controllermodeoverride event}
{
  type=mtc,
  category=code,
  name={OPTIONAL\_STOP},
  description={A setting or operator selection that changes the behavior of the controller on a piece of equipment.  The \gls{valid data value} \must be ON or OFF.The program execution is stopped after a specific program block is executed when OPTIONAL\_STOP is ON.    \newline In the case of a G-Code program, a program BLOCK containing a M01 code designates the command for an OPTIONAL\_STOP.EXECUTION \must change to OPTIONAL\_STOP after a program block specifying an optional stop is executed and the OPTIONAL\_STOP selection is ON.},
  kind={subtype,event},
  facet={\gls{string}},
  enumeration={\gls{on value},\gls{off value}}
}


\newglossaryentry{singleblock controllermodeoverride event}
{
  type=mtc,
  category=code,
  name={SINGLE\_BLOCK},
  description={A setting or operator selection that changes the behavior of the controller on a piece of equipment.  The \gls{valid data value} \must be ON or OFF. Program execution is paused after each BLOCK of code is executed when SINGLE\_BLOCK is ON.   \newline When SINGLE\_BLOCK is ON, EXECUTION \must change to INTERRUPTED after completion of each BLOCK of code. },
  kind={subtype,event},
  facet={\gls{string}},
  enumeration={\gls{on value},\gls{off value}}
}


\newglossaryentry{toolchangestop controllermodeoverride event}
{
  type=mtc,
  category=code,
  name={TOOL\_CHANGE\_STOP},
  description={A setting or operator selection that changes the behavior of the controller on a piece of equipment.  The \gls{valid data value} \must be ON or OFF. Program execution is paused when a command is executed requesting a cutting tool to be changed. EXECUTION \must change to INTERRUPTED after completion of the command requesting a cutting tool to be changed and TOOL\_CHANGE\_STOP is ON.},
  kind={subtype,event},
  facet={\gls{string}},
  enumeration={\gls{on value},\gls{off value}}
}


\newglossaryentry{coupledaxes event}
{
  type=mtc,
  category=code,
  name={COUPLED\_AXES},
  elementname=\cfont{CoupledAxes},
  description={Refers to the set of associated axes.},
  kind={type,event},
  facet={\gls{arraystring}}
}


\newglossaryentry{direction event}
{
  type=mtc,
  category=code,
  name={DIRECTION},
  elementname=\cfont{Direction},
  description={The direction of motion.},
  kind={type,event},
  facet={\gls{string}}
}


\newglossaryentry{linear direction event}
{
  type=mtc,
  category=code,
  name={LINEAR},
  description={The direction of motion of a linear motion.   The \gls{valid data value} \must be POSTIVE or NEGATIVE.},
  kind={subtype,event},
  facet={\gls{string}},
  enumeration={\gls{positive value},\gls{negative value}}
}


\newglossaryentry{rotary direction event}
{
  type=mtc,
  category=code,
  name={ROTARY},
  description={The rotational direction of a rotary motion using the right hand rule convention.The \gls{valid data value} \must be CLOCKWISE or COUNTER\_CLOCKWISE.},
  kind={subtype,event},
  facet={\gls{string}},
  enumeration={\gls{clockwise value},\gls{counterclockwise value}}
}


\newglossaryentry{doorstate event}
{
  type=mtc,
  category=code,
  name={DOOR\_STATE},
  elementname=\cfont{DoorState},
  description={The operational state of a \gls{door} type component or composition element.},
  kind={type,event},
  facet={\gls{string}},
  enumeration={\gls{closed value},\gls{closed value},\gls{unlatched value}}
}


\newglossaryentry{emergencystop event}
{
  type=mtc,
  category=code,
  name={EMERGENCY\_STOP},
  elementname=\cfont{EmergencyStop},
  description={The current state of the emergency stop signal for a piece of equipment, controller path, or any other component or subsystem of a piece of equipment.},
  kind={type,event},
  facet={\gls{string}},
  enumeration={\gls{armed value},\gls{triggered value}}
}


\newglossaryentry{endofbar event}
{
  type=mtc,
  category=code,
  name={END\_OF\_BAR},
  elementname=\cfont{EndOfBar},
  description={An indication of whether the end of a piece of bar stock being feed by a bar feeder has been reached.},
  kind={type,event},
  facet={\gls{string}},
  enumeration={\gls{yes value},\gls{no value}}
}


\newglossaryentry{auxiliary endofbar event}
{
  type=mtc,
  category=code,
  name={AUXILIARY},
  description={When multiple locations on a piece of bar stock are referenced as the indication for the END\_OF\_BAR, the additional location(s) \must be designated as AUXILIARY indication(s) for the END\_OF\_BAR.  },
  kind={subtype,event},
  facet={\gls{string}},
  enumeration={\gls{yes value},\gls{no value}}
}


\newglossaryentry{primary endofbar event}
{
  type=mtc,
  category=code,
  name={PRIMARY},
  description={Specific applications MAY reference one or more locations on a piece of bar stock as the indication for the END\_OF\_BAR.  The main or most important location \must be designated as the PRIMARY indication for the END\_OF\_BAR.   \newline If no \gls{subtype} is specified, PRIMARY \must be the default END\_OF\_BAR indication.},
  kind={subtype,event},
  facet={\gls{string}},
  enumeration={\gls{yes value},\gls{no value}}
}


\newglossaryentry{equipmentmode event}
{
  type=mtc,
  category=code,
  name={EQUIPMENT\_MODE},
  elementname=\cfont{EquipmentMode},
  description={An indication that a piece of equipment, or a sub-part of a piece of equipment, is performing specific types of activities.},
  kind={type,event},
  facet={\gls{string}},
  enumeration={\gls{on value},\gls{off value}}
}


\newglossaryentry{delay equipmentmode event}
{
  type=mtc,
  category=code,
  name={DELAY},
  description={An indication that a piece of equipment is waiting for an event or an action to occur.},
  kind={subtype,event},
  facet={\gls{string}},
  enumeration={\gls{on value},\gls{off value}}
}


\newglossaryentry{loaded equipmentmode event}
{
  type=mtc,
  category=code,
  name={LOADED},
  description={An indication that the sub-parts of a piece of equipment are under load. \newline Example: For traditional machine tools, this is an indication that the cutting tool is assumed to be engaged with the part. The \gls{valid data value} \must be ON or OFF.},
  kind={subtype,event},
  facet={\gls{string}},
  enumeration={\gls{on value},\gls{off value}}
}


\newglossaryentry{operating equipmentmode event}
{
  type=mtc,
  category=code,
  name={OPERATING},
  description={An indication that the major sub-parts of a piece of equipment are powered or performing any activity whether producing a part or product or not.   \newline Example: For traditional machine tools, this includes when the piece of equipment is \gls{working equipmentmode event} or it is idle.The \gls{valid data value} \must be ON or OFF.},
  kind={subtype,event},
  facet={\gls{string}},
  enumeration={\gls{on value},\gls{off value}}
}


\newglossaryentry{powered equipmentmode event}
{
  type=mtc,
  category=code,
  name={POWERED},
  description={An indication that primary power is applied to the piece of equipment and, as a minimum, the controller or logic portion of the piece of equipment is powered and functioning or components that are required to remain on are powered. Example: Heaters for an extrusion machine that required to be powered even when the equipment is turned off.The \gls{valid data value} \must be ON or OFF.},
  kind={subtype,event},
  facet={\gls{string}},
  enumeration={\gls{on value},\gls{off value}}
}


\newglossaryentry{working equipmentmode event}
{
  type=mtc,
  category=code,
  name={WORKING},
  description={An indication that a piece of equipment is performing any activity  the equipment is active and performing a function under load or not. \newline Example: For traditional machine tools, this includes when the piece of equipment is LOADED, making rapid moves, executing a tool change, etc.The \gls{valid data value} \must be ON or OFF.},
  kind={subtype,event},
  facet={\gls{string}},
  enumeration={\gls{on value},\gls{off value}}
}


\newglossaryentry{execution event}
{
  type=mtc,
  category=code,
  name={EXECUTION},
  elementname=\cfont{Execution},
  description={The execution status of the \gls{controller}.},
  kind={type,event},
  facet={\gls{string}},
  enumeration={\gls{ready value},\gls{active value},\gls{interrupted value},\gls{feedhold value},\gls{stopped value},\gls{optionalstop value},\gls{programstopped value},\gls{programcompleted value}}
}


\newglossaryentry{functionalmode event}
{
  type=mtc,
  category=code,
  name={FUNCTIONAL\_MODE},
  elementname=\cfont{FunctionalMode},
  description={The current intended production status of the device or component.},
  kind={type,event},
  facet={\gls{string}},
  enumeration={\gls{production value},\gls{setup value},\gls{teardown value},\gls{maintenance value},\gls{processdevelopment value}}
}


\newglossaryentry{hardness event}
{
  type=mtc,
  category=code,
  name={HARDNESS},
  elementname=\cfont{Hardness},
  description={The measurement of the hardness of a material.},
  kind={type,event},
  facet={\gls{float}}
}


\newglossaryentry{brinell hardness event}
{
  type=mtc,
  category=code,
  name={BRINELL},
  description={A scale to measure the resistance to deformation of a surface.},
  kind={subtype,event},
  facet={\gls{float}}
}


\newglossaryentry{leeb hardness event}
{
  type=mtc,
  category=code,
  name={LEEB},
  description={A scale to measure the elasticity of a surface.},
  kind={subtype,event},
  facet={\gls{float}}
}


\newglossaryentry{mohs hardness event}
{
  type=mtc,
  category=code,
  name={MOHS},
  description={A scale to measure the resistance to scratching of a surface.},
  kind={subtype,event},
  facet={\gls{float}}
}


\newglossaryentry{rockwell hardness event}
{
  type=mtc,
  category=code,
  name={ROCKWELL},
  description={A scale to measure the resistance to deformation of a surface.},
  kind={subtype,event},
  facet={\gls{float}}
}


\newglossaryentry{shore hardness event}
{
  type=mtc,
  category=code,
  name={SHORE},
  description={A scale to measure the resistance to deformation of a surface.},
  kind={subtype,event},
  facet={\gls{float}}
}


\newglossaryentry{vickers hardness event}
{
  type=mtc,
  category=code,
  name={VICKERS},
  description={A scale to measure the resistance to deformation of a surface.},
  kind={subtype,event},
  facet={\gls{float}}
}


\newglossaryentry{interfacestate event}
{
  type=mtc,
  category=code,
  name={INTERFACE\_STATE},
  elementname=\cfont{InterfaceState},
  description={The current functional or operational state of an \gls{interface} type element indicating whether the interface is active or is not currently functioning.},
  kind={type,event},
  facet={\gls{string}},
  enumeration={\gls{enabled value},\gls{disabled value}}
}


\newglossaryentry{line event}
{
  type=mtc,
  category=code,
  name=\deprecated{LINE},
  elementname=\deprecated{\cfont{Line}},
  description={DEPRECATED in Version 1.4.0.},
  kind={type,event},
  facet={\gls{float}}
}


\newglossaryentry{maximum line event}
{
  type=mtc,
  category=code,
  name={MAXIMUM},
  description={The maximum line number of the code being executed.},
  kind={subtype,event},
  facet={\gls{float}}
}


\newglossaryentry{minimum line event}
{
  type=mtc,
  category=code,
  name={MINIMUM},
  description={The minimum line number of the code being executed.},
  kind={subtype,event},
  facet={\gls{float}}
}


\newglossaryentry{linelabel event}
{
  type=mtc,
  category=code,
  name={LINE\_LABEL},
  elementname=\cfont{LineLabel},
  description={An optional identifier for a \gls{block event} of code in a \gls{program event}.},
  kind={type,event},
  facet={\gls{string}}
}


\newglossaryentry{linenumber event}
{
  type=mtc,
  category=code,
  name={LINE\_NUMBER},
  elementname=\cfont{LineNumber},
  description={A reference to the position of a block of program code within a control program.},
  kind={type,event},
  facet={\gls{integer}}
}


\newglossaryentry{absolute linenumber event}
{
  type=mtc,
  category=code,
  name={ABSOLUTE},
  description={The position of a block of program code relative to the beginning of the control program.},
  kind={subtype,event},
  facet={\gls{integer}}
}


\newglossaryentry{incremental linenumber event}
{
  type=mtc,
  category=code,
  name={INCREMENTAL},
  description={The position of a block of program code relative to the occurrence of the last LINE\_LABEL encountered in the control program.},
  kind={subtype,event},
  facet={\gls{integer}}
}


\newglossaryentry{material event}
{
  type=mtc,
  category=code,
  name={MATERIAL},
  elementname=\cfont{Material},
  description={The identifier of a material used or consumed in the manufacturing process.},
  kind={type,event},
  facet={\gls{string}}
}


\newglossaryentry{message event}
{
  type=mtc,
  category=code,
  name={MESSAGE},
  elementname=\cfont{Message},
  representation=\cfont{MessageDiscrete},
  description={Any text string of information to be transferred from a piece of equipment to a client software application.},
  kind={type,event},
  facet={\gls{string}}
}


\newglossaryentry{operatorid event}
{
  type=mtc,
  category=code,
  name={OPERATOR\_ID},
  elementname=\cfont{OperatorId},
  description={The identifier of the person currently responsible for operating the piece of equipment.},
  kind={type,event},
  facet={\gls{string}}
}


\newglossaryentry{palletid event}
{
  type=mtc,
  category=code,
  name={PALLET\_ID},
  elementname=\cfont{PalletId},
  description={The identifier for a pallet.},
  kind={type,event},
  facet={\gls{string}}
}


\newglossaryentry{partcount event}
{
  type=mtc,
  category=code,
  name={PART\_COUNT},
  elementname=\cfont{PartCount},
  representation=\cfont{PartCountDiscrete},
  description={The current count of parts produced as represented by the \gls{controller} component.},
  kind={type,event},
  facet={\gls{float}}
}


\newglossaryentry{all partcount event}
{
  type=mtc,
  category=code,
  name={ALL},
  description={The count of all the parts produced.  If the subtype is not given, this is the default.},
  kind={subtype,event},
  facet={\gls{float}}
}


\newglossaryentry{bad partcount event}
{
  type=mtc,
  category=code,
  name={BAD},
  description={Indicates the count of incorrect parts produced.},
  kind={subtype,event},
  facet={\gls{float}}
}


\newglossaryentry{good partcount event}
{
  type=mtc,
  category=code,
  name={GOOD},
  description={Indicates the count of correct parts made.},
  kind={subtype,event},
  facet={\gls{float}}
}


\newglossaryentry{remaining partcount event}
{
  type=mtc,
  category=code,
  name={REMAINING},
  description={The number of parts remaining in stock or to be produced.},
  facet={\gls{float}}
}


\newglossaryentry{target partcount event}
{
  type=mtc,
  category=code,
  name={TARGET},
  description={Indicates the number of parts that are projected or planned to be produced.},
  kind={subtype,event},
  facet={\gls{float}}
}


\newglossaryentry{partid event}
{
  type=mtc,
  category=code,
  name={PART\_ID},
  elementname=\cfont{PartId},
  description={An identifier of a part in a manufacturing operation.},
  kind={type,event},
  facet={\gls{string}}
}


\newglossaryentry{partnumber event}
{
  type=mtc,
  category=code,
  name={PART\_NUMBER},
  elementname=\cfont{PartNumber},
  description={An identifier of a part or product moving through the manufacturing process.  The \gls{valid data value} \must be a text string. },
  kind={type,event},
  facet={\gls{string}}
}


\newglossaryentry{pathfeedrateoverride event}
{
  type=mtc,
  category=code,
  name={PATH\_FEEDRATE\_OVERRIDE},
  elementname=\cfont{PathFeedrateOverride},
  description={The value of a signal or calculation issued to adjust the feedrate for the axes associated with a Path component that may represent a single axis or the coordinated movement of multiple axes.},
  kind={type,event},
  facet={\gls{float}}
}


\newglossaryentry{jog pathfeedrateoverride event}
{
  type=mtc,
  category=code,
  name={JOG},
  description={The value of a signal or calculation issued to adjust the feedrate of the axes associated with a Path component when the axes, or a single axis, are being operated in a manual mode or method (jogging).   \newline When the JOG subtype of PATH\_FEEDRATE\_OVERRIDE is applied, the resulting commanded feedrate for the axes, or a single axis, associated with the path are limited to the value of the original JOG subtype of the PATH\_FEEDRATE multiplied by the value of the JOG subtype of PATH\_FEEDRATE\_OVERRIDE.},
  kind={subtype,event},
  facet={\gls{float}}
}


\newglossaryentry{programmed pathfeedrateoverride event}
{
  type=mtc,
  category=code,
  name={PROGRAMMED},
  description={The value of a signal or calculation issued to adjust the feedrate of the axes associated with a Path component when the axes, or a single axis, are operating as specified by a logic or motion program or set by a switch. \newline When the PROGRAMMED subtype of PATH\_FEEDRATE\_OVERRIDE is applied, the resulting commanded feedrate for the axes, or a single axis, associated with the path are limited to the value of the original PROGRAMMED subtype of the PATH\_FEEDRATE multiplied by the value of the PROGRAMMED subtype of PATH\_FEEDRATE\_OVERRIDE.},
  kind={subtype,event},
  facet={\gls{float}}
}


\newglossaryentry{rapid pathfeedrateoverride event}
{
  type=mtc,
  category=code,
  name={RAPID},
  description={The value of a signal or calculation issued to adjust the feedrate of the axes associated with a Path component when the axes, or a single axis, are being operated in a rapid positioning mode or method (rapid).   \newline When the RAPID subtype of PATH\_FEEDRATE\_OVERRIDE is applied, the resulting commanded feedrate for the axes, or a single axis, associated with the path are limited to the value of the original RAPID subtype of the PATH\_FEEDRATE multiplied by the value of the RAPID subtype of PATH\_FEEDRATE\_OVERRIDE.},
  kind={subtype,event},
  facet={\gls{float}}
}


\newglossaryentry{pathmode event}
{
  type=mtc,
  category=code,
  name={PATH\_MODE},
  elementname=\cfont{PathMode},
  description={Describes the operational relationship between a \gls{path} \gls{structural element} and another \gls{path} \gls{structural element} for pieces of equipment comprised of multiple logical groupings of controlled axes or other logical operations.},
  kind={type,event},
  facet={\gls{string}},
  enumeration={\gls{independent value},\gls{master value},\gls{synchronous value},\gls{mirror value}}
}


\newglossaryentry{powerstate event}
{
  type=mtc,
  category=code,
  name={POWER\_STATE},
  elementname=\cfont{PowerState},
  description={The indication of the status of the source of energy for a \gls{structural element} to allow it to perform its intended function or the state of an enabling signal providing permission for the \gls{structural element} to perform its functions.},
  kind={type,event},
  facet={\gls{string}},
  enumeration={\gls{on value},\gls{off value}}
}


\newglossaryentry{control powerstate event}
{
  type=mtc,
  category=code,
  name={CONTROL},
  description={The state of the enabling signal or control logic that enables or disables the function or operation of the \gls{structural element}.},
  kind={subtype,event},
  facet={\gls{string}},
  enumeration={\gls{on value},\gls{off value}}
}


\newglossaryentry{line powerstate event}
{
  type=mtc,
  category=code,
  name={LINE},
  description={The state of the power source for the \gls{structural element}.},
  kind={subtype,event},
  facet={\gls{string}},
  enumeration={\gls{on value},\gls{off value}}
}


\newglossaryentry{powerstatus event}
{
  type=mtc,
  category=code,
  name=\deprecated{POWER\_STATUS},
  elementname=\deprecated{\cfont{PowerStatus}},
  description={DEPRECATED in Version 1.1.0.},
  kind={type,event},
  facet={\gls{string}}
}


\newglossaryentry{program event}
{
  type=mtc,
  category=code,
  name={PROGRAM},
  elementname=\cfont{Program},
  description={The name of the logic or motion program being executed by the \gls{controller} component.},
  kind={type,event},
  facet={\gls{string}}
}


\newglossaryentry{programcomment event}
{
  type=mtc,
  category=code,
  name={PROGRAM\_COMMENT},
  elementname=\cfont{ProgramComment},
  description={A comment or non-executable statement in the control program.The \gls{valid data value} \must be a text string.},
  kind={type,event},
  facet={\gls{string}}
}


\newglossaryentry{programedit event}
{
  type=mtc,
  category=code,
  name={PROGRAM\_EDIT},
  elementname=\cfont{ProgramEdit},
  description={An indication of the status of the \gls{controller} components program editing mode. \newline On many controls, a program can be edited while another program is currently being executed.},
  kind={type,event},
  facet={\gls{string}},
  enumeration={\gls{active value},\gls{ready value},\gls{notready value}}
}


\newglossaryentry{programeditname event}
{
  type=mtc,
  category=code,
  name={PROGRAM\_EDIT\_NAME},
  elementname=\cfont{ProgramEditName},
  description={The name of the program being edited. \newline This is used in conjunction with \gls{programedit event} when in \gls{active value} state. \newline The \gls{valid data value} \must be a text string.},
  kind={type,event},
  facet={\gls{string}}
}


\newglossaryentry{programheader event}
{
  type=mtc,
  category=code,
  name={PROGRAM\_HEADER},
  elementname=\cfont{ProgramHeader},
  description={The non-executable header section of the control program.},
  kind={type,event},
  facet={\gls{string}}
}


\newglossaryentry{rotarymode event}
{
  type=mtc,
  category=code,
  name={ROTARY\_MODE},
  elementname=\cfont{RotaryMode},
  description={The current operating mode for a Rotary type axis.},
  kind={type,event},
  facet={\gls{string}},
  enumeration={\gls{spindle value},\gls{index value},\gls{contour value}}
}


\newglossaryentry{rotaryvelocityoverride event}
{
  type=mtc,
  category=code,
  name={ROTARY\_VELOCITY\_OVERRIDE},
  elementname=\cfont{RotaryVelocityOverride},
  description={The value of a command issued to adjust the programmed velocity for a \gls{rotary} type axis.This command represents a percentage change to the velocity calculated by a logic or motion program or set by a switch for a \gls{rotary} type axis.},
  kind={type,event},
  facet={\gls{float}}
}


\newglossaryentry{serialnumber event}
{
  type=mtc,
  category=code,
  name={SERIAL\_NUMBER},
  elementname=\cfont{SerialNumber},
  description={The serial number associated with a \gls{component}, \gls{asset}, or \gls{device}. The \gls{valid data value} \must be a text string.},
  kind={type,event},
  facet={\gls{string}}
}


\newglossaryentry{spindleinterlock event}
{
  type=mtc,
  category=code,
  name={SPINDLE\_INTERLOCK},
  elementname=\cfont{SpindleInterlock},
  description={An indication of the status of the spindle for a piece of equipment when power has been removed and it is free to rotate.},
  kind={type,event},
  facet={\gls{string}},
  enumeration={\gls{active value},\gls{inactive value}}
}


\newglossaryentry{toolassetid event}
{
  type=mtc,
  category=code,
  name={TOOL\_ASSET\_ID},
  elementname=\cfont{ToolAssetId},
  description={The identifier of an individual tool asset.The \gls{valid data value} \must be a text string.},
  kind={type,event},
  facet={\gls{string}}
}


\newglossaryentry{toolid event}
{
  type=mtc,
  category=code,
  name={TOOL\_ID},
  elementname=\cfont{ToolId},
  description={DEPRECATED in Version 1.2.0.   See \gls{toolassetid event}. \deprecated{The identifier of the tool currently in use for a given Path.}},
  kind={type,event},
  facet={\gls{string}}
}


\newglossaryentry{toolnumber event}
{
  type=mtc,
  category=code,
  name={TOOL\_NUMBER},
  elementname=\cfont{ToolNumber},
  description={The identifier assigned by the \gls{controller} component to a cutting tool when in use by a piece of equipment. \newline The \gls{valid data value} \must be a text string.},
  kind={type,event},
  facet={\gls{string}}
}


\newglossaryentry{tooloffset event}
{
  type=mtc,
  category=code,
  name={TOOL\_OFFSET},
  elementname=\cfont{ToolOffset},
  description={A reference to the tool offset variables applied to the active cutting tool associated with a \gls{path} in a \gls{controller} type component.},
  kind={type,event},
  facet={\gls{float}}
}


\newglossaryentry{length tooloffset event}
{
  type=mtc,
  category=code,
  name={LENGTH},
  description={A reference to a length type tool offset variable.},
  kind={subtype,event},
  facet={\gls{float}}
}


\newglossaryentry{radial tooloffset event}
{
  type=mtc,
  category=code,
  name={RADIAL},
  description={A reference to a radial type tool offset variable.},
  kind={subtype,event},
  facet={\gls{float}}
}


\newglossaryentry{user event}
{
  type=mtc,
  category=code,
  name={USER},
  elementname=\cfont{User},
  description={The identifier of the person currently responsible for operating the piece of equipment.},
  kind={type,event},
  facet={\gls{string}}
}


\newglossaryentry{maintenance user event}
{
  type=mtc,
  category=code,
  name={MAINTENANCE},
  description={The identifier of the person currently responsible for performing maintenance on the piece of equipment.},
  kind={subtype,event},
  facet={\gls{string}}
}


\newglossaryentry{operator user event}
{
  type=mtc,
  category=code,
  name={OPERATOR},
  description={The identifier of the person currently responsible for operating the piece of equipment.},
  kind={subtype,event},
  facet={\gls{string}}
}


\newglossaryentry{setup user event}
{
  type=mtc,
  category=code,
  name={SET\_UP},
  description={The identifier of the person currently responsible for preparing a piece of equipment for production or restoring the piece of equipment to a neutral state after production.},
  kind={subtype,event},
  facet={\gls{string}}
}


\newglossaryentry{wire event}
{
  type=mtc,
  category=code,
  name={WIRE},
  elementname=\cfont{Wire},
  description={The identifier for the type of wire used as the cutting mechanism in Electrical Discharge Machining or similar processes. \newline The \gls{valid data value} \must be a text string.},
  kind={type,event},
  facet={\gls{string}}
}


\newglossaryentry{workoffset event}
{
  type=mtc,
  category=code,
  name={WORK\_OFFSET},
  elementname=\cfont{WorkOffset},
  description={A reference to the offset variables for a work piece or part associated with a \gls{path} in a \gls{controller} type component.},
  kind={type,event},
  facet={\gls{float}}
}


\newglossaryentry{workholdingid event}
{
  type=mtc,
  category=code,
  name={WORKHOLDING\_ID},
  elementname=\cfont{WorkholdingId},
  description={The identifier for the current workholding or part clamp in use by a piece of equipment. \newline The \gls{valid data value} \must be a text string.},
  kind={type,event},
  facet={\gls{string}}
}


\newglossaryentry{actuator condition}
{
  type=mtc,
  category=code,
  name={ACTUATOR},
  elementname=\cfont{Actuator},
  description={An indication of a fault associated with an actuator.},
  kind={type,condition}
}


\newglossaryentry{chuckinterlock condition}
{
  type=mtc,
  category=code,
  name={CHUCK\_INTERLOCK},
  elementname=\cfont{ChuckInterlock},
  description={An indication of the operational condition of the interlock function for an electronically controller chuck.},
  kind={type,condition}
}


\newglossaryentry{communications condition}
{
  type=mtc,
  category=code,
  name={COMMUNICATIONS},
  elementname=\cfont{Communications},
  description={An indication that the piece of equipment has experienced a communications failure.},
  kind={type,condition}
}


\newglossaryentry{datarange condition}
{
  type=mtc,
  category=code,
  name={DATA\_RANGE},
  elementname=\cfont{DataRange},
  description={An indication that the value of the data associated with a measured value or a calculation is outside of an expected range.},
  kind={type,condition}
}


\newglossaryentry{direction condition}
{
  type=mtc,
  category=code,
  name={DIRECTION},
  elementname=\cfont{Direction},
  description={An indication of a fault associated with the direction of motion of a \gls{structural element}.},
  kind={type,condition}
}


\newglossaryentry{endofbar condition}
{
  type=mtc,
  category=code,
  name={END\_OF\_BAR},
  elementname=\cfont{EndOfBar},
  description={An indication that the end of a piece of bar stock has been reached.},
  kind={type,condition}
}


\newglossaryentry{hardware condition}
{
  type=mtc,
  category=code,
  name={HARDWARE},
  elementname=\cfont{Hardware},
  description={An indication of a fault associated with the hardware subsystem of the \gls{structural element}.},
  kind={type,condition}
}


\newglossaryentry{interfacestate condition}
{
  type=mtc,
  category=code,
  name={INTERFACE\_STATE},
  elementname=\cfont{InterfaceState},
  description={An indication of the operation condition of an \gls{interface component} component.},
  kind={type,condition}
}


\newglossaryentry{logicprogram condition}
{
  type=mtc,
  category=code,
  name={LOGIC\_PROGRAM},
  elementname=\cfont{LogicProgram},
  description={An indication that an error occurred in the logic program or programmable logic controller (PLC) associated with a piece of equipment.},
  kind={type,condition}
}


\newglossaryentry{motionprogram condition}
{
  type=mtc,
  category=code,
  name={MOTION\_PROGRAM},
  elementname=\cfont{MotionProgram},
  description={An indication that an error occurred in the motion program associated with a piece of equipment.},
  kind={type,condition}
}


\newglossaryentry{system condition}
{
  type=mtc,
  category=code,
  name={SYSTEM},
  elementname=\cfont{System},
  description={},
  kind={type,condition}
}


\newglossaryentry{event category}
{
  type=mtc,
  category=code,
  name={EVENT},
  description={}
}


\newglossaryentry{condition category}
{
  type=mtc,
  category=code,
  name={CONDITION},
  description={}
}


\newglossaryentry{compositionid}
{
  type=mtc,
  category=code,
  name={compositionId},
  kind={attribute},
  description={}
}


\newglossaryentry{actioncomplete}
{
  type=mtc,
  category=code,
  name={ACTION\_COMPLETE},
  kind={resettrigger},
  description={The value of the \gls{data entity} that is measuring an action or operation is to be reset upon completion of that action or operation.}
}


\newglossaryentry{annual}
{
  type=mtc,
  category=code,
  name={ANNUAL},
  kind={resettrigger},
  description={The value of the \gls{data entity} is to be reset at the end of a 12-month period.}
}


\newglossaryentry{day}
{
  type=mtc,
  category=code,
  name={DAY},
  kind={resettrigger},
  description={The value of the \gls{data entity} is to be reset at the end of a 24-hour period.}
}


\newglossaryentry{life}
{
  type=mtc,
  category=code,
  name={LIFE},
  kind={resettrigger},
  description={The value of the data item is not reset and accumulates for the entire life of the piece of equipment.}
}


\newglossaryentry{maintenance}
{
  type=mtc,
  category=code,
  name={MAINTENANCE},
  kind={resettrigger},
  description={The value of the data item is to be reset upon completion of a maintenance event.}
}


\newglossaryentry{month}
{
  type=mtc,
  category=code,
  name={MONTH},
  kind={resettrigger},
  description={The value of the \gls{data entity} is to be reset at the end of a monthly period.}
}


\newglossaryentry{poweron}
{
  type=mtc,
  category=code,
  name={POWER\_ON},
  kind={resettrigger},
  description={The value of the \gls{data entity} is to be reset when power was applied to the piece of equipment after a planned or unplanned interruption of power has occurred.}
}


\newglossaryentry{shift}
{
  type=mtc,
  category=code,
  name={SHIFT},
  kind={resettrigger},
  description={The value of the \gls{data entity} is to be reset at the end of a work shift.}
}


\newglossaryentry{week}
{
  type=mtc,
  category=code,
  name={WEEK},
  kind={resettrigger},
  description={The value of the \gls{data entity} is to be reset at the end of a 7-day period.}
}


\newglossaryentry{warning}
{
  type=mtc,
  category=code,
  name={Warning},
  description={}
}


\newglossaryentry{normal}
{
  type=mtc,
  category=code,
  name={Normal},
  description={}
}


\newglossaryentry{fault}
{
  type=mtc,
  category=code,
  name={Fault},
  description={}
}


\newglossaryentry{low}
{
  type=mtc,
  category=code,
  name={LOW},
  description={}
}


\newglossaryentry{high}
{
  type=mtc,
  category=code,
  name={HIGH},
  description={}
}


\newglossaryentry{automatic}
{
  type=mtc,
  category=code,
  name={AUTOMATIC},
  description={}
}


\newglossaryentry{ampere}
{
  type=mtc,
  category=code,
  name={AMPERE},
  description={Amps},
  kind={units}
}


\newglossaryentry{celsius}
{
  type=mtc,
  category=code,
  name={CELSIUS},
  description={Degrees Celsius},
  kind={units}
}


\newglossaryentry{count}
{
  type=mtc,
  category=code,
  name={COUNT},
  description={A counted event},
  kind={units}
}


\newglossaryentry{decibel}
{
  type=mtc,
  category=code,
  name={DECIBEL},
  description={Sound Level},
  kind={units}
}


\newglossaryentry{degree}
{
  type=mtc,
  category=code,
  name={DEGREE},
  description={Angle in degrees},
  kind={units}
}


\newglossaryentry{degreepersecond}
{
  type=mtc,
  category=code,
  name={DEGREE/SECOND},
  description={Angular degrees per second},
  kind={units}
}


\newglossaryentry{degreepersecondsquared}
{
  type=mtc,
  category=code,
  name={DEGREE/SECOND$^2$},
  description={Angular acceleration in degrees per second squared},
  kind={units}
}


\newglossaryentry{hertz}
{
  type=mtc,
  category=code,
  name={HERTZ},
  description={Frequency measured in cycles per second},
  kind={units}
}


\newglossaryentry{joule}
{
  type=mtc,
  category=code,
  name={JOULE},
  description={A measurement of energy.},
  kind={units}
}


\newglossaryentry{kilogram}
{
  type=mtc,
  category=code,
  name={KILOGRAM},
  description={Kilograms},
  kind={units}
}


\newglossaryentry{liter}
{
  type=mtc,
  category=code,
  name={LITER},
  description={Liters},
  kind={units}
}


\newglossaryentry{literpersecond}
{
  type=mtc,
  category=code,
  name={LITER/SECOND},
  description={Liters per second},
  kind={units}
}


\newglossaryentry{microradian}
{
  type=mtc,
  category=code,
  name={MICRO\_RADIAN},
  description={Measurement of Tilt},
  kind={units}
}


\newglossaryentry{millimeter}
{
  type=mtc,
  category=code,
  name={MILLIMETER},
  description={Millimeters},
  kind={units}
}


\newglossaryentry{millimeterpersecond}
{
  type=mtc,
  category=code,
  name={MILLIMETER/SECOND},
  description={Millimeters per second},
  kind={units}
}


\newglossaryentry{millimeterpersecondsquared}
{
  type=mtc,
  category=code,
  name={MILLIMETER/SECOND$^2$},
  description={Acceleration in millimeters per second squared},
  kind={units}
}


\newglossaryentry{millimeter3d}
{
  type=mtc,
  category=code,
  name={MILLIMETER\_3D},
  description={A point in space identified by X, Y, and Z positions and represented by a space-delimited set of numbers each expressed in millimeters.},
  kind={units}
}


\newglossaryentry{newton}
{
  type=mtc,
  category=code,
  name={NEWTON},
  description={Force in Newtons},
  kind={units}
}


\newglossaryentry{newtonmeter}
{
  type=mtc,
  category=code,
  name={NEWTON\_METER},
  description={Torque, a unit for force times distance.},
  kind={units}
}


\newglossaryentry{ohm}
{
  type=mtc,
  category=code,
  name={OHM},
  description={Measure of Electrical Resistance},
  kind={units}
}


\newglossaryentry{pascal}
{
  type=mtc,
  category=code,
  name={PASCAL},
  description={Pressure in Newtons per square meter},
  kind={units}
}


\newglossaryentry{pascalsecond}
{
  type=mtc,
  category=code,
  name={PASCAL\_SECOND},
  description={Measurement of Viscosity},
  kind={units}
}


\newglossaryentry{percent}
{
  type=mtc,
  category=code,
  name={PERCENT},
  description={Percentage},
  kind={units}
}


\newglossaryentry{ph}
{
  type=mtc,
  category=code,
  name={PH},
  description={A measure of the acidity or alkalinity of a solution},
  kind={units}
}


\newglossaryentry{revolutionperminute}
{
  type=mtc,
  category=code,
  name={REVOLUTION/MINUTE},
  description={Revolutions per minute},
  kind={units}
}


\newglossaryentry{second}
{
  type=mtc,
  category=code,
  name={SECOND},
  description={A measurement of time.},
  kind={units}
}


\newglossaryentry{siemenspermeter}
{
  type=mtc,
  category=code,
  name={SIEMENS/METER},
  description={A measurement of Electrical Conductivity},
  kind={units}
}


\newglossaryentry{volt}
{
  type=mtc,
  category=code,
  name={VOLT},
  description={Volts},
  kind={units}
}


\newglossaryentry{voltampere}
{
  type=mtc,
  category=code,
  name={VOLT\_AMPERE},
  description={Volt-Ampere  (VA)},
  kind={units}
}


\newglossaryentry{voltamperereactive}
{
  type=mtc,
  category=code,
  name={VOLT\_AMPERE\_REACTIVE},
  description={Volt-Ampere Reactive  (VAR)},
  kind={units}
}


\newglossaryentry{watt}
{
  type=mtc,
  category=code,
  name={WATT},
  description={Watts},
  kind={units}
}


\newglossaryentry{wattsecond}
{
  type=mtc,
  category=code,
  name={WATT\_SECOND},
  description={Measurement of electrical energy, equal to one Joule},
  kind={units}
}


\newglossaryentry{centipoise}
{
  type=mtc,
  category=code,
  name={CENTIPOISE},
  description={A measure of Viscosity},
  kind={nativeUnits}
}


\newglossaryentry{degreeperminute}
{
  type=mtc,
  category=code,
  name={DEGREE/MINUTE},
  description={Rotational velocity in degrees per minute},
  kind={nativeUnits}
}


\newglossaryentry{fahrenheit}
{
  type=mtc,
  category=code,
  name={FAHRENHEIT},
  description={Temperature in Fahrenheit},
  kind={nativeUnits}
}


\newglossaryentry{foot}
{
  type=mtc,
  category=code,
  name={FOOT},
  description={Feet},
  kind={nativeUnits}
}


\newglossaryentry{footperminute}
{
  type=mtc,
  category=code,
  name={FOOT/MINUTE},
  description={Feet per minute},
  kind={nativeUnits}
}


\newglossaryentry{footpersecond}
{
  type=mtc,
  category=code,
  name={FOOT/SECOND},
  description={Feet per second},
  kind={nativeUnits}
}


\newglossaryentry{footpersecondsquared}
{
  type=mtc,
  category=code,
  name={FOOT/SECOND$^2$},
  description={Acceleration in feet per second squared},
  kind={nativeUnits}
}


\newglossaryentry{foot3d}
{
  type=mtc,
  category=code,
  name={FOOT\_3D},
  description={A point in space identified by X, Y, and Z positions and represented by a space-delimited set of numbers each expressed in feet.},
  kind={nativeUnits}
}


\newglossaryentry{gallonperminute}
{
  type=mtc,
  category=code,
  name={GALLON/MINUTE},
  description={Gallons per minute.},
  kind={nativeUnits}
}


\newglossaryentry{inch}
{
  type=mtc,
  category=code,
  name={INCH},
  description={Inches},
  kind={nativeUnits}
}


\newglossaryentry{inchperminute}
{
  type=mtc,
  category=code,
  name={INCH/MINUTE},
  description={Inches per minute},
  kind={nativeUnits}
}


\newglossaryentry{inchpersecond}
{
  type=mtc,
  category=code,
  name={INCH/SECOND},
  description={Inches per second},
  kind={nativeUnits}
}


\newglossaryentry{inchpersecondsquared}
{
  type=mtc,
  category=code,
  name={INCH/SECOND$^2$},
  description={Acceleration in inches per second squared},
  kind={nativeUnits}
}


\newglossaryentry{inch3d}
{
  type=mtc,
  category=code,
  name={INCH\_3D},
  description={A point in space identified by X, Y, and Z positions and represented by a space-delimited set of numbers each expressed in inches.},
  kind={nativeUnits}
}


\newglossaryentry{inchpound}
{
  type=mtc,
  category=code,
  name={INCH\_POUND},
  description={A measure of torque in inch pounds.},
  kind={nativeUnits}
}


\newglossaryentry{kelvin}
{
  type=mtc,
  category=code,
  name={KELVIN},
  description={A measurement of temperature},
  kind={nativeUnits}
}


\newglossaryentry{kilowatt}
{
  type=mtc,
  category=code,
  name={KILOWATT},
  description={A measurement in kilowatt.},
  kind={nativeUnits}
}


\newglossaryentry{kilowatthour}
{
  type=mtc,
  category=code,
  name={KILOWATT\_HOUR},
  description={Kilowatt hours which is 3.6 mega joules.},
  kind={nativeUnits}
}


\newglossaryentry{liter nativeunits}
{
  type=mtc,
  category=code,
  name={LITER},
  description={Measurement of volume of a fluid},
  kind={nativeUnits}
}


\newglossaryentry{literperminute}
{
  type=mtc,
  category=code,
  name={LITER/MINUTE},
  description={Measurement of rate of flow of a fluid},
  kind={nativeUnits}
}


\newglossaryentry{millimeterperminute}
{
  type=mtc,
  category=code,
  name={MILLIMETER/MINUTE},
  description={Velocity in millimeters per minute},
  kind={nativeUnits}
}


\newglossaryentry{other}
{
  type=mtc,
  category=code,
  name={OTHER},
  description={Unsupported units},
  kind={nativeUnits}
}


\newglossaryentry{pound}
{
  type=mtc,
  category=code,
  name={POUND},
  description={US pounds},
  kind={nativeUnits}
}


\newglossaryentry{poundperinchsquared}
{
  type=mtc,
  category=code,
  name={POUND/INCH$^2$},
  description={Pressure in pounds per square inch (PSI).},
  kind={nativeUnits}
}


\newglossaryentry{radian}
{
  type=mtc,
  category=code,
  name={RADIAN},
  description={Angle in radians},
  kind={nativeUnits}
}


\newglossaryentry{radianperminute}
{
  type=mtc,
  category=code,
  name={RADIAN/MINUTE},
  description={Velocity in radians per minute.},
  kind={nativeUnits}
}


\newglossaryentry{radianpersecond}
{
  type=mtc,
  category=code,
  name={RADIAN/SECOND},
  description={Velocity in radians per second},
  kind={nativeUnits}
}


\newglossaryentry{radianpersecondsquared}
{
  type=mtc,
  category=code,
  name={RADIAN/SECOND$^2$},
  description={Rotational acceleration in radian per second squared},
  kind={nativeUnits}
}


\newglossaryentry{revolutionpersecond}
{
  type=mtc,
  category=code,
  name={REVOLUTION/SECOND},
  description={Rotational velocity in revolution per second},
  kind={nativeUnits}
}


\newglossaryentry{active value}
{
  type=mtc,
  category=code,
  name={ACTIVE},
  description={},
  kind={enum}
}


\newglossaryentry{inactive value}
{
  type=mtc,
  category=code,
  name={INACTIVE},
  description={},
  kind={enum}
}


\newglossaryentry{available value}
{
  type=mtc,
  category=code,
  name={AVAILABLE},
  description={},
  kind={enum}
}


\newglossaryentry{unavailable value}
{
  type=mtc,
  category=code,
  name={UNAVAILABLE},
  description={},
  kind={enum}
}


\newglossaryentry{tandem value}
{
  type=mtc,
  category=code,
  name={TANDEM},
  description={},
  kind={enum}
}


\newglossaryentry{synchronous value}
{
  type=mtc,
  category=code,
  name={SYNCHRONOUS},
  description={},
  kind={enum}
}


\newglossaryentry{master value}
{
  type=mtc,
  category=code,
  name={MASTER},
  description={},
  kind={enum}
}


\newglossaryentry{slave value}
{
  type=mtc,
  category=code,
  name={SLAVE},
  description={},
  kind={enum}
}


\newglossaryentry{home value}
{
  type=mtc,
  category=code,
  name={HOME},
  description={},
  kind={enum}
}


\newglossaryentry{travel value}
{
  type=mtc,
  category=code,
  name={TRAVEL},
  description={},
  kind={enum}
}


\newglossaryentry{parked value}
{
  type=mtc,
  category=code,
  name={PARKED},
  description={},
  kind={enum}
}


\newglossaryentry{stopped value}
{
  type=mtc,
  category=code,
  name={STOPPED},
  description={},
  kind={enum}
}


\newglossaryentry{open value}
{
  type=mtc,
  category=code,
  name={OPEN},
  description={},
  kind={enum}
}


\newglossaryentry{closed value}
{
  type=mtc,
  category=code,
  name={CLOSED},
  description={},
  kind={enum}
}


\newglossaryentry{unlatched value}
{
  type=mtc,
  category=code,
  name={UNLATCHED},
  description={},
  kind={enum}
}


\newglossaryentry{right value}
{
  type=mtc,
  category=code,
  name={RIGHT},
  description={},
  kind={enum}
}


\newglossaryentry{left value}
{
  type=mtc,
  category=code,
  name={LEFT},
  description={},
  kind={enum}
}


\newglossaryentry{transitioning value}
{
  type=mtc,
  category=code,
  name={TRANSITIONING},
  description={},
  kind={enum}
}


\newglossaryentry{on value}
{
  type=mtc,
  category=code,
  name={ON},
  description={},
  kind={enum}
}


\newglossaryentry{off value}
{
  type=mtc,
  category=code,
  name={OFF},
  description={},
  kind={enum}
}


\newglossaryentry{up value}
{
  type=mtc,
  category=code,
  name={UP},
  description={},
  kind={enum}
}


\newglossaryentry{down value}
{
  type=mtc,
  category=code,
  name={DOWN},
  description={},
  kind={enum}
}


\newglossaryentry{automatic value}
{
  type=mtc,
  category=code,
  name={AUTOMATIC},
  description={},
  kind={enum}
}


\newglossaryentry{manual value}
{
  type=mtc,
  category=code,
  name={MANUAL},
  description={},
  kind={enum}
}


\newglossaryentry{manualdatainput value}
{
  type=mtc,
  category=code,
  name={MANUAL\_DATA\_INPUT},
  description={},
  kind={enum}
}


\newglossaryentry{semiautomatic value}
{
  type=mtc,
  category=code,
  name={SEMI\_AUTOMATIC},
  description={},
  kind={enum}
}


\newglossaryentry{edit value}
{
  type=mtc,
  category=code,
  name={EDIT},
  description={},
  kind={enum}
}


\newglossaryentry{clockwise value}
{
  type=mtc,
  category=code,
  name={CLOCKWISE},
  description={},
  kind={enum}
}


\newglossaryentry{counterclockwise value}
{
  type=mtc,
  category=code,
  name={COUNTER\_CLOCKWISE},
  description={},
  kind={enum}
}


\newglossaryentry{positive value}
{
  type=mtc,
  category=code,
  name={POSITIVE},
  description={},
  kind={enum}
}


\newglossaryentry{negative value}
{
  type=mtc,
  category=code,
  name={NEGATIVE},
  description={},
  kind={enum}
}


\newglossaryentry{armed value}
{
  type=mtc,
  category=code,
  name={ARMED},
  description={},
  kind={enum}
}


\newglossaryentry{triggered value}
{
  type=mtc,
  category=code,
  name={TRIGGERED},
  description={},
  kind={enum}
}


\newglossaryentry{ready value}
{
  type=mtc,
  category=code,
  name={READY},
  description={},
  kind={enum}
}


\newglossaryentry{interrupted value}
{
  type=mtc,
  category=code,
  name={INTERRUPTED},
  description={},
  kind={enum}
}


\newglossaryentry{feedhold value}
{
  type=mtc,
  category=code,
  name={FEED\_HOLD},
  description={},
  kind={enum}
}


\newglossaryentry{optionalstop value}
{
  type=mtc,
  category=code,
  name={OPTIONAL\_STOP},
  description={},
  kind={enum}
}


\newglossaryentry{programstopped value}
{
  type=mtc,
  category=code,
  name={PROGRAM\_STOPPED},
  description={},
  kind={enum}
}


\newglossaryentry{programcompleted value}
{
  type=mtc,
  category=code,
  name={PROGRAM\_COMPLETED},
  description={},
  kind={enum}
}


\newglossaryentry{production value}
{
  type=mtc,
  category=code,
  name={PRODUCTION},
  description={},
  kind={enum}
}


\newglossaryentry{setup value}
{
  type=mtc,
  category=code,
  name={SETUP},
  description={},
  kind={enum}
}


\newglossaryentry{teardown value}
{
  type=mtc,
  category=code,
  name={TEARDOWN},
  description={},
  kind={enum}
}


\newglossaryentry{maintenance value}
{
  type=mtc,
  category=code,
  name={MAINTENANCE},
  description={},
  kind={enum}
}


\newglossaryentry{processdevelopment value}
{
  type=mtc,
  category=code,
  name={PROCESS\_DEVELOPMENT},
  description={},
  kind={enum}
}


\newglossaryentry{disabled value}
{
  type=mtc,
  category=code,
  name={DISABLED},
  description={},
  kind={enum}
}


\newglossaryentry{enabled value}
{
  type=mtc,
  category=code,
  name={ENABLED},
  description={},
  kind={enum}
}


\newglossaryentry{independent value}
{
  type=mtc,
  category=code,
  name={INDEPENDENT},
  description={},
  kind={enum}
}


\newglossaryentry{mirror value}
{
  type=mtc,
  category=code,
  name={MIRROR},
  description={},
  kind={enum}
}


\newglossaryentry{yes value}
{
  type=mtc,
  category=code,
  name={YES},
  description={},
  kind={enum}
}


\newglossaryentry{no value}
{
  type=mtc,
  category=code,
  name={NO},
  description={},
  kind={enum}
}


\newglossaryentry{notready value}
{
  type=mtc,
  category=code,
  name={NOT\_READY},
  description={},
  kind={enum}
}


\newglossaryentry{spindle value}
{
  type=mtc,
  category=code,
  name={SPINDLE},
  description={},
  kind={enum}
}


\newglossaryentry{index value}
{
  type=mtc,
  category=code,
  name={INDEX},
  description={},
  kind={enum}
}


\newglossaryentry{contour value}
{
  type=mtc,
  category=code,
  name={CONTOUR},
  description={},
  kind={enum}
}


\newglossaryentry{mtconnectasset}
{
  type=mtc,
  category=code,
  name={MTConnectAsset},
  plural={MTConnectAssets},
  description={}
}


\newglossaryentry{asset information model}
{
  type=mtc,
  name={Asset Information Model},
  description={}
}


\newglossaryentry{assetid}
{
  type=mtc,
  category=code,
  name={assetId},
  description={The unique identifier for the \gls{mtconnect asset}. The identifier \MUST be unique with respect to all other \glspl{asset} in an MTConnect installation. The identifier \SHOULD be globally unique with respect to all other \glspl{asset}.}
}


\newglossaryentry{version mtconnectassets}
{
  type=mtc,
  category=code,
  name={version},
  description={The protocol version number. This is the \gls{major} and \gls{minor} version number of the MTConnect Standard being used. For example, if the version number of the Standard used is \cfont{10.21.33}, the \gls{version mtconnectassets} will be \cfont{10.21}.}
}


\newglossaryentry{major}
{
  type=mtc,
  name={major},
  description={}
}


\newglossaryentry{minor}
{
  type=mtc,
  name={minor},
  description={}
}


\newglossaryentry{creationtime mtconnectassets}
{
  type=mtc,
  category=code,
  name={creationTime},
  description={The time the response was created. }
}


\newglossaryentry{testindicator mtconnectassets}
{
  type=mtc,
  category=code,
  name={testIndicator},
  description={Optional flag that indicates the system is operating in test mode. This data is only for testing and indicates that the data is simulated. }
}


\newglossaryentry{instanceid mtconnectassets}
{
  type=mtc,
  category=code,
  name={instanceId},
  description={A number indicating which invocation of the \gls{agent}. This is used to differentiate between separate instances of the \gls{agent}. This value \MUST have a maximum value of $2^{64}-1$ and \MUST be stored in an unsigned 64-bit integer. }
}


\newglossaryentry{sender mtconnectassets}
{
  type=mtc,
  category=code,
  name={sender},
  description={The \gls{agent} identification information. }
}


\newglossaryentry{assetbuffersize mtconnectassets}
{
  type=mtc,
  category=code,
  name={assetBufferSize},
  description={The maximum number of \glspl{mtconnect asset} that will be retained by the \gls{agent}. The \gls{assetbuffersize mtconnectassets} \MUST be an unsigned positive integer value with a maximum value of $2^{32}-1$. }
}


\newglossaryentry{assetcount mtconnectassets}
{
  type=mtc,
  category=code,
  name={assetCount},
  description={The total number of \glspl{mtconnect asset} in an\gls{agent}. This \MUST be an unsigned positive integer value with a maximum value of $2^{32}-1$. This value \MUSTNOT be greater than \gls{assetbuffersize mtconnectassets}.}
}


\newglossaryentry{asset mtconnectassets}
{
  type= mtc,
  category=code,
  name= {Asset},
  plural= {Assets},
  description= {An abstract XML element. Replaced in the XML document by types of \gls{asset mtconnectassets} elements representing entities that are not pieces of equipment.},
  descriptionplural={XML container that consists of one or more types of \gls{asset mtconnectassets} XML elements. }
}


\newglossaryentry{deviceuuid}
{
  type=mtc,
  category=code,
  name={deviceUuid},
  description={The piece of equipments UUID that supplied this data. This is an optional element references to the UUID attribute given in the \gls{device} element. This can be any series of numbers and letters as defined by the XML type \gls{nmtoken}.}
}


\newglossaryentry{removed}
{
  type=mtc,
  category=code,
  name={removed},
  description={This is an optional attribute that is an indicator that the \gls{mtconnect asset} has been removed from the piece of equipment. If the \gls{asset} is marked as removed,
  it will not be visible to the client application unless the=true parameter is provided in the URL. If this attribute is not present it MUST be assumed to be false. The value is an \cfont{xsi:boolean} type and \MUST be \cfont{true} or \cfont{false}.}
}


\newglossaryentry{asset buffer}
{
  type=mtc,
  name={asset buffer},
  description={}
}


\newglossaryentry{cuttingtool}
{
  type=mtc,
  category=code,
  name={CuttingTool},
  description={}
}


\newglossaryentry{assetchanged event}
{
  type=mtc,
  category=code,
  name={ASSET\_CHANGED},
  elementname={AssetChanged},
  description={The value of the \gls{cdata} for the event \MUST be the \gls{assetid} of the asset that has been added or changed. There will not be a separate message for new assets.},
  kind={type,event},
  facet={\gls{string}}
}


\newglossaryentry{assetremoved event}
{
  type=mtc,
  category=code,
  name={ASSET\_REMOVED},
  elementname={AssetRemoved},
  description={The value of the \gls{cdata} for the event \MUST be the \gls{assetid} of the asset that has been removed. The asset will still be visible if requested with the \gls{includeremoved} parameter as described in the protocol section. When assets are removed they are not moved to the beginning of the most recently modified list. },
  kind={type,event},
  facet={\gls{string}}
}


\newglossaryentry{includeremoved}
{
  type=mtc,
  category=code,
  name={includeRemoved},
  description={}
}


\newglossaryentry{assettype}
{
  type=mtc,
  category=code,
  name={assetType},
  description={}
}


\newglossaryentry{cuttingtoolarchetype}
{
  type=mtc,
  category=code,
  name={CuttingToolArchetype},
  description={}
}


\newglossaryentry{assetid cuttingtool}
{
  type=mtc,
  category=code,
  name={assetId},
  description={The unique identifier of the instance of this tool. This will be the same as the \gls{toolid cuttingtool} and \gls{serialnumber cuttingtool} in most cases. The \gls{assetid cuttingtool} \SHOULD be the combination of the \gls{toolid cuttingtool} and \gls{serialnumber cuttingtool} as in \gls{toolid cuttingtool}. \gls{serialnumber cuttingtool} or an equivalent implementation dependent identification scheme.}
}


\newglossaryentry{serialnumber cuttingtool}
{
  type=mtc,
  category=code,
  name={serialNumber},
  description={The unique identifier for this assembly. This is defined as an XML string type and is implementation dependent.}
}


\newglossaryentry{toolid cuttingtool}
{
  type=mtc,
  category=code,
  name={toolId},
  description={The identifier for a class of Cutting Tools. This is defined as an XML string type and is implementation dependent. }
}


\newglossaryentry{manufacturers cuttingtool}
{
  type=mtc,
  category=code,
  name={manufacturers},
  description={An optional attribute referring to the manufacturer(s) of this Cutting Tool, for this element, this will reference the Tool Item and Adaptive Items specifically. }
}


\newglossaryentry{removed cuttingtool}
{
  type=mtc,
  category=code,
  name={removed},
  description={This is an indicator that the Cutting Tool has been removed from the piece of equipment. }
}


\newglossaryentry{cuttingitem}
{
  type=mtc,
  category=code,
  name={CuttingItem},
  plural={CuttingItems},
  descriptionplural={An optional set of individual Cutting Items.},
  description={}
}


\newglossaryentry{cuttingtooldefinition}
{
  type=mtc,
  category=code,
  name={CuttingToolDefinition},
  description={Reference to an ISO 13399.}
}


\newglossaryentry{information model}
{
  type=mtc,
  name={Information Model},
  plural={Information Models},
  description={}
}


\newglossaryentry{cutterstatus}
{
  type=mtc,
  category=code,
  name={CutterStatus},
  description={The status of this assembly.}
}


\newglossaryentry{status cutterstatus}
{
  type=mtc,
  category=code,
  name={Status},
  description={The status of the Cutting Tool.}
}


\newglossaryentry{toollife}
{
  type=mtc,
  category=code,
  name={ToolLife},
  description={The Cutting Tool life as related to this assembly.}
}


\newglossaryentry{location}
{
  type=mtc,
  category=code,
  name={Location},
  description={The Pot or Spindle this tool currently resides in.}
}


\newglossaryentry{reconditioncount}
{
  type=mtc,
  category=code,
  name={ReconditionCount},
  description={The number of times this cutter has been reconditioned.}
}


\newglossaryentry{cuttingtoollifecycle}
{
  type=mtc,
  category=code,
  name={CuttingToolLifeCycle},
  description={Data regarding the use of this tool.}
}


\newglossaryentry{format cuttingtooldefinition}
{
  type=mtc,
  category=code,
  name={format},
  description={Identifies the expected representation of the enclosed data.}
}


\newglossaryentry{xml format}
{
  type=mtc,
  category=code,
  name={XML},
  description={The default value for the definition. The content will be an XML document.}
}


\newglossaryentry{express format}
{
  type=mtc,
  category=code,
  name={EXPRESS},
  description={The document will confirm to the ISO 10303 Part 21 standard.}
}


\newglossaryentry{text format}
{
  type=mtc,
  category=code,
  name={TEXT},
  description={The document will be a text representation of the tool data.}
}


\newglossaryentry{undefined format}
{
  type=mtc,
  category=code,
  name={UNDEFINED},
  description={The document will be provided in an undefined format.}
}


\newglossaryentry{cuttingtooldefinition deprecated}
{
  type=mtc,
  category=code,
  name=\deprecated{CuttingToolDefinition},
  description={\DEPRECATED for \gls{cuttingtool} in Version 1.3.0.   \newline \deprecated{Reference to an ISO 13399.}}
}


\newglossaryentry{cuttingtoolarchetypereference}
{
  type=mtc,
  category=code,
  name={CuttingToolArchetypeReference},
  description={The content of this XML element is the \gls{assetid cuttingtool} of the \gls{cuttingtoolarchetype} document. It \MAY also contain a source attribute that gives the URL of the archetype data as well.}
}


\newglossaryentry{new status}
{
  type=mtc,
  category=code,
  name={NEW},
  description={A new tool that has not been used or first use. Marks the start of the tool history.}
}


\newglossaryentry{available status}
{
  type=mtc,
  category=code,
  name={AVAILABLE},
  description={Indicates the tool is available for use. If this is not present, the tool is currently not ready to be used.}
}


\newglossaryentry{unavailable status}
{
  type=mtc,
  category=code,
  name={UNAVAILABLE},
  description={Indicates the tool is unavailable for use in metal removal. If this is not present, the tool is currently not ready to be used.}
}


\newglossaryentry{allocated status}
{
  type=mtc,
  category=code,
  name={ALLOCATED},
  description={Indicates if this tool is has been committed to a piece of equipment for use and is not available for use in any other piece of equipment. If this is not present, this tool has not been allocated for this piece of equipment and can be used by another piece of equipment.}
}


\newglossaryentry{unallocated status}
{
  type=mtc,
  category=code,
  name={UNALLOCATED},
  description={Indicates this Cutting Tool has not been committed to a process and can be allocated.}
}


\newglossaryentry{measured status}
{
  type=mtc,
  category=code,
  name={MEASURED},
  description={The tool has been measured.}
}


\newglossaryentry{reconditioned status}
{
  type=mtc,
  category=code,
  name={RECONDITIONED},
  description={The Cutting Tool has been reconditioned. See \gls{reconditioncount} for the number of times this cutter has been reconditioned.}
}


\newglossaryentry{used status}
{
  type=mtc,
  category=code,
  name={USED},
  description={The Cutting Tool is in process and has remaining tool life.}
}


\newglossaryentry{expired status}
{
  type=mtc,
  category=code,
  name={EXPIRED},
  description={The Cutting Tool has reached the end of its useful life.}
}


\newglossaryentry{broken status}
{
  type=mtc,
  category=code,
  name={BROKEN},
  description={Premature tool failure.}
}


\newglossaryentry{notregistered status}
{
  type=mtc,
  category=code,
  name={NOT\_REGISTERED},
  description={This Cutting Tool cannot be used until it is entered into the system.}
}


\newglossaryentry{unknown status}
{
  type=mtc,
  category=code,
  name={UNKNOWN},
  description={The Cutting Tool is an indeterminate state. This is the default value.}
}


\newglossaryentry{countdirection}
{
  type=mtc,
  category=code,
  name={countDirection},
  description={Indicates if the tool life counts from zero to maximum or maximum to zero.}
}


\newglossaryentry{warning toollife}
{
  type=mtc,
  category=code,
  name={warning},
  description={The point at which a tool life warning will be raised.}
}


\newglossaryentry{limit}
{
  type=mtc,
  category=code,
  name={limit},
  description={The end of life limit for this tool.}
}


\newglossaryentry{initial}
{
  type=mtc,
  category=code,
  name={initial},
  description={The initial life of the tool when it is new.}
}


\newglossaryentry{minutes type}
{
  type=mtc,
  category=code,
  name={MINUTES},
  description={The tool life measured in minutes. All units for minimum, maximum, and nominal \MUST be provided in minutes.}
}


\newglossaryentry{partcount type}
{
  type=mtc,
  category=code,
  name={PART\_COUNT},
  description={The tool life measured in parts. All units for minimum, maximum, and nominal \MUST be provided as the number of parts.}
}


\newglossaryentry{wear type}
{
  type=mtc,
  category=code,
  name={WEAR},
  description={The tool life measured in tool wear. Wear \MUST be provided in millimeters as an offset to nominal. All units for minimum, maximum, and nominal \MUST be given as millimeter offsets as well. }
}


\newglossaryentry{up countdirection}
{
  type=mtc,
  category=code,
  name={UP},
  description={The tool life counts up from zero to the maximum.}
}


\newglossaryentry{down countdirection}
{
  type=mtc,
  category=code,
  name={DOWN},
  description={The tool life counts down from the maximum to zero.}
}


\newglossaryentry{positiveoverlap}
{
  type=mtc,
  category=code,
  name={positiveOverlap},
  description={The number of locations at higher index value from this location.}
}


\newglossaryentry{negativeoverlap}
{
  type=mtc,
  category=code,
  name={negativeOverlap},
  description={The number of location at lower index values from this location.}
}


\newglossaryentry{pot type}
{
  type=mtc,
  category=code,
  name={POT},
  description={The number of the pot in the tool handling system.}
}


\newglossaryentry{station type}
{
  type=mtc,
  category=code,
  name={STATION},
  description={The tool location in a horizontal turning machine.}
}


\newglossaryentry{crib type}
{
  type=mtc,
  category=code,
  name={CRIB},
  description={The location with regard to a tool crib.}
}


\newglossaryentry{maximumcount}
{
  type=mtc,
  category=code,
  name={maximumCount},
  description={The maximum number of times this tool may be reconditioned.}
}


\newglossaryentry{source cuttingtoolarchetypereference}
{
  type=mtc,
  category=code,
  name={Source},
  description={The URL of the \gls{cuttingtoolarchetype} \gls{information model}.}
}


\newglossaryentry{programtoolgroup}
{
  type=mtc,
  category=code,
  name={ProgramToolGroup},
  description={The tool group this tool is assigned in the part program.}
}


\newglossaryentry{programtoolnumber}
{
  type=mtc,
  category=code,
  name={ProgramToolNumber},
  description={The number of the tool as referenced in the part program.}
}


\newglossaryentry{processspindlespeed}
{
  type=mtc,
  category=code,
  name={ProcessSpindleSpeed},
  description={The constrained process spindle speed for this tool.}
}


\newglossaryentry{processfeedrate}
{
  type=mtc,
  category=code,
  name={ProcessFeedRate},
  description={The constrained process feed rate for this tool in mm/s.}
}


\newglossaryentry{connectioncodemachineside}
{
  type=mtc,
  category=code,
  name={ConnectionCodeMachineSide},
  description={Identifier for the capability to connect any component of the Cutting Tool together, except Assembly Items, on the machine side. Code: \cfont{CCMS}}
}


\newglossaryentry{measurement}
{
  type=mtc,
  category=code,
  name={Measurement},
  plural={Measurements},
  description={},
  descriptionplural={A collection of measurements for the tool assembly.}
}


\newglossaryentry{xs:any}
{
  type=mtc,
  category=code,
  name={xs:any},
  description={Any additional properties not in the current document model. \MUST be in separate XML namespace.}
}


\newglossaryentry{maximum processspindlespeed}
{
  type=mtc,
  category=code,
  name={maximum},
  description={The upper bound for the tools target spindle speed.}
}


\newglossaryentry{minimum processspindlespeed}
{
  type=mtc,
  category=code,
  name={minimum},
  description={The lower bound for the tools spindle speed.}
}


\newglossaryentry{nominal processspindlespeed}
{
  type=mtc,
  category=code,
  name={nominal},
  description={The nominal speed the tool is designed to operate at.}
}


\newglossaryentry{maximum processfeedrate}
{
  type=mtc,
  category=code,
  name={maximum},
  description={The upper bound for the tools process target feedrate.}
}


\newglossaryentry{minimum processfeedrate}
{
  type=mtc,
  category=code,
  name={minimum},
  description={The lower bound for the tools feedrate.}
}


\newglossaryentry{nominal processfeedrate}
{
  type=mtc,
  category=code,
  name={nominal},
  description={The nominal feedrate the tool is designed to operate at.}
}


\newglossaryentry{commonmeasurement}
{
  type=mtc,
  category=code,
  name={CommonMeasurement},
  description={}
}


\newglossaryentry{assemblymeasurement}
{
  type=mtc,
  category=code,
  name={AssemblyMeasurement},
  description={}
}


\newglossaryentry{cuttingitemmeasurement}
{
  type=mtc,
  category=code,
  name={CuttingItemMeasurement},
  description={}
}


\newglossaryentry{code measurement}
{
  type=mtc,
  category=code,
  name={code},
  description={A shop specific code for this measurement. ISO 13399 codes \MAY be used for these codes as well.}
}


\newglossaryentry{maximum measurement}
{
  type=mtc,
  category=code,
  name={maximum},
  description={The maximum value for this measurement. Exceeding this value would indicate the tool is not usable.}
}


\newglossaryentry{minimum measurement}
{
  type=mtc,
  category=code,
  name={minimum},
  description={The minimum value for this measurement. Exceeding this value would indicate the tool is not usable.}
}


\newglossaryentry{nominal measurement}
{
  type=mtc,
  category=code,
  name={nominal},
  description={The as advertised value for this measurement.}
}


\newglossaryentry{significantdigits measurement}
{
  type=mtc,
  category=code,
  name={significantDigits},
  description={The number of significant digits in the reported value. This is used by applications to determine accuracy of values. This \MAY be specified for all numeric values.}
}


\newglossaryentry{units measurement}
{
  type=mtc,
  category=code,
  name={units},
  description={The units for the measurements. }
}


\newglossaryentry{nativeunits measurement}
{
  type=mtc,
  category=code,
  name={nativeunits},
  description={The units the measurement was originally recorded in. }
}


\newglossaryentry{bodydiametermax}
{
  type=mtc,
  category=code,
  name={BodyDiameterMax},
  code=\cfont{BDX},
  description={The largest diameter of the body of a Tool Item. },
  units=\cfont{\gls{millimeter}}
}


\newglossaryentry{bodylengthmax}
{
  type=mtc,
  category=code,
  name={BodyLengthMax},
  code=\cfont{LBX},
  description={The distance measured along the X axis from that point of the item closest to the workpiece, including the Cutting Item for a Tool Item but excluding a protruding locking mechanism for an Adaptive Item, to either the front of the flange on a flanged body or the beginning of the connection interface feature on the machine side for cylindrical or prismatic shanks.},
  units=\cfont{\gls{millimeter}}
}


\newglossaryentry{depthofcutmax}
{
  type=mtc,
  category=code,
  name={DepthOfCutMax},
  code=\cfont{APMX},
  description={The maximum engagement of the cutting edge or edges with the workpiece measured perpendicular to the feed motion. },
  units=\cfont{\gls{millimeter}}
}


\newglossaryentry{cuttingdiametermax}
{
  type=mtc,
  category=code,
  name={CuttingDiameterMax},
  code=\cfont{DC},
  description={The maximum diameter of a circle on which the defined point Pk of each of the master inserts is located on a Tool Item. The normal of the machined peripheral surface points towards the axis of the Cutting Tool. },
  units=\cfont{\gls{millimeter}}
}


\newglossaryentry{flangediametermax}
{
  type=mtc,
  category=code,
  name={FlangeDiameterMax},
  code=\cfont{DF},
  description={The dimension between two parallel tangents on the outside edge of a flange. },
  units=\cfont{\gls{millimeter}}
}


\newglossaryentry{overalltoollength}
{
  type=mtc,
  category=code,
  name={OverallToolLength},
  code=\cfont{OAL},
  description={The largest length dimension of the Cutting Tool including the master insert where applicable.  },
  units=\cfont{\gls{millimeter}}
}


\newglossaryentry{shankdiameter}
{
  type=mtc,
  category=code,
  name={ShankDiameter},
  code=\cfont{DMM},
  description={The dimension of the diameter of a cylindrical portion of a Tool Item or an Adaptive Item that can participate in a connection. },
  units=\cfont{\gls{millimeter}}
}


\newglossaryentry{shankheight}
{
  type=mtc,
  category=code,
  name={ShankHeight},
  code=\cfont{H},
  description={The dimension of the height of the shank. },
  units=\cfont{\gls{millimeter}}
}


\newglossaryentry{shanklength}
{
  type=mtc,
  category=code,
  name={ShankLength},
  code=\cfont{LS},
  description={The dimension of the length of the shank. },
  units=\cfont{\gls{millimeter}}
}


\newglossaryentry{usablelengthmax}
{
  type=mtc,
  category=code,
  name={UsableLengthMax},
  code=\cfont{LUX},
  description={Maximum length of a Cutting Tool that can be used in a particular cutting operation including the non-cutting portions of the tool.},
  units=\cfont{\gls{millimeter}}
}


\newglossaryentry{protrudinglength}
{
  type=mtc,
  category=code,
  name={ProtrudingLength},
  code=\cfont{LPR},
  description={The dimension from the yz-plane to the furthest point of the Tool Item or Adaptive Item measured in the -X direction. },
  units=\cfont{\gls{millimeter}}
}


\newglossaryentry{weight}
{
  type=mtc,
  category=code,
  name={Weight},
  code=\cfont{WT},
  description={The total weight of the Cutting Tool in grams. The force exerted by the mass of the Cutting Tool. },
  units=\cfont{GRAM}
}


\newglossaryentry{functionallength}
{
  type=mtc,
  category=code,
  name={FunctionalLength},
  code=\cfont{LF},
  description={The distance from the gauge plane or from the end of the shank to the furthest point on the tool, if a gauge plane does not exist, to the cutting reference point determined by the main function of the tool. The \gls{cuttingtool} functional length will be the length of the entire tool, not a single Cutting Item. Each \gls{cuttingitem} can have an independent \gls{functionallength} represented in its measurements. },
  units=\cfont{\gls{millimeter}}
}


\newglossaryentry{count cuttingitems}
{
  type=mtc,
  category=code,
  name={count},
  description={The number of Cutting Items. }
}


\newglossaryentry{indices cuttingitem}
{
  type=mtc,
  category=code,
  name={indices},
  description={The number or numbers representing the individual Cutting Item or items on the tool. }
}


\newglossaryentry{itemid cuttingitem}
{
  type=mtc,
  category=code,
  name={itemId},
  description={The manufacturer identifier of this Cutting Item. }
}


\newglossaryentry{manufacturers cuttingitem}
{
  type=mtc,
  category=code,
  name={manufacturers},
  description={The manufacturers of the Cutting Item. }
}


\newglossaryentry{grade cuttingitem}
{
  type=mtc,
  category=code,
  name={grade},
  description={The material composition for this Cutting Item.}
}


\newglossaryentry{measurement cuttingitem}
{
  type=mtc,
  category=code,
  name={Measurement},
  plural={Measurements},
  description={},
  descriptionplural={A collection of measurements relating to this Cutting Item.}
}


\newglossaryentry{locus cuttingitem}
{
  type=mtc,
  category=code,
  name={Locus},
  description={A free form description of the location on the Cutting Tool.}
}


\newglossaryentry{itemlife cuttingitem}
{
  type=mtc,
  category=code,
  name={ItemLife},
  description={The life of this Cutting Item.}
}


\newglossaryentry{cuttingreferencepoint}
{
  type=mtc,
  category=code,
  name={CuttingReferencePoint},
  code=\cfont{CRP},
  description={The theoretical sharp point of the Cutting Tool from which the major functional dimensions are taken. },
  units=\cfont{\gls{millimeter}}
}


\newglossaryentry{cuttingedgelength}
{
  type=mtc,
  category=code,
  name={CuttingEdgeLength},
  code=\cfont{L},
  description={The theoretical length of the cutting edge of a Cutting Item over sharp corners.},
  units=\cfont{\gls{millimeter}}
}


\newglossaryentry{driveangle}
{
  type=mtc,
  category=code,
  name={DriveAngle},
  code=\cfont{DRVA},
  description={Angle between the driving mechanism locator on a Tool Item and the main cutting edge. },
  units=\cfont{\gls{degree}}
}


\newglossaryentry{flangediameter}
{
  type=mtc,
  category=code,
  name={FlangeDiameter},
  code=\cfont{DF},
  description={The dimension between two parallel tangents on the outside edge of a flange. },
  units=\cfont{\gls{millimeter}}
}


\newglossaryentry{functionalwidth}
{
  type=mtc,
  category=code,
  name={FunctionalWidth},
  code=\cfont{WF},
  description={The distance between the cutting reference point and the rear backing surface of a turning tool or the axis of a boring bar.},
  units=\cfont{\gls{millimeter}}
}


\newglossaryentry{incribedcirclediameter}
{
  type=mtc,
  category=code,
  name={IncribedCircleDiameter},
  code=\cfont{IC},
  description={The diameter of a circle to which all edges of a equilateral and round regular insert are tangential. },
  units=\cfont{\gls{millimeter}}
}


\newglossaryentry{pointangle}
{
  type=mtc,
  category=code,
  name={PointAngle},
  code=\cfont{SIG},
  description={The angle between the major cutting edge and the same cutting edge rotated by 180 degrees about the tool axis.},
  units=\cfont{\gls{degree}}
}


\newglossaryentry{toolcuttingedgeangle}
{
  type=mtc,
  category=code,
  name={ToolCuttingEdgeAngle},
  code=\cfont{KAPR},
  description={The angle between the tool cutting edge plane and the tool feed plane measured in a plane parallel the xy-plane. },
  units=\cfont{\gls{degree}}
}


\newglossaryentry{toolleadangle}
{
  type=mtc,
  category=code,
  name={ToolLeadAngle},
  code=\cfont{PSIR},
  description={The angle between the tool cutting edge plane and a plane perpendicular to the tool feed plane measured in a plane parallel the xy-plane. },
  units=\cfont{\gls{degree}}
}


\newglossaryentry{toolorientation}
{
  type=mtc,
  category=code,
  name={ToolOrientation},
  code=\cfont{N/A},
  description={The angle of the tool with respect to the workpiece for a given process. The value is application specific. },
  units=\cfont{\gls{degree}}
}


\newglossaryentry{wiperedgelength}
{
  type=mtc,
  category=code,
  name={WiperEdgeLength},
  code=\cfont{BS},
  description={The measure of the length of a wiper edge of a Cutting Item.},
  units=\cfont{\gls{millimeter}}
}


\newglossaryentry{stepdiameterlength}
{
  type=mtc,
  category=code,
  name={StepDiameterLength},
  code=\cfont{SDLx},
  description={The length of a portion of a stepped tool that is related to a corresponding cutting diameter measured from the cutting reference point of that cutting diameter to the point on the next cutting edge at which the diameter starts to change.},
  units=\cfont{\gls{millimeter}}
}


\newglossaryentry{stepincludedangle}
{
  type=mtc,
  category=code,
  name={StepIncludedAngle},
  code=\cfont{STAx},
  description={The angle between a major edge on a step of a stepped tool and the same cutting edge rotated 180 degrees about its tool axis.},
  units=\cfont{\gls{degree}}
}


\newglossaryentry{cuttingdiameter}
{
  type=mtc,
  category=code,
  name={CuttingDiameter},
  code=\cfont{DCx},
  description={The diameter of a circle on which the defined point Pk located on this Cutting Tool. The normal of the machined peripheral surface points towards the axis of the Cutting Tool.},
  units=\cfont{\gls{millimeter}}
}


\newglossaryentry{cuttingheight}
{
  type=mtc,
  category=code,
  name={CuttingHeight},
  code=\cfont{HF},
  description={The distance from the basal plane of the Tool Item to the cutting point. },
  units=\cfont{\gls{millimeter}}
}


\newglossaryentry{cornerradius}
{
  type=mtc,
  category=code,
  name={CornerRadius},
  code=\cfont{RE},
  description={The nominal radius of a rounded corner measured in the X Y-plane. },
  units=\cfont{\gls{millimeter}}
}


\newglossaryentry{functionallength cuttingitem}
{
  type=mtc,
  category=code,
  name={FunctionalLength},
  code=\cfont{LFx},
  description={The distance from the gauge plane or from the end of the shank of the Cutting Tool, if a gauge plane does not exist, to the cutting reference point determined by the main function of the tool. This measurement will be with reference to the Cutting Tool and \MUSTNOT exist without a Cutting Tool.},
  units=\cfont{\gls{millimeter}}
}


\newglossaryentry{chamferflatlength}
{
  type=mtc,
  category=code,
  name={ChamferFlatLength},
  code=\cfont{BCH},
  description={The flat length of a chamfer. },
  units=\cfont{\gls{millimeter}}
}


\newglossaryentry{chamferwidth}
{
  type=mtc,
  category=code,
  name={ChamferWidth},
  code=\cfont{CHW},
  description={The width of the chamfer. },
  units=\cfont{\gls{millimeter}}
}


\newglossaryentry{insertwidth}
{
  type=mtc,
  category=code,
  name={InsertWidth},
  code=\cfont{W1},
  description={W1 is used for the insert width when an inscribed circle diameter is not practical.},
  units=\cfont{\gls{millimeter}}
}


\newglossaryentry{interaction model}
{
  type=mtc,
  name={Interaction Model},
  description={}
}


\newglossaryentry{publish}
{
  type=mtc,
  name={Publish},
  description={}
}


\newglossaryentry{subscribe}
{
  type=mtc,
  name={Subscribe},
  description={}
}


\newglossaryentry{requester}
{
  type=mtc,
  name={Requester},
  description={}
}


\newglossaryentry{responder}
{
  type=mtc,
  name={Responder},
  description={}
}


\newglossaryentry{fail value}
{
  type=mtc,
  category=code,
  name={FAIL},
  description={},
  kind={enum}
}


\newglossaryentry{complete value}
{
  type=mtc,
  category=code,
  name={COMPLETE},
  description={},
  kind={enum}
}


\newglossaryentry{doorinterface}
{
  type=mtc,
  category=code,
  name={DoorInterface},
  description={\gls{doorinterface} provides the set of information used to coordinate the operations between two pieces of equipment, one of which controls the operation of a door. }
}


\newglossaryentry{chuckinterface}
{
  type=mtc,
  category=code,
  name={ChuckInterface},
  description={\gls{chuckinterface} provides the set of information used to coordinate the operations between two pieces of equipment, one of which controls the operation of a chuck.  }
}


\newglossaryentry{barfeederinterface}
{
  type=mtc,
  category=code,
  name={BarFeederInterface},
  description={\gls{barfeederinterface} provides the set of information used to coordinate the operations between a Bar Feeder and another piece of equipment.  }
}


\newglossaryentry{materialhandlerinterface}
{
  type=mtc,
  category=code,
  name={MaterialHandlerInterface},
  description={\gls{materialhandlerinterface} provides the set of information used to coordinate the operations between a piece of equipment and another associated piece of equipment used to automatically handle various types of materials or services associated with the original piece of equipment. }
}


\newglossaryentry{request subtype interface}
{
  type=mtc,
  category=code,
  name={REQUEST},
  description={}
}


\newglossaryentry{response subtype interface}
{
  type=mtc,
  category=code,
  name={RESPONSE},
  description={}
}


\newglossaryentry{opendoor event}
{
  type=mtc,
  category=code,
  name={OPEN\_DOOR},
  elementname=\cfont{OpenDoor},
  description={Service to open a door. },
  kind={type,event},
  facet={\gls{string}}
}


\newglossaryentry{closedoor event}
{
  type=mtc,
  category=code,
  name={CLOSE\_DOOR},
  elementname=\cfont{CloseDoor},
  description={Service to close a door.},
  kind={type,event},
  facet={\gls{string}}
}


\newglossaryentry{openchuck event}
{
  type=mtc,
  category=code,
  name={OPEN\_CHUCK},
  elementname=\cfont{OpenChuck},
  description={Service to open a chuck. },
  kind={type,event},
  facet={\gls{string}}
}


\newglossaryentry{closechuck event}
{
  type=mtc,
  category=code,
  name={CLOSE\_CHUCK},
  elementname=\cfont{CloseChuck},
  description={Service to close a chuck.},
  kind={type,event},
  facet={\gls{string}}
}


\newglossaryentry{materialfeed event}
{
  type=mtc,
  category=code,
  name={MATERIAL\_FEED},
  elementname=\cfont{MaterialFeed},
  description={Service to advance material or feed product to a piece of equipment from a continuous or bulk source. },
  kind={type,event},
  facet={\gls{string}}
}


\newglossaryentry{materialchange event}
{
  type=mtc,
  category=code,
  name={MATERIAL\_CHANGE},
  elementname=\cfont{MaterialChange},
  description={Service to change the type of material or product being loaded or fed to a piece of equipment.},
  kind={type,event},
  facet={\gls{string}}
}


\newglossaryentry{materialretract event}
{
  type=mtc,
  category=code,
  name={MATERIAL\_RETRACT},
  elementname=\cfont{MaterialRetract},
  description={Service to remove or retract material or product.},
  kind={type,event},
  facet={\gls{string}}
}


\newglossaryentry{partchange event}
{
  type=mtc,
  category=code,
  name={PART\_CHANGE},
  elementname=\cfont{PartChange},
  description={Service to change the part or product associated with a piece of equipment to a different part or product.  },
  kind={type,event},
  facet={\gls{string}}
}


\newglossaryentry{materialload event}
{
  type=mtc,
  category=code,
  name={MATERIAL\_LOAD},
  elementname=\cfont{MaterialLoad},
  description={Service to load a piece of material or product.},
  kind={type,event},
  facet={\gls{string}}
}


\newglossaryentry{materialunload event}
{
  type=mtc,
  category=code,
  name={MATERIAL\_UNLOAD},
  elementname=\cfont{MaterialUnload},
  description={Service to unload a piece of material or product.},
  kind={type,event},
  facet={\gls{string}}
}


\newglossaryentry{integer}
{
  name={integer},
  description={},
  kind={facet}
}


\newglossaryentry{string}
{
  name={string},
  description={},
  kind={facet}
}


\newglossaryentry{float}
{
  name={float},
  description={},
  kind={facet}
}


\newglossaryentry{boolean}
{
  name={boolean},
  description={},
  kind={facet}
}


\newglossaryentry{datetime}
{
  name={datetime},
  description={},
  kind={facet}
}


\newglossaryentry{url}
{
  name={url},
  description={},
  kind={facet}
}


\newglossaryentry{array3d}
{
  name={array3d},
  description={},
  kind={facet},
  facet={\gls{float}}
}

\newglossaryentry{arraystring}
{
  name={arraystring},
  description={},
  kind={facet},
  facet={\gls{string}}
}
